\documentclass[envcountchap,graybox]{svmono}

\addtolength{\textwidth}{1mm}

\usepackage{amsmath,amssymb}

\usepackage{amsfonts}
%\usepackage{breqn}
\usepackage{DLMFmath}
\usepackage{DRMFfcns}

\usepackage{mathptmx}
\usepackage{helvet}
\usepackage{courier}

\usepackage{makeidx}
\usepackage{graphicx}

\usepackage{multicol}
\usepackage[bottom]{footmisc}

\usepackage[pdftex]{hyperref} 
\usepackage {xparse} 
\usepackage{cite} 
\makeindex

\def\bibname{Bibliography}
\def\refname{Bibliography}

\def\theequation{\thesection.\arabic{equation}}

\smartqed

\let\corollary=\undefined
\spnewtheorem{corollary}[theorem]{Corollary}{\bfseries}{\itshape}

\newcounter{rom}

\newcommand{\hyp}[5]{\mbox{}_{#1}{F}_{#2}
\left(\genfrac{}{}{0pt}{}{#3}{#4}\,;\,#5\right)}

\newcommand{\qhyp}[5]{\mbox{}_{#1}{\phi}_{#2}
\left(\genfrac{}{}{0pt}{}{#3}{#4}\,;\,q,\,#5\right)}

\newcommand{\mathindent}{\hspace{7.5mm}}

\newcommand{\e}{\textrm{e}}

\renewcommand{\E}{\textrm{E}}

\renewcommand{\textfraction}{-1}

\renewcommand{\Gamma}{\varGamma}

\renewcommand{\leftlegendglue}{\hfil}

\settowidth{\tocchpnum}{14\enspace}
\settowidth{\tocsecnum}{14.30\enspace}
\settowidth{\tocsubsecnum}{14.12.1\enspace}

\makeatletter
\def\cleartoversopage{\clearpage\if@twoside\ifodd\c@page
         \hbox{}\newpage\if@twocolumn\hbox{}\newpage\fi
         \else\fi\fi}

\newcommand{\clearemptyversopage}{
        \clearpage{\pagestyle{empty}\cleartoversopage}}
\makeatother

\oddsidemargin -1.5cm
\topmargin -2.0cm
\textwidth 16.3cm
\textheight 25cm

\newcommand\sa{\smallskipamount}
\newcommand\sLP{\\[\sa]}
\newcommand\sPP{\\[\sa]\indent}
\newcommand\ba{\bigskipamount}
\newcommand\bLP{\\[\ba]}
\newcommand\CC{\mathbb{C}}
\newcommand\RR{\mathbb{R}}
\newcommand\ZZ{\mathbb{Z}}
\newcommand\al\alpha
\newcommand\be\beta
\newcommand\ga\gamma
\newcommand\de\delta
\newcommand\tha\theta
\newcommand\la\lambda
\newcommand\om\omega
\newcommand\Ga{\Gamma}
\newcommand\half{\frac12}
\newcommand\thalf{\tfrac12}
\newcommand\iy\infty
\newcommand\wt{\widetilde}
\newcommand\const{{\rm const.}\,}
\newcommand\Zpos{\ZZ_{>0}}
\newcommand\Znonneg{\ZZ_{\ge0}}
\newcommand{\hyp}[5]{\,\mbox{}_{#1}F_{#2}\!\left(
  \genfrac{}{}{0pt}{}{#3}{#4};#5\right)}
\newcommand{\qhyp}[5]{\,\mbox{}_{#1}\phi_{#2}\!\left(
  \genfrac{}{}{0pt}{}{#3}{#4};#5\right)}
\newcommand\LHS{left-hand side}
\newcommand\RHS{right-hand side}
\renewcommand\Re{{\rm Re}\,}
\renewcommand\Im{{\rm Im}\,}

\NewDocumentCommand\mycite{m g}{%
  \IfNoValueTF{#2}
    {[\hyperlink{#1}{#1}]}
    {[\hyperlink{#1}{#1}, #2]}%
}
\newcommand\mybibitem[1]{\bibitem[#1]{#1}\hypertarget{#1}{}}
\begin{document}

\author{Roelof Koekoek\\[2.5mm]Peter A. Lesky\\[2.5mm]Ren\'e F. Swarttouw}
\title{Hypergeometric orthogonal polynomials and their\\$q$-analogues}
\subtitle{-- Monograph --}
\maketitle

\frontmatter

\large

\addtocounter{chapter}{8}
\pagenumbering{roman}
\chapter{Hypergeometric orthogonal polynomials}
\label{HyperOrtPol}


In this chapter we deal with all families of hypergeometric orthogonal polynomials
appearing in the Askey scheme on page~\pageref{scheme}. For each family of orthogonal
polynomials we state the most important properties such as a representation as a
hypergeometric function, orthogonality relation(s), the three-term recurrence relation,
the second-order differential or difference equation, the forward shift (or degree lowering)
and backward shift (or degree raising) operator, a Rodrigues-type formula and some generating
functions. In each case we use the notation which seems to be most common in the literature.
Moreover, in each case we mention the connection between various families by stating the
appropriate limit relations. See also \cite{Terwilliger2006} for an algebraic approach of
this Askey scheme and \cite{TemmeLopez2001} for a view from asymptotic analysis.
For notations the reader is referred to chapter~\ref{Definitions}.

\section{Wilson}\index{Wilson polynomials}

\par

\subsection*{Hypergeometric representation}
\begin{eqnarray}
\label{DefWilson}
& &\frac{W_n(x^2;a,b,c,d)}{(a+b)_n(a+c)_n(a+d)_n}\nonumber\\
& &{}=\hyp{4}{3}{-n,n+a+b+c+d-1,a+ix,a-ix}{a+b,a+c,a+d}{1}.
\end{eqnarray}

\newpage

\subsection*{Orthogonality relation}
If $Re(a,b,c,d)>0$ and non-real parameters occur in conjugate pairs, then
\begin{eqnarray}
\label{OrthIWilson}
& &\frac{1}{2\pi}\int_0^{\infty}
\left|\frac{\Gamma(a+ix)\Gamma(b+ix)\Gamma(c+ix)\Gamma(d+ix)}{\Gamma(2ix)}\right|^2\nonumber\\
& &{}\mathindent\times W_m(x^2;a,b,c,d)W_n(x^2;a,b,c,d)\,dx\nonumber\\
& &{}=\frac{\Gamma(n+a+b)\cdots\Gamma(n+c+d)}{\Gamma(2n+a+b+c+d)}(n+a+b+c+d-1)_nn!\,\delta_{mn},
\end{eqnarray}
where
\begin{eqnarray*}
& &\Gamma(n+a+b)\cdots\Gamma(n+c+d)\\
& &{}=\Gamma(n+a+b)\Gamma(n+a+c)\Gamma(n+a+d)\Gamma(n+b+c)\Gamma(n+b+d)\Gamma(n+c+d).
\end{eqnarray*}
If $a<0$ and $a+b$, $a+c$, $a+d$ are positive or a pair of complex conjugates
occur with positive real parts, then
\begin{eqnarray}
\label{OrtIIWilson}
& &\frac{1}{2\pi}\int_0^{\infty}\left|\frac{\Gamma(a+ix)\Gamma(b+ix)\Gamma(c+ix)\Gamma(d+ix)}{\Gamma(2ix)}\right|^2\nonumber\\
& &{}\mathindent\mathindent\times W_m(x^2;a,b,c,d)W_n(x^2;a,b,c,d)\,dx\nonumber\\
& &{}\mathindent{}+\frac{\Gamma(a+b)\Gamma(a+c)\Gamma(a+d)\Gamma(b-a)\Gamma(c-a)\Gamma(d-a)}{\Gamma(-2a)}\nonumber\\
& &{}\mathindent\mathindent{}\times\sum_{\begin{array}{c}
{\scriptstyle k=0,1,2\ldots}\\{\scriptstyle a+k<0}\end{array}}
\frac{(2a)_k(a+1)_k(a+b)_k(a+c)_k(a+d)_k}{(a)_k(a-b+1)_k(a-c+1)_k(a-d+1)_kk!}\nonumber\\
& &{}\mathindent\mathindent\mathindent\mathindent{}\times W_m(-(a+k)^2;a,b,c,d)W_n(-(a+k)^2;a,b,c,d)\nonumber\\
& &{}=\frac{\Gamma(n+a+b)\cdots\Gamma(n+c+d)}{\Gamma(2n+a+b+c+d)}(n+a+b+c+d-1)_nn!\,\delta_{mn}.
\end{eqnarray}

\subsection*{Recurrence relation}
\begin{equation}
\label{RecWilson}
-\left(a^2+x^2\right){\tilde{W}}_n(x^2)
=A_n{\tilde{W}}_{n+1}(x^2)-\left(A_n+C_n\right){\tilde{W}}_n(x^2)+C_n{\tilde{W}}_{n-1}(x^2),
\end{equation}
where
$${\tilde{W}}_n(x^2):={\tilde{W}}_n(x^2;a,b,c,d)=\frac{W_n(x^2;a,b,c,d)}{(a+b)_n(a+c)_n(a+d)_n}$$
and
$$\left\{\begin{array}{l}
\displaystyle A_n=\frac{(n+a+b+c+d-1)(n+a+b)(n+a+c)(n+a+d)}{(2n+a+b+c+d-1)(2n+a+b+c+d)}\\[5mm]
\displaystyle C_n=\frac{n(n+b+c-1)(n+b+d-1)(n+c+d-1)}{(2n+a+b+c+d-2)(2n+a+b+c+d-1)}.
\end{array}\right.$$

\subsection*{Normalized recurrence relation}
\begin{equation}
\label{NormRecWilson}
xp_n(x)=p_{n+1}(x)+(A_n+C_n-a^2)p_n(x)+A_{n-1}C_np_{n-1}(x),
\end{equation}
where
$$W_n(x^2;a,b,c,d)=(-1)^n(n+a+b+c+d-1)_np_n(x^2).$$

\subsection*{Difference equation}
\begin{eqnarray}
\label{dvWilson}
& &n(n+a+b+c+d-1)y(x)\nonumber\\
& &{}=B(x)y(x+i)-\left[B(x)+D(x)\right]y(x)+D(x)y(x-i),
\end{eqnarray}
where
$$y(x)=W_n(x^2;a,b,c,d)$$
and
$$\left\{\begin{array}{l}
\displaystyle B(x)=\frac{(a-ix)(b-ix)(c-ix)(d-ix)}{2ix(2ix-1)}\\[5mm]
\displaystyle D(x)=\frac{(a+ix)(b+ix)(c+ix)(d+ix)}{2ix(2ix+1)}.
\end{array}\right.$$

\subsection*{Forward shift operator}
\begin{eqnarray}
\label{shift1WilsonI}
& &W_n((x+\textstyle\frac{1}{2}i)^2;a,b,c,d)
-W_n((x-\textstyle\frac{1}{2}i)^2;a,b,c,d)\nonumber\\
& &{}=-2inx(n+a+b+c+d-1)W_{n-1}(x^2;a+\textstyle\frac{1}{2},b+\textstyle\frac{1}{2},
c+\textstyle\frac{1}{2},d+\textstyle\frac{1}{2})
\end{eqnarray}
or equivalently
\begin{eqnarray}
\label{shift1WilsonII}
& &\frac{\delta W_n(x^2;a,b,c,d)}{\delta x^2}\nonumber\\
& &{}=-n(n+a+b+c+d-1)W_{n-1}(x^2;a+\textstyle\frac{1}{2},b+\textstyle\frac{1}{2},
c+\textstyle\frac{1}{2},d+\textstyle\frac{1}{2}).
\end{eqnarray}

\newpage

\subsection*{Backward shift operator}
\begin{eqnarray}
\label{shift2WilsonI}
& &(a-\textstyle\frac{1}{2}-ix)(b-\textstyle\frac{1}{2}-ix)
(c-\textstyle\frac{1}{2}-ix)(d-\textstyle\frac{1}{2}-ix)
W_n((x+\textstyle\frac{1}{2}i)^2;a,b,c,d)\nonumber\\
& &{}\mathindent{}-(a-\textstyle\frac{1}{2}+ix)(b-\textstyle\frac{1}{2}+ix)
(c-\textstyle\frac{1}{2}+ix)(d-\textstyle\frac{1}{2}+ix)
W_n((x-\textstyle\frac{1}{2}i)^2;a,b,c,d)\nonumber\\
& &{}=-2ix W_{n+1}(x^2;a-\textstyle\frac{1}{2},b-\textstyle\frac{1}{2},
c-\textstyle\frac{1}{2},d-\textstyle\frac{1}{2})
\end{eqnarray}
or equivalently
\begin{eqnarray}
\label{shift2WilsonII}
& &\frac{\delta\left[\omega(x;a,b,c,d)W_n(x^2;a,b,c,d)\right]}{\delta x^2}\nonumber\\
& &{}=\omega(x;a-\textstyle\frac{1}{2},b-\textstyle\frac{1}{2},
c-\textstyle\frac{1}{2},d-\textstyle\frac{1}{2})
W_{n+1}(x^2;a-\textstyle\frac{1}{2},b-\textstyle\frac{1}{2},c-\textstyle\frac{1}{2},d-\textstyle\frac{1}{2}),
\end{eqnarray}
where
$$\omega(x;a,b,c,d):=\frac{1}{2ix}\left|\frac{\Gamma(a+ix)\Gamma(b+ix)\Gamma(c+ix)\Gamma(d+ix)}{\Gamma(2ix)}\right|^2.$$

\subsection*{Rodrigues-type formula}
\begin{eqnarray}
\label{RodWilson}
& &\omega(x;a,b,c,d)W_n(x^2;a,b,c,d)\nonumber\\
& &{}=\left(\frac{\delta}{\delta x^2}\right)^n
\left[\omega(x;a+\textstyle\frac{1}{2}n,b+\textstyle\frac{1}{2}n,
c+\textstyle\frac{1}{2}n,d+\textstyle\frac{1}{2}n)\right].
\end{eqnarray}

\subsection*{Generating functions}
\begin{equation}
\label{GenWilson1}
\hyp{2}{1}{a+ix,b+ix}{a+b}{t}\,\hyp{2}{1}{c-ix,d-ix}{c+d}{t}=
\sum_{n=0}^{\infty}\frac{W_n(x^2;a,b,c,d)t^n}{(a+b)_n(c+d)_nn!}.
\end{equation}

\begin{equation}
\label{GenWilson2}
\hyp{2}{1}{a+ix,c+ix}{a+c}{t}\,\hyp{2}{1}{b-ix,d-ix}{b+d}{t}=
\sum_{n=0}^{\infty}\frac{W_n(x^2;a,b,c,d)t^n}{(a+c)_n(b+d)_nn!}.
\end{equation}

\begin{equation}
\label{GenWilson3}
\hyp{2}{1}{a+ix,d+ix}{a+d}{t}\,\hyp{2}{1}{b-ix,c-ix}{b+c}{t}=
\sum_{n=0}^{\infty}\frac{W_n(x^2;a,b,c,d)t^n}{(a+d)_n(b+c)_nn!}.
\end{equation}

\begin{eqnarray}
\label{GenWilson4}
& &(1-t)^{1-a-b-c-d}\nonumber\\
& &{}\mathindent\times\hyp{4}{3}{\frac{1}{2}(a+b+c+d-1),\frac{1}{2}(a+b+c+d),a+ix,a-ix}
{a+b,a+c,a+d}{-\frac{4t}{(1-t)^2}}\nonumber\\
& &{}=\sum_{n=0}^{\infty}\frac{(a+b+c+d-1)_n}{(a+b)_n(a+c)_n(a+d)_nn!}W_n(x^2;a,b,c,d)t^n.
\end{eqnarray}

\subsection*{Limit relations}

\subsubsection*{Wilson $\rightarrow$ Continuous dual Hahn}
The continuous dual Hahn polynomials given by (\ref{DefContDualHahn}) can be found from the
Wilson polynomials by dividing by $(a+d)_n$ and letting $d\rightarrow\infty$:
\begin{equation}
\lim_{d\rightarrow\infty}\frac{W_n(x^2;a,b,c,d)}{(a+d)_n}=S_n(x^2;a,b,c).
\end{equation}

\subsubsection*{Wilson $\rightarrow$ Continuous Hahn}
The continuous Hahn polynomials given by (\ref{DefContHahn}) are obtained from the Wilson
polynomials by the substitutions $a\rightarrow a-it$, $b\rightarrow b-it$, $c\rightarrow c+it$,
$d\rightarrow d+it$ and $x\rightarrow x+t$ and the limit $t\rightarrow\infty$ in the following
way:
\begin{equation}
\lim_{t\rightarrow\infty}
\frac{W_n((x+t)^2;a-it,b-it,c+it,d+it)}{(-2t)^nn!}=p_n(x;a,b,c,d).
\end{equation}

\subsubsection*{Wilson $\rightarrow$ Jacobi}
The Jacobi polynomials given by (\ref{DefJacobi}) can be found from the Wilson
polynomials by substituting $a=b=\frac{1}{2}(\alpha+1)$, $c=\frac{1}{2}(\beta+1)+it$,
$d=\frac{1}{2}(\beta+1)-it$ and $x\rightarrow t\sqrt{\frac{1}{2}(1-x)}$ in the
definition (\ref{DefWilson}) of the Wilson polynomials and taking the limit
$t\rightarrow\infty$. In fact we have
\begin{eqnarray}
& &\lim_{t\rightarrow\infty}\frac{W_n(\frac{1}{2}(1-x)t^2;\frac{1}{2}(\alpha+1),
\frac{1}{2}(\alpha+1),\frac{1}{2}(\beta+1)+it,\frac{1}{2}(\beta+1)-it)}{t^{2n}n!}\nonumber\\
& &{}=P_n^{(\alpha,\beta)}(x).
\end{eqnarray}

\subsection*{Remarks}
Note that for $k<n$ we have
$$\frac{(a+b)_n(a+c)_n(a+d)_n}{(a+b)_k(a+c)_k(a+d)_k}=(a+b+k)_{n-k}(a+c+k)_{n-k}(a+d+k)_{n-k},$$
which implies that the Wilson polynomials defined by (\ref{DefWilson}) can also be
seen as polynomials in the parameters $a$, $b$, $c$ and $d$.

\noindent
If we set
$$a=\textstyle\frac{1}{2}(\gamma+\delta+1),$$
$$b=\textstyle\frac{1}{2}(2\alpha-\gamma-\delta+1),$$
$$c=\textstyle\frac{1}{2}(2\beta-\gamma+\delta+1),$$
$$d=\textstyle\frac{1}{2}(\gamma-\delta+1),$$
and
$$ix\rightarrow x+\textstyle\frac{1}{2}(\gamma+\delta+1)$$
in
\begin{equation}
{\tilde{W}}_n(x^2;a,b,c,d)=\frac{W_n(x^2;a,b,c,d)}{(a+b)_n(a+c)_n(a+d)_n}
%  \constraint{ $\alpha+1=-N\textrm{or}\beta+\delta+1=-N\textrm{or}\gamma+1=-N$ }
\end{equation}
given by (\ref{DefWilson}) and take
with $N$ a nonnegative integer, we obtain the Racah polynomials given by (\ref{DefRacah}).
% RS: add begin\label{sec9.1}
%
\paragraph{\large\bf KLSadd: Symmetry}The Wilson polynomial $W_n(y;a,b,c,d)$ is symmetric
in $a,b,c,d$.
\\
This follows from the orthogonality relation (9.1.2)
together with the value of its coefficient of $y^n$ given in (9.1.5b).
Alternatively, combine (9.1.1) with \mycite{AAR}{Theorem 3.1.1}.\\
As a consequence, it is sufficient to give generating function (9.1.12). Then the generating
functions (9.1.13), (9.1.14) will follow by symmetry in the parameters.
%
\paragraph{\large\bf KLSadd: Hypergeometric representation}In addition to (9.1.1) we have (see \myciteKLS{513}{(2.2)}):
\begin{multline}
W_n(x^2;a,b,c,d)
=\frac{(a-ix)_n (b-ix)_n (c-ix)_n (d-ix)_n}{(-2ix)_n}\\
\times\hyp76{2ix-n,ix-\thalf n+1,a+ix,b+ix,c+ix,d+ix,-n}
{ix-\thalf n,1-n-a+ix,1-n-b+ix,1-n-c+ix,1-n-d+ix}1.
\label{112}
\end{multline}
The symmetry in $a,b,c,d$ is clear from \eqref{112}.
%
\paragraph{\large\bf KLSadd: Special value}\begin{equation}
W_n(-a^2;a,b,c,d)=(a+b)_n(a+c)_n(a+d)_n\,,
\label{91}
\end{equation}
and similarly for arguments $-b^2$, $-c^2$ and
$-d^2$ by symmetry of $W_n$ in $a,b,c,d$.
%
\paragraph{\large\bf KLSadd: Uniqueness of orthogonality measure}Under the assumptions on $a,b,c,d$ for (9.1.2) or (9.1.3) the orthogonality
measure is unique up to constant factor.

For the proof assume without
loss of generality (by the symmetry in $a,b,c,d$) that $\Re a\ge0$.
Write the \RHS\ of (9.1.2) or (9.1.3) as $h_n\de_{m,n}$.
Observe from (9.1.2) and \eqref{91} that
\[
\frac{|W_n(-a^2;a,b,c,d)|^2}{h_n} = O(n^{4\Re a-1})\quad\hbox{as $n\to\iy$.}
\]
Therefore \eqref{90} holds, from which the uniqueness of the orthogonality
measure follows.

By a similar, but necessarily more complicated argument Ismail et al.\
\myciteKLS{281}{Section 3} proved the uniqueness of orthogonality measure for
associated Wilson polynomials.
%
% RS: add end
\subsection*{References}
\cite{Askey89I}, \cite{AskeyWilson82}, \cite{AskeyWilson85}, \cite{AtakRahmanSuslov},
\cite{Ismail2005II}, \cite{IsmailLetMasVal}, \cite{IsmailLetValWimp90},
\cite{IsmailLetValWimp91}, \cite{Koorn85}, \cite{Koorn88}, \cite{LeskyWaibel},
\cite{Masson91}, \cite{Miller87}, \cite{MimachiII}, \cite{ValentAssche}, \cite{Wilson80}, 
\cite{Wilson91}.


\section{Racah}\index{Racah polynomials}

\par\setcounter{equation}{0}

\subsection*{Hypergeometric representation}
\begin{eqnarray}
\label{DefRacah}
& &R_n(\lambda(x);\alpha,\beta,\gamma,\delta)\nonumber\\
& &{}=\hyp{4}{3}{-n,n+\alpha+\beta+1,-x,x+\gamma+\delta+1}{\alpha+1,\beta+\delta+1,\gamma+1}{1},
\quad n=0,1,2,\ldots,N,
\end{eqnarray}
where
$$\lambda(x)=x(x+\gamma+\delta+1)$$
and
$$\alpha+1=-N\quad\textrm{or}\quad\beta+\delta+1=-N\quad\textrm{or}\quad\gamma+1=-N$$
with $N$ a nonnegative integer.

\subsection*{Orthogonality relation}
\begin{eqnarray}
\label{OrtRacah}
& &\sum_{x=0}^N\frac{(\alpha+1)_x(\beta+\delta+1)_x(\gamma+1)_x(\gamma+\delta+1)_x((\gamma+\delta+3)/2)_x}
{(-\alpha+\gamma+\delta+1)_x(-\beta+\gamma+1)_x((\gamma+\delta+1)/2)_x(\delta+1)_xx!}\nonumber\\
& &{}\mathindent\times R_m(\lambda(x))R_n(\lambda(x))\nonumber\\
& &{}=M\frac{(n+\alpha+\beta+1)_n(\alpha+\beta-\gamma+1)_n(\alpha-\delta+1)_n(\beta+1)_nn!}
{(\alpha+\beta+2)_{2n}(\alpha+1)_n(\beta+\delta+1)_n(\gamma+1)_n}\,\delta_{mn},
\end{eqnarray}
where
$$R_n(\lambda(x)):=R_n(\lambda(x);\alpha,\beta,\gamma,\delta)$$
and
$$M=\left\{\begin{array}{ll}
\displaystyle\frac{(-\beta)_N(\gamma+\delta+2)_N}{(-\beta+\gamma+1)_N(\delta+1)_N}&\quad\textrm{if}\quad\alpha+1=-N\\[5mm]
\displaystyle\frac{(-\alpha+\delta)_N(\gamma+\delta+2)_N}{(-\alpha+\gamma+\delta+1)_N(\delta+1)_N}&\quad\textrm{if}\quad\beta+\delta+1=-N\\[5mm]
\displaystyle\frac{(\alpha+\beta+2)_N(-\delta)_N}{(\alpha-\delta+1)_N(\beta+1)_N}&\quad\textrm{if}\quad\gamma+1=-N.
\end{array}\right.$$

\subsection*{Recurrence relation}
\begin{equation}
\label{RecRacah}
\lambda(x)R_n(\lambda(x))
=A_nR_{n+1}(\lambda(x))-\left(A_n+C_n\right)R_n(\lambda(x))+C_nR_{n-1}(\lambda(x)),
\end{equation}
where
$$R_n(\lambda(x)):=R_n(\lambda(x);\alpha,\beta,\gamma,\delta)$$
and
$$\left\{\begin{array}{l}
\displaystyle A_n=\frac{(n+\alpha+1)(n+\alpha+\beta+1)(n+\beta+\delta+1)(n+\gamma+1)}{(2n+\alpha+\beta+1)(2n+\alpha+\beta+2)}\\[5mm]
\displaystyle C_n=\frac{n(n+\alpha+\beta-\gamma)(n+\alpha-\delta)(n+\beta)}{(2n+\alpha+\beta)(2n+\alpha+\beta+1)},
\end{array}\right.$$
hence
$$A_n=\left\{\begin{array}{ll}
\displaystyle\frac{(n+\beta-N)(n+\beta+\delta+1)(n+\gamma+1)(n-N)}{(2n+\beta-N)(2n+\beta-N+1)}&\quad\textrm{if}\quad\alpha+1=-N\\[5mm]
\displaystyle\frac{(n+\alpha+1)(n+\alpha+\beta+1)(n+\gamma+1)(n-N)}{(2n+\alpha+\beta+1)(2n+\alpha+\beta+2)}&\quad\textrm{if}\quad\beta+\delta+1=-N\\[5mm]
\displaystyle\frac{(n+\alpha+1)(n+\alpha+\beta+1)(n+\beta+\delta+1)(n-N)}{(2n+\alpha+\beta+1)(2n+\alpha+\beta+2)}&\quad\textrm{if}\quad\gamma+1=-N
\end{array}\right.$$
and
$$C_n=\left\{\begin{array}{ll}
\displaystyle\frac{n(n+\beta)(n+\beta-\gamma-N-1)(n-\delta-N-1)}{(2n+\beta-N-1)(2n+\beta-N)}&\quad\textrm{if}\quad\alpha+1=-N\\[5mm]
\displaystyle\frac{n(n+\alpha+\beta+N+1)(n+\alpha+\beta-\gamma)(n+\beta)}{(2n+\alpha+\beta)(2n+\alpha+\beta+1)}&\quad\textrm{if}\quad\beta+\delta+1=-N\\[5mm]
\displaystyle\frac{n(n+\alpha+\beta+N+1)(n+\alpha-\delta)(n+\beta)}{(2n+\alpha+\beta)(2n+\alpha+\beta+1)}&\quad\textrm{if}\quad\gamma+1=-N.
\end{array}\right.$$

\subsection*{Normalized recurrence relation}
\begin{equation}
\label{NormRecRacah}
xp_n(x)=p_{n+1}(x)-(A_n+C_n)p_n(x)+A_{n-1}C_np_{n-1}(x),
\end{equation}
where
$$R_n(\lambda(x);\alpha,\beta,\gamma,\delta)=
\frac{(n+\alpha+\beta+1)_n}{(\alpha+1)_n(\beta+\delta+1)_n(\gamma+1)_n}p_n(\lambda(x)).$$

\subsection*{Difference equation}
\begin{equation}
\label{dvRacah}
n(n+\alpha+\beta+1)y(x)=B(x)y(x+1)-\left[B(x)+D(x)\right]y(x)+D(x)y(x-1),
\end{equation}
where
$$y(x)=R_n(\lambda(x);\alpha,\beta,\gamma,\delta)$$
and
$$\left\{\begin{array}{l}
\displaystyle B(x)=\frac{(x+\alpha+1)(x+\beta+\delta+1)(x+\gamma+1)(x+\gamma+\delta+1)}
{(2x+\gamma+\delta+1)(2x+\gamma+\delta+2)}\\[5mm]
\displaystyle D(x)=\frac{x(x-\alpha+\gamma+\delta)(x-\beta+\gamma)(x+\delta)}{(2x+\gamma+\delta)(2x+\gamma+\delta+1)}.
\end{array}\right.$$

\subsection*{Forward shift operator}
\begin{eqnarray}
\label{shift1RacahI}
& &R_n(\lambda(x+1);\alpha,\beta,\gamma,\delta)-R_n(\lambda(x);\alpha,\beta,\gamma,\delta)\nonumber\\
& &{}=\frac{n(n+\alpha+\beta+1)}{(\alpha+1)(\beta+\delta+1)(\gamma+1)}\nonumber\\
& &{}\mathindent{}\times(2x+\gamma+\delta+2)R_{n-1}(\lambda(x);\alpha+1,\beta+1,\gamma+1,\delta)
\end{eqnarray}
or equivalently
\begin{eqnarray}
\label{shift1RacahII}
& &\frac{\Delta R_n(\lambda(x);\alpha,\beta,\gamma,\delta)}{\Delta\lambda(x)}\nonumber\\
& &{}=\frac{n(n+\alpha+\beta+1)}{(\alpha+1)(\beta+\delta+1)(\gamma+1)}
R_{n-1}(\lambda(x);\alpha+1,\beta+1,\gamma+1,\delta).
\end{eqnarray}

\subsection*{Backward shift operator}
\begin{eqnarray}
\label{shift2RacahI}
& &(x+\alpha)(x+\beta+\delta)(x+\gamma)(x+\gamma+\delta)R_n(\lambda(x);\alpha,\beta,\gamma,\delta)\nonumber\\
& &{}\mathindent{}-x(x-\beta+\gamma)(x-\alpha+\gamma+\delta)(x+\delta)R_n(\lambda(x-1);\alpha,\beta,\gamma,\delta)\nonumber\\
& &{}=\alpha\gamma(\beta+\delta)(2x+\gamma+\delta)R_{n+1}(\lambda(x);\alpha-1,\beta-1,\gamma-1,\delta)
\end{eqnarray}
or equivalently
\begin{eqnarray}
\label{shift2RacahII}
& &\frac{\nabla\left[\omega(x;\alpha,\beta,\gamma,\delta)R_n(\lambda(x);\alpha,\beta,\gamma,\delta)\right]}{\nabla\lambda(x)}\nonumber\\
& &{}=\frac{1}{\gamma+\delta}\omega(x;\alpha-1,\beta-1,\gamma-1,\delta)
R_{n+1}(\lambda(x);\alpha-1,\beta-1,\gamma-1,\delta),
\end{eqnarray}
where
$$\omega(x;\alpha,\beta,\gamma,\delta)=\frac{(\alpha+1)_x(\beta+\delta+1)_x(\gamma+1)_x(\gamma+\delta+1)_x}
{(-\alpha+\gamma+\delta+1)_x(-\beta+\gamma+1)_x(\delta+1)_xx!}.$$

\subsection*{Rodrigues-type formula}
\begin{eqnarray}
\label{RodRacah}
& &\omega(x;\alpha,\beta,\gamma,\delta)R_n(\lambda(x);\alpha,\beta,\gamma,\delta)\nonumber\\
& &{}=(\gamma+\delta+1)_n\left(\nabla_{\lambda}\right)^n\left[\omega(x;\alpha+n,\beta+n,\gamma+n,\delta)\right],
\end{eqnarray}
where
$$\nabla_{\lambda}:=\frac{\nabla}{\nabla\lambda(x)}.$$

\subsection*{Generating functions}
For $x=0,1,2,\ldots,N$ we have
\begin{equation}
\label{GenRacah1}
\hyp{2}{1}{-x,-x+\alpha-\gamma-\delta}{\alpha+1}{t}\,\hyp{2}{1}{x+\beta+\delta+1,x+\gamma+1}{\beta+1}{t}
{}=\sum_{n=0}^N\frac{(\beta+\delta+1)_n(\gamma+1)_n}{(\beta+1)_nn!}
R_n(\lambda(x);\alpha,\beta,\gamma,\delta)t^n,
%  \constraint{ $\beta+\delta+1=-N \textrm{or} \gamma+1=-N$ }
\end{equation}

\begin{eqnarray}
\label{GenRacah2}
\hyp{2}{1}{-x,-x+\beta-\gamma}{\beta+\delta+1}{t}\,\hyp{2}{1}{x+\alpha+1,x+\gamma+1}{\alpha-\delta+1}{t}
{}=\sum_{n=0}^N\frac{(\alpha+1)_n(\gamma+1)_n}{(\alpha-\delta+1)_nn!}
R_n(\lambda(x);\alpha,\beta,\gamma,\delta)t^n,
%  \constraint{ $\alpha+1=-N \textrm{or} \gamma+1=-N$ }
\end{eqnarray}

\begin{eqnarray}
\label{GenRacah3}
\hyp{2}{1}{-x,-x-\delta}{\gamma+1}{t}\,\hyp{2}{1}{x+\alpha+1,x+\beta+\delta+1}{\alpha+\beta-\gamma+1}{t}\
{}=\sum_{n=0}^N\frac{(\alpha+1)_n(\beta+\delta+1)_n}{(\alpha+\beta-\gamma+1)_nn!}
R_n(\lambda(x);\alpha,\beta,\gamma,\delta)t^n,
%  \constraint{ $\alpha+1=-N \textrm{or} \beta+\delta+1=-N$ }
\end{eqnarray}

\begin{eqnarray}
\label{GenRacah4}
& &\Bigg[(1-t)^{-\alpha-\beta-1}\nonumber\\
& &{}\mathindent\times\hyp{4}{3}{\frac{1}{2}(\alpha+\beta+1),\frac{1}{2}(\alpha+\beta+2),-x,x+\gamma+\delta+1}
{\alpha+1,\beta+\delta+1,\gamma+1}{-\frac{4t}{(1-t)^2}}\Bigg]_N\nonumber\\
& &{}=\sum_{n=0}^N\frac{(\alpha+\beta+1)_n}{n!}R_n(\lambda(x);\alpha,\beta,\gamma,\delta)t^n.
\end{eqnarray}

\subsection*{Limit relations}

\subsubsection*{Racah $\rightarrow$ Hahn}
The Hahn polynomials given by (\ref{DefHahn}) can be obtained from the Racah polynomials
by taking $\gamma+1=-N$ and letting $\delta\rightarrow\infty$:
\begin{equation}
\lim_{\delta\rightarrow\infty}
R_n(\lambda(x);\alpha,\beta,-N-1,\delta)=Q_n(x;\alpha,\beta,N).
\end{equation}
The Hahn polynomials given by (\ref{DefHahn}) can also be obtained from the Racah polynomials
by taking $\delta=-\beta-N-1$ and letting $\gamma\rightarrow\infty$:
\begin{equation}
\lim_{\gamma\rightarrow\infty}
R_n(\lambda(x);\alpha,\beta,\gamma,-\beta-N-1)=Q_n(x;\alpha,\beta,N).
\end{equation}
Another way to do this is to take $\alpha+1=-N$ and $\beta\rightarrow\beta+\gamma+N+1$ and then
take the limit $\delta\rightarrow\infty$. In that case we obtain the Hahn polynomials given by
(\ref{DefHahn}) in the following way:
\begin{equation}
\lim_{\delta\rightarrow\infty}
R_n(\lambda(x);-N-1,\beta+\gamma+N+1,\gamma,\delta)=Q_n(x;\gamma,\beta,N).
\end{equation}

\subsubsection*{Racah $\rightarrow$ Dual Hahn}
The dual Hahn polynomials given by (\ref{DefDualHahn}) are obtained from the Racah polynomials
if we take $\alpha+1=-N$ and let $\beta\rightarrow\infty$:
\begin{equation}
\lim_{\beta\rightarrow\infty}
R_n(\lambda(x);-N-1,\beta,\gamma,\delta)=R_n(\lambda(x);\gamma,\delta,N).
\end{equation}
The dual Hahn polynomials given by (\ref{DefDualHahn}) are also obtained from the Racah polynomials
if we take $\beta=-\delta-N-1$ and let $\alpha\rightarrow\infty$:
\begin{equation}
\lim_{\alpha\rightarrow\infty}
R_n(\lambda(x);\alpha,-\delta-N-1,\gamma,\delta)=R_n(\lambda(x);\gamma,\delta,N).
\end{equation}
Finally, the dual Hahn polynomials given by (\ref{DefDualHahn}) are also obtained from the Racah polynomials
if we take $\gamma+1=-N$ and $\delta\rightarrow\alpha+\delta+N+1$ and take the limit $\beta\rightarrow\infty$:
\begin{equation}
\lim_{\beta\rightarrow\infty}
R_n(\lambda(x);\alpha,\beta,-N-1,\alpha+\delta+N+1)=R_n(\lambda(x);\alpha,\delta,N).
\end{equation}

\newpage

\subsection*{Remark}
If we set $\alpha=a+b-1$, $\beta=c+d-1$, $\gamma=a+d-1$, $\delta=a-d$ and $x\rightarrow
-a+ix$ in the definition (\ref{DefRacah}) of the Racah polynomials we obtain
the Wilson polynomials given by (\ref{DefWilson}):
\begin{eqnarray*}
& &R_n(\lambda(-a+ix);a+b-1,c+d-1,a+d-1,a-d)\\
& &{}=\tilde{W}_n(x^2;a,b,c,d)=\frac{W_n(x^2;a,b,c,d)}{(a+b)_n(a+c)_n(a+d)_n}.
\end{eqnarray*}
% RS: add begin\label{sec9.2}
\paragraph{\large\bf KLSadd: Racah in terms of Wilson}In the Remark on p.196 Racah polynomials are expressed in terms of
Wilson polynomials. This can be equivalently written as
\begin{multline}
R_n\big(x(x-N+\de);\al,\be,-N-1,\de\big)\\
=\frac{W_n\big(-(x+\thalf(\de-N))^2;\thalf(\de-N),\al+1-\thalf(\de-N),
\be+\thalf(\de+N)+1,-\half(\de+N)\big)}
{(\al+1)_n (\be+\de+1)_n (-N)_n}\,.
\label{146}
\end{multline}
%
% RS: add end
\subsection*{References}
\cite{Askey89I}, \cite{AskeyWilson79}, \cite{AskeyWilson85},
\cite{AtakRahmanSuslov}, \cite{AtakSuslov88}, \cite{Dunkl84},
\cite{Koorn88}, \cite{Lesky93}, \cite{Lesky95II}, \cite{LeskyWaibel},
\cite{Nikiforov+}, \cite{NikiforovUvarov}, \cite{Perlstadt}, \cite{Rahman80}, 
\cite{Rahman81II}, \cite{Wilson80}.


\section{Continuous dual Hahn}\index{Continuous dual Hahn polynomials}
\index{Dual Hahn polynomials!Continuous}\index{Hahn polynomials!Continuous dual}

\par\setcounter{equation}{0}

\subsection*{Hypergeometric representation}
\begin{equation}
\label{DefContDualHahn}
\frac{S_n(x^2;a,b,c)}{(a+b)_n(a+c)_n}=\hyp{3}{2}{-n,a+ix,a-ix}{a+b,a+c}{1}.
\end{equation}

\subsection*{Orthogonality relation}
If $a$, $b$ and $c$ are positive except possibly for a pair of complex conjugates with positive real parts, then
\begin{eqnarray}
\label{OrtIContDualHahn}
& &\frac{1}{2\pi}\int_0^{\infty}\left|\frac{\Gamma(a+ix)\Gamma(b+ix)\Gamma(c+ix)}{\Gamma(2ix)}\right|^2
S_m(x^2;a,b,c)S_n(x^2;a,b,c)\,dx\nonumber\\
& &{}=\Gamma(n+a+b)\Gamma(n+a+c)\Gamma(n+b+c)n!\,\delta_{mn}.
\end{eqnarray}
If $a<0$ and $a+b$, $a+c$ are positive or a pair of complex conjugates
with positive real parts, then
\begin{eqnarray}
\label{OrtIIContDualHahn}
& &\frac{1}{2\pi}\int_0^{\infty}\left|\frac{\Gamma(a+ix)\Gamma(b+ix)\Gamma(c+ix)}{\Gamma(2ix)}\right|^2
S_m(x^2;a,b,c)S_n(x^2;a,b,c)\,dx\nonumber\\
& &{}\mathindent{}+\frac{\Gamma(a+b)\Gamma(a+c)\Gamma(b-a)\Gamma(c-a)}{\Gamma(-2a)}\nonumber\\
& &{}\mathindent\mathindent{}\times\sum_{\begin{array}{c}
{\scriptstyle k=0,1,2\ldots}\\{\scriptstyle a+k<0}\end{array}}
\frac{(2a)_k(a+1)_k(a+b)_k(a+c)_k}{(a)_k(a-b+1)_k(a-c+1)_kk!}(-1)^k\nonumber\\
& &{}\mathindent\mathindent\mathindent\mathindent{}\times S_m(-(a+k)^2;a,b,c)S_n(-(a+k)^2;a,b,c)\nonumber\\
& &{}=\Gamma(n+a+b)\Gamma(n+a+c)\Gamma(n+b+c)n!\,\delta_{mn}.
\end{eqnarray}

\subsection*{Recurrence relation}
\begin{equation}
\label{RecContDualHahn}
-\left(a^2+x^2\right){\tilde {S}}_n(x^2)=
A_n{\tilde {S}}_{n+1}(x^2)-\left(A_n+C_n\right){\tilde {S}}_n(x^2)+C_n{\tilde {S}}_{n-1}(x^2),
\end{equation}
where
$${\tilde {S}}_n(x^2):={\tilde {S}}_n(x^2;a,b,c)=\frac{S_n(x^2;a,b,c)}{(a+b)_n(a+c)_n}$$
and
$$\left\{\begin{array}{l}
\displaystyle A_n=(n+a+b)(n+a+c)\\[5mm]
\displaystyle C_n=n(n+b+c-1).
\end{array}\right.$$

\subsection*{Normalized recurrence relation}
\begin{equation}
\label{NormRecContDualHahn}
xp_n(x)=p_{n+1}(x)+(A_n+C_n-a^2)p_n(x)+A_{n-1}C_np_{n-1}(x),
\end{equation}
where
$$S_n(x^2;a,b,c)=(-1)^np_n(x^2).$$

\subsection*{Difference equation}
\begin{equation}
\label{dvContDualHahn}
ny(x)=B(x)y(x+i)-\left[B(x)+D(x)\right]y(x)+D(x)y(x-i),
\end{equation}
where
$$y(x)=S_n(x^2;a,b,c)$$
and
$$\left\{\begin{array}{l}
\displaystyle B(x)=\frac{(a-ix)(b-ix)(c-ix)}{2ix(2ix-1)}\\[5mm]
\displaystyle D(x)=\frac{(a+ix)(b+ix)(c+ix)}{2ix(2ix+1)}.
\end{array}\right.$$

\subsection*{Forward shift operator}
\begin{eqnarray}
\label{shift1ContDualHahnI}
& &S_n((x+\textstyle\frac{1}{2}i)^2;a,b,c)-S_n((x-\textstyle\frac{1}{2}i)^2;a,b,c)\nonumber\\
& &{}=-2inxS_{n-1}(x^2;a+\textstyle\frac{1}{2},b+\textstyle\frac{1}{2},c+\textstyle\frac{1}{2})
\end{eqnarray}
or equivalently
\begin{equation}
\label{shift1ContDualHahnII}
\frac{\delta S_n(x^2;a,b,c)}{\delta x^2}=-nS_{n-1}(x^2;a+\textstyle\frac{1}{2},
b+\textstyle\frac{1}{2},c+\textstyle\frac{1}{2}).
\end{equation}

\subsection*{Backward shift operator}
\begin{eqnarray}
\label{shift2ContDualHahnI}
& &(a-\textstyle\frac{1}{2}-ix)(b-\textstyle\frac{1}{2}-ix)
(c-\textstyle\frac{1}{2}-ix) S_n((x+\textstyle\frac{1}{2}i)^2;a,b,c)\nonumber\\
& &{}\mathindent{}-(a-\textstyle\frac{1}{2}+ix)(b-\textstyle\frac{1}{2}+ix)
(c-\textstyle\frac{1}{2}+ix) S_n((x-\textstyle\frac{1}{2}i)^2;a,b,c)\nonumber\\
& &{}=-2ixS_{n+1}(x^2;a-\textstyle\frac{1}{2},b-\textstyle\frac{1}{2},
c-\textstyle\frac{1}{2})
\end{eqnarray}
or equivalently
\begin{eqnarray}
\label{shift2ContDualHahnII}
& &\frac{\delta\left[\omega(x;a,b,c)S_n(x^2;a,b,c)\right]}{\delta x^2}\nonumber\\
& &{}=\omega(x;a-\textstyle\frac{1}{2},b-\textstyle\frac{1}{2},c-\textstyle\frac{1}{2})
S_{n+1}(x^2;a-\textstyle\frac{1}{2},b-\textstyle\frac{1}{2},c-\textstyle\frac{1}{2}),
\end{eqnarray}
where
$$\omega(x;a,b,c)=\frac{1}{2ix}\left|\frac{\Gamma(a+ix)\Gamma(b+ix)\Gamma(c+ix)}{\Gamma(2ix)}\right|^2.$$

\subsection*{Rodrigues-type formula}
\begin{equation}
\label{RodContDualHahn}
\omega(x;a,b,c)S_n(x^2;a,b,c)=
\left(\frac{\delta}{\delta x^2}\right)^n\left[\omega(x;a+\textstyle\frac{1}{2}n,
b+\textstyle\frac{1}{2}n,c+\textstyle\frac{1}{2}n)\right].
\end{equation}

\subsection*{Generating functions}
\begin{equation}
\label{GenContDualHahn1}
(1-t)^{-c+ix}\,\hyp{2}{1}{a+ix,b+ix}{a+b}{t}
=\sum_{n=0}^{\infty}\frac{S_n(x^2;a,b,c)}{(a+b)_nn!}t^n.
\end{equation}

\begin{equation}
\label{GenContDualHahn2}
(1-t)^{-b+ix}\,\hyp{2}{1}{a+ix,c+ix}{a+c}{t}
=\sum_{n=0}^{\infty}\frac{S_n(x^2;a,b,c)}{(a+c)_nn!}t^n.
\end{equation}

\begin{equation}
\label{GenContDualHahn3}
(1-t)^{-a+ix}\,\hyp{2}{1}{b+ix,c+ix}{b+c}{t}
=\sum_{n=0}^{\infty}\frac{S_n(x^2;a,b,c)}{(b+c)_nn!}t^n.
\end{equation}

\begin{equation}
\label{GenContDualHahn4}
\e^t\,\hyp{2}{2}{a+ix,a-ix}{a+b,a+c}{-t}=\sum_{n=0}^{\infty}
\frac{S_n(x^2;a,b,c)}{(a+b)_n(a+c)_nn!}t^n.
\end{equation}

\begin{eqnarray}
\label{GenContDualHahn5}
& &(1-t)^{-\gamma}\,\hyp{3}{2}{\gamma,a+ix,a-ix}{a+b,a+c}{\frac{t}{t-1}}\nonumber\\
& &{}=\sum_{n=0}^{\infty}\frac{(\gamma)_nS_n(x^2;a,b,c)}{(a+b)_n(a+c)_nn!}t^n,
\quad\textrm{$\gamma$ arbitrary}.
\end{eqnarray}

\subsection*{Limit relations}

\subsubsection*{Wilson $\rightarrow$ Continuous dual Hahn}
The continuous dual Hahn polynomials can be found from the Wilson polynomials
given by (\ref{DefWilson}) by dividing by $(a+d)_n$ and letting $d\rightarrow\infty$:
$$\lim_{d\rightarrow\infty}\frac{W_n(x^2;a,b,c,d)}{(a+d)_n}=S_n(x^2;a,b,c).$$

\subsubsection*{Continuous dual Hahn $\rightarrow$ Meixner-Pollaczek}
The Meixner-Pollaczek polynomials given by (\ref{DefMP}) can be obtained from the continuous
dual Hahn polynomials by the substitutions $x\rightarrow x-t$, $a=\lambda+it$, $b=\lambda-it$
and $c=t\cot\phi$ and the limit $t\rightarrow\infty$:
\begin{equation}
\lim_{t\rightarrow\infty}\frac{S_n((x-t)^2;\lambda+it,\lambda-it,t\cot\phi)}{t^nn!}
=\frac{P_n^{(\lambda)}(x;\phi)}{(\sin\phi)^n}.
\end{equation}

\subsection*{Remark}
Since we have for $k<n$
$$\frac{(a+b)_n(a+c)_n}{(a+b)_k(a+c)_k}=(a+b+k)_{n-k}(a+c+k)_{n-k},$$
the continuous dual Hahn polynomials defined by (\ref{DefContDualHahn}) can also be seen as
polynomials in the parameters $a$, $b$ and $c$.
% RS: add begin\label{sec9.3}
%
\paragraph{\large\bf KLSadd: Symmetry}The continuous dual Hahn polynomial $S_n(y;a,b,c)$ is symmetric
in $a,b,c$.\\
This follows from the orthogonality relation (9.3.2)
together with the value of its coefficient of $y^n$ given in (9.3.5b).
Alternatively, combine (9.3.1) with \mycite{AAR}{Corollary 3.3.5}.\\
As a consequence, it is sufficient to give generating function (9.3.12). Then the generating
functions (9.3.13), (9.3.14) will follow by symmetry in the parameters.
%
\paragraph{\large\bf KLSadd: Special value}\begin{equation}
S_n(-a^2;a,b,c)=(a+b)_n(a+c)_n\,,
\label{92}
\end{equation}
and similarly for arguments $-b^2$ and $-c^2$ by symmetry of $S_n$ in $a,b,c$.
%
\paragraph{\large\bf KLSadd: Uniqueness of orthogonality measure}Under the assumptions on $a,b,c$ for (9.3.2) or (9.3.3) the orthogonality
measure is unique up to constant factor.

For the proof assume without
loss of generality (by the symmetry in $a,b,c,d$) that $\Re a\geq0$.
Write the \RHS\ of (9.3.2) or (9.3.3) as $h_n\de_{m,n}$.
Observe from (9.3.2) and \eqref{92} that
\[
\frac{|S_n(-a^2;a,b,c)|^2}{h_n} = O(n^{2\Re a-1})\quad
\hbox{as $n\to\iy$.}
\]
Therefore \eqref{90} holds, from which the uniqueness of the orthogonality
measure follows.
%
% RS: add end
\subsection*{References}
\cite{AskeyWilson85}, \cite{IsmailLetVal89}, \cite{Koorn85}, \cite{Koorn88},
\cite{Lesky94II}, \cite{Lesky95I}, \cite{Lesky95II}, \cite{Letessier84},
\cite{Letessier86}, \cite{MimachiII}, \cite{Neuman}, \cite{ValentAssche}.


\section{Continuous Hahn}\index{Continuous Hahn polynomials}
\index{Hahn polynomials!Continuous}

\par\setcounter{equation}{0}

\subsection*{Hypergeometric representation}
\begin{eqnarray}
\label{DefContHahn}
& &p_n(x;a,b,c,d)\nonumber\\
& &{}=i^n\frac{(a+c)_n(a+d)_n}{n!}\,\hyp{3}{2}{-n,n+a+b+c+d-1,a+ix}{a+c,a+d}{1}.
\end{eqnarray}

\subsection*{Orthogonality relation}
If $Re(a,b,c,d)>0$, $c=\bar{a}$ and $d=\bar{b}$, then
\begin{eqnarray}
\label{OrtContHahn}
& &\frac{1}{2\pi}\int_{-\infty}^{\infty}\Gamma(a+ix)\Gamma(b+ix)\Gamma(c-ix)\Gamma(d-ix)
p_m(x;a,b,c,d)p_n(x;a,b,c,d)\,dx\nonumber\\
& &{}=\frac{\Gamma(n+a+c)\Gamma(n+a+d)\Gamma(n+b+c)\Gamma(n+b+d)}
{(2n+a+b+c+d-1)\Gamma(n+a+b+c+d-1)n!}\,\delta_{mn}.
\end{eqnarray}

\subsection*{Recurrence relation}
\begin{equation}
\label{RecContHahn}
(a+ix)\tilde{p}_n(x)=A_n\tilde{p}_{n+1}(x)-\left(A_n+C_n\right)\tilde{p}_n(x)+C_n\tilde{p}_{n-1}(x),
\end{equation}
where
$$\tilde{p}_n(x):=\tilde{p}_n(x;a,b,c,d)=\frac{n!}{i^n(a+c)_n(a+d)_n}p_n(x;a,b,c,d)$$
and
$$\left\{\begin{array}{l}
\displaystyle A_n=-\frac{(n+a+b+c+d-1)(n+a+c)(n+a+d)}{(2n+a+b+c+d-1)(2n+a+b+c+d)}\\[5mm]
\displaystyle C_n=\frac{n(n+b+c-1)(n+b+d-1)}{(2n+a+b+c+d-2)(2n+a+b+c+d-1)}.
\end{array}\right.$$

\subsection*{Normalized recurrence relation}
\begin{equation}
\label{NormRecContHahn}
xp_n(x)=p_{n+1}(x)+i(A_n+C_n+a)p_n(x)-A_{n-1}C_np_{n-1}(x),
\end{equation}
where
$$p_n(x;a,b,c,d)=\frac{(n+a+b+c+d-1)_n}{n!}p_n(x).$$

\subsection*{Difference equation}
\begin{eqnarray}
\label{dvContHahn}
& &n(n+a+b+c+d-1)y(x)\nonumber\\
& &{}=B(x)y(x+i)-\left[B(x)+D(x)\right]y(x)+D(x)y(x-i),
\end{eqnarray}
where
$$y(x)=p_n(x;a,b,c,d)$$
and
$$\left\{\begin{array}{l}
\displaystyle B(x)=(c-ix)(d-ix)\\[5mm]
\displaystyle D(x)=(a+ix)(b+ix).
\end{array}\right.$$

\subsection*{Forward shift operator}
\begin{eqnarray}
\label{shift1ContHahnI}
& &p_n(x+\textstyle\frac{1}{2}i;a,b,c,d)-p_n(x-\textstyle\frac{1}{2}i;a,b,c,d)\nonumber\\
& &{}=i(n+a+b+c+d-1)p_{n-1}(x;a+\textstyle\frac{1}{2},
b+\textstyle\frac{1}{2},c+\textstyle\frac{1}{2},d+\textstyle\frac{1}{2})
\end{eqnarray}
or equivalently
\begin{equation}
\label{shift1ContHahnII}
\frac{\delta p_n(x;a,b,c,d)}{\delta x}=(n+a+b+c+d-1)
p_{n-1}(x;a+\textstyle\frac{1}{2},b+\textstyle\frac{1}{2},
c+\textstyle\frac{1}{2},d+\textstyle\frac{1}{2}).
\end{equation}

\subsection*{Backward shift operator}
\begin{eqnarray}
\label{shift2ContHahnI}
& &(c-\textstyle\frac{1}{2}-ix)(d-\textstyle\frac{1}{2}-ix)
p_n(x+\textstyle\frac{1}{2}i;a,b,c,d)\nonumber\\
& &{}\mathindent{}-(a-\textstyle\frac{1}{2}+ix)(b-\textstyle\frac{1}{2}+ix)
p_n(x-\textstyle\frac{1}{2}i;a,b,c,d)\nonumber\\
& &{}=\frac{n+1}{i}p_{n+1}(x;a-\textstyle\frac{1}{2},
b-\textstyle\frac{1}{2},c-\textstyle\frac{1}{2},d-\textstyle\frac{1}{2})
\end{eqnarray}
or equivalently
\begin{eqnarray}
\label{shift2ContHahnII}
& &\frac{\delta\left[\omega(x;a,b,c,d)p_n(x;a,b,c,d)\right]}{\delta x}\nonumber\\
& &{}=-(n+1)\omega(x;a-\textstyle\frac{1}{2},b-\textstyle\frac{1}{2},
c-\textstyle\frac{1}{2},d-\textstyle\frac{1}{2})\nonumber\\
& &{}\mathindent\times p_{n+1}(x;a-\textstyle\frac{1}{2},b-\textstyle\frac{1}{2},c-\textstyle\frac{1}{2},d-\textstyle\frac{1}{2}),
\end{eqnarray}
where
$$\omega(x;a,b,c,d)=\Gamma(a+ix)\Gamma(b+ix)\Gamma(c-ix)\Gamma(d-ix).$$

\subsection*{Rodrigues-type formula}
\begin{eqnarray}
\label{RodContHahn}
& &\omega(x;a,b,c,d)p_n(x;a,b,c,d)\nonumber\\
& &{}=\frac{(-1)^n}{n!}\left(\frac{\delta}{\delta x}\right)^n
\left[\omega(x;a+\textstyle\frac{1}{2}n,b+\textstyle\frac{1}{2}n,
c+\textstyle\frac{1}{2}n,d+\textstyle\frac{1}{2}n)\right].
\end{eqnarray}

\subsection*{Generating functions}
\begin{equation}
\label{GenContHahn1}
\hyp{1}{1}{a+ix}{a+c}{-it}\,\hyp{1}{1}{d-ix}{b+d}{it}=
\sum_{n=0}^{\infty}\frac{p_n(x;a,b,c,d)}{(a+c)_n(b+d)_n}t^n.
\end{equation}

\begin{equation}
\label{GenContHahn2}
\hyp{1}{1}{a+ix}{a+d}{-it}\,\hyp{1}{1}{c-ix}{b+c}{it}=
\sum_{n=0}^{\infty}\frac{p_n(x;a,b,c,d)}{(a+d)_n(b+c)_n}t^n.
\end{equation}

\begin{eqnarray}
\label{GenContHahn3}
& &(1-t)^{1-a-b-c-d}\,\hyp{3}{2}{\frac{1}{2}(a+b+c+d-1),\frac{1}{2}(a+b+c+d),a+ix}
{a+c,a+d}{-\frac{4t}{(1-t)^2}}\nonumber\\
& &{}=\sum_{n=0}^{\infty}
\frac{(a+b+c+d-1)_n}{(a+c)_n(a+d)_ni^n}p_n(x;a,b,c,d)t^n.
\end{eqnarray}

\subsection*{Limit relations}

\subsubsection*{Wilson $\rightarrow$ Continuous Hahn}
The continuous Hahn polynomials are obtained from the Wilson polynomials given by
(\ref{DefWilson}) by the substitution $a\rightarrow a-it$, $b\rightarrow b-it$,
$c\rightarrow c+it$, $d\rightarrow d+it$ and $x\rightarrow x+t$ and the limit
$t\rightarrow\infty$ in the following way:
$$\lim_{t\rightarrow\infty}
\frac{W_n((x+t)^2;a-it,b-it,c+it,d+it)}{(-2t)^nn!}=p_n(x;a,b,c,d).$$

\subsubsection*{Continuous Hahn $\rightarrow$ Meixner-Pollaczek}
The Meixner-Pollaczek polynomials given by (\ref{DefMP}) can be obtained from the
continuous Hahn polynomials by setting $x\rightarrow x+t$, $a=\lambda-it$, $c=\lambda+it$
and $b=d=t\tan\phi$ and taking the limit $t\rightarrow\infty$:
\begin{equation}
\lim_{t\rightarrow\infty}\frac{p_n(x+t;\lambda-it,t\tan\phi,\lambda+it,t\tan\phi)}{t^n}
=\frac{P_n^{(\lambda)}(x;\phi)}{(\cos\phi)^n}.
\end{equation}

\subsubsection*{Continuous Hahn $\rightarrow$ Jacobi}
The Jacobi polynomials given by (\ref{DefJacobi}) follow from the continuous Hahn polynomials
by the substitution $x\rightarrow \frac{1}{2}xt$, $a=\frac{1}{2}(\alpha+1-it)$,
$b=\frac{1}{2}(\beta+1+it)$, $c=\frac{1}{2}(\alpha+1+it)$ and $d=\frac{1}{2}(\beta+1-it)$,
division by $t^n$ and the limit $t\rightarrow\infty$:
\begin{eqnarray}
& &\lim_{t\rightarrow\infty}
\frac{p_n(\frac{1}{2}xt;\frac{1}{2}(\alpha+1-it),\frac{1}{2}(\beta+1+it),
\frac{1}{2}(\alpha+1+it),\frac{1}{2}(\beta+1-it))}{t^n}\nonumber\\
& &{}=P_n^{(\alpha,\beta)}(x).
\end{eqnarray}

\subsubsection*{Continuous Hahn $\rightarrow$ Pseudo Jacobi}
The pseudo Jacobi polynomials given by (\ref{DefPseudoJacobi}) follow from the continuous
Hahn polynomials by the substitution $x\rightarrow xt$, $a=\frac{1}{2}(-N+i\nu-2t)$,
$b=\frac{1}{2}(-N-i\nu+2t)$, $c=\frac{1}{2}(-N-i\nu-2t)$ and $d=\frac{1}{2}(-N+i\nu+2t)$,
division by $t^n$ and the limit $t\rightarrow\infty$:
\begin{eqnarray}
& &\lim_{t\rightarrow\infty}\frac{p_n(xt;\frac{1}{2}(-N+i\nu-2t),\frac{1}{2}(-N-i\nu+2t),
\frac{1}{2}(-N+i\nu-2t),\frac{1}{2}(-N-i\nu+2t))}{t^n}\nonumber\\
& &{}=\frac{(n-2N-1)_n}{n!}P_n(x;\nu,N).
\end{eqnarray}

\subsection*{Remark}
Since we have for $k<n$
$$\frac{(a+b)_n(a+c)_n}{(a+b)_k(a+c)_k}=(a+b+k)_{n-k}(a+c+k)_{n-k},$$
the continuous Hahn polynomials defined by (\ref{DefContHahn}) can also be seen as polynomials
in the parameters $a$, $b$ and $c$.
% RS: add begin\label{sec9.4}
%
\paragraph{\large\bf KLSadd: Orthogonality relation and symmetry}The orthogonality relation (9.4.2) holds under the more general assumption that
$\Re(a,b,c,d)>0$ and $(c,d)=(\overline a,\overline b)$ or $(\overline b,\overline a)$.\\
Thus, under these assumptions, the continuous Hahn polynomial
$p_n(x;a,b,c,d)$
is symmetric in $a,b$ and in $c,d$.
This follows from the orthogonality relation (9.4.2)
together with the value of its coefficient of $x^n$ given in (9.4.4b).\\
As a consequence, it is sufficient to give generating function (9.4.11). Then the generating
function (9.4.12) will follow by symmetry in the parameters.
%
\paragraph{\large\bf KLSadd: Uniqueness of orthogonality measure}The coefficient of $p_{n-1}(x)$ in (9.4.4) behaves as $O(n^2)$ as $n\to\iy$.
Hence \eqref{93} holds, by which the orthogonality measure is unique.
%
\paragraph{\large\bf KLSadd: Special cases}In the following special case there is a reduction to
Meixner-Pollaczek:
\begin{equation}
p_n(x;a,a+\thalf,a,a+\thalf)=
\frac{(2a)_n (2a+\thalf)_n}{(4a)_n}\,P_n^{(2a)}(2x;\thalf\pi).
\end{equation}
See \myciteKLS{342}{(2.6)} (note that in \myciteKLS{342}{(2.3)} the
Meixner-Pollaczek polynonmials are defined different from (9.7.1),
without a constant factor in front).

For $0<a<1$ the continuous Hahn polynomials $p_n(x;a,1-a,a,1-a)$
are orthogonal on $(-\iy,\iy)$ with respect to the weight function
$\big(\cosh(2\pi x)-\cos(2\pi a)\big)^{-1}$
(by straightforward computation from (9.4.2)).
For $a=\tfrac14$ the two special cases coincide:
Meixner-Pollaczek with weight function $\big(\cosh(2\pi x)\big)^{-1}$.
%
% RS: add end
\subsection*{References}
\cite{Askey85}, \cite{Askey89I}, \cite{AtakRahmanSuslov},
\cite{AtakSuslov85}, \cite{Badertscher}, \cite{Gupta91}, \cite{Koelink96II},
\cite{Koorn88}, \cite{Lesky94II}, \cite{Lesky95II}, \cite{Lesky97}.


\section{Hahn}\index{Hahn polynomials}

\par\setcounter{equation}{0}

\subsection*{Hypergeometric representation}
\begin{equation}
\label{DefHahn}
Q_n(x;\alpha,\beta,N)=\hyp{3}{2}{-n,n+\alpha+\beta+1,-x}{\alpha+1,-N}{1},\quad n=0,1,2,\ldots,N.
\end{equation}

\subsection*{Orthogonality relation}
For $\alpha>-1$ and $\beta>-1$, or for $\alpha<-N$ and $\beta<-N$, we have
\begin{eqnarray}
\label{OrtHahn}
& &\sum_{x=0}^N\binom{\alpha +x}{x}\binom{\beta+N-x}{N-x}Q_m(x;\alpha,\beta,N)Q_n(x;\alpha,\beta,N)\nonumber\\
& &{}=\frac{(-1)^n(n+\alpha+\beta+1)_{N+1}(\beta+1)_nn!}{(2n+\alpha+\beta+1)(\alpha+1)_n(-N)_nN!}\,\delta_{mn}.
\end{eqnarray}

\subsection*{Recurrence relation}
\begin{equation}
\label{RecHahn}
-xQ_n(x)=A_nQ_{n+1}(x)-\left(A_n+C_n\right)Q_n(x)+C_nQ_{n-1}(x),
\end{equation}
where
$$Q_n(x):=Q_n(x;\alpha,\beta,N)$$
and
$$\left\{\begin{array}{l}
\displaystyle A_n=\frac{(n+\alpha+\beta+1)(n+\alpha+1)(N-n)}{(2n+\alpha+\beta+1)(2n+\alpha+\beta+2)}\\[5mm]
\displaystyle C_n=\frac{n(n+\alpha+\beta+N+1)(n+\beta)}{(2n+\alpha+\beta)(2n+\alpha+\beta+1)}.
\end{array}\right.$$

\subsection*{Normalized recurrence relation}
\begin{equation}
\label{NormRecHahn}
xp_n(x)=p_{n+1}(x)+\left(A_n+C_n\right)p_n(x)+A_{n-1}C_np_{n-1}(x),
\end{equation}
where
$$Q_n(x;\alpha,\beta,N)=\frac{(n+\alpha+\beta+1)_n}{(\alpha+1)_n(-N)_n}p_n(x).$$

\subsection*{Difference equation}
\begin{equation}
\label{dvHahn}
n(n+\alpha+\beta+1)y(x)=B(x)y(x+1)-\left[B(x)+D(x)\right]y(x)+D(x)y(x-1),
\end{equation}
where
$$y(x)=Q_n(x;\alpha,\beta,N)$$
and
$$\left\{\begin{array}{l}
\displaystyle B(x)=(x+\alpha+1)(x-N)\\[5mm]
\displaystyle D(x)=x(x-\beta-N-1).
\end{array}\right.$$

\subsection*{Forward shift operator}
\begin{eqnarray}
\label{shift1HahnI}
& &Q_n(x+1;\alpha,\beta,N)-Q_n(x;\alpha,\beta,N)\nonumber\\
& &{}=-\frac{n(n+\alpha+\beta+1)}{(\alpha+1)N}Q_{n-1}(x;\alpha+1,\beta+1,N-1)
\end{eqnarray}
or equivalently
\begin{equation}
\label{shift1HahnII}
\Delta Q_n(x;\alpha,\beta,N)=-\frac{n(n+\alpha+\beta+1)}{(\alpha+1)N}Q_{n-1}(x;\alpha+1,\beta+1,N-1).
\end{equation}

\subsection*{Backward shift operator}
\begin{eqnarray}
\label{shift2HahnI}
& &(x+\alpha)(N+1-x)Q_n(x;\alpha,\beta,N)-x(\beta+N+1-x)Q_n(x-1;\alpha,\beta,N)\nonumber\\
& &{}=\alpha(N+1)Q_{n+1}(x;\alpha-1,\beta-1,N+1)
\end{eqnarray}
or equivalently
\begin{eqnarray}
\label{shift2HahnII}
& &\nabla\left[\omega(x;\alpha,\beta,N)Q_n(x;\alpha,\beta,N)\right]\nonumber\\
& &{}=\frac{N+1}{\beta}\omega(x;\alpha-1,\beta-1,N+1)Q_{n+1}(x;\alpha-1,\beta-1,N+1),
\end{eqnarray}
where
$$\omega(x;\alpha,\beta,N)=\binom{\alpha+x}{x}\binom{\beta+N-x}{N-x}.$$

\subsection*{Rodrigues-type formula}
\begin{eqnarray}
\label{RodHahn}
& &\omega(x;\alpha,\beta,N)Q_n(x;\alpha,\beta,N)\nonumber\\
& &{}=\frac{(-1)^n(\beta+1)_n}{(-N)_n}\nabla^n\left[\omega(x;\alpha+n,\beta+n,N-n)\right].
\end{eqnarray}

\subsection*{Generating functions}
For $x=0,1,2,\ldots,N$ we have
\begin{equation}
\label{GenHahn1}
\hyp{1}{1}{-x}{\alpha+1}{-t}\,\hyp{1}{1}{x-N}{\beta+1}{t}
=\sum_{n=0}^N\frac{(-N)_n}{(\beta+1)_nn!}Q_n(x;\alpha,\beta,N)t^n.
\end{equation}

\begin{eqnarray}
\label{GenHahn2}
& &\hyp{2}{0}{-x,-x+\beta+N+1}{-}{-t}\,\hyp{2}{0}{x-N,x+\alpha+1}{-}{t}\nonumber\\
& &{}=\sum_{n=0}^N\frac{(-N)_n(\alpha+1)_n}{n!}Q_n(x;\alpha,\beta,N)t^n.
\end{eqnarray}

\begin{eqnarray}
\label{GenHahn3}
& &\left[(1-t)^{-\alpha-\beta-1}\,\hyp{3}{2}{\frac{1}{2}(\alpha+\beta+1),\frac{1}{2}(\alpha+\beta+2),-x}
{\alpha+1,-N}{-\frac{4t}{(1-t)^2}}\right]_N\nonumber\\
& &{}=\sum_{n=0}^N\frac{(\alpha+\beta+1)_n}{n!}Q_n(x;\alpha,\beta,N)t^n.
\end{eqnarray}

\subsection*{Limit relations}

\subsubsection*{Racah $\rightarrow$ Hahn}
If we take $\gamma+1=-N$ and let $\delta\rightarrow\infty$ in the definition (\ref{DefRacah})
of the Racah polynomials, we obtain the Hahn polynomials. Hence
$$\lim_{\delta\rightarrow\infty}
R_n(\lambda(x);\alpha,\beta,-N-1,\delta)=Q_n(x;\alpha,\beta,N).$$
And if we take $\delta=-\beta-N-1$ and let $\gamma\rightarrow\infty$ in the definition (\ref{DefRacah})
of the Racah polynomials, we also obtain the Hahn polynomials:
$$\lim_{\gamma\rightarrow\infty}
R_n(\lambda(x);\alpha,\beta,\gamma,-\beta-N-1)=Q_n(x;\alpha,\beta,N).$$
Another way to do this is to take $\alpha+1=-N$ and $\beta\rightarrow\beta+\gamma+N+1$ in
the definition (\ref{DefRacah}) of the Racah polynomials and then take the limit
$\delta\rightarrow\infty$. In that case we obtain the Hahn polynomials in the following way:
$$\lim_{\delta\rightarrow\infty}
R_n(\lambda(x);-N-1,\beta+\gamma+N+1,\gamma,\delta)=Q_n(x;\gamma,\beta,N).$$

\subsubsection*{Hahn $\rightarrow$ Jacobi}
To find the Jacobi polynomials given by (\ref{DefJacobi}) from the Hahn polynomials we take
$x\rightarrow Nx$ and let $N\rightarrow\infty$. In fact we have
\begin{equation}
\lim_{N\rightarrow\infty}
Q_n(Nx;\alpha,\beta,N)=\frac{P_n^{(\alpha,\beta)}(1-2x)}{P_n^{(\alpha,\beta)}(1)}.
\end{equation}

\subsubsection*{Hahn $\rightarrow$ Meixner}
The Meixner polynomials given by (\ref{DefMeixner}) can be obtained from the Hahn polynomials
by taking $\alpha=b-1$, $\beta=N(1-c)c^{-1}$ and letting $N\rightarrow\infty$:
\begin{equation}
\lim_{N\rightarrow\infty}
Q_n(x;b-1,N(1-c)c^{-1},N)=M_n(x;b,c).
\end{equation}

\subsubsection*{Hahn $\rightarrow$ Krawtchouk}
The Krawtchouk polynomials given by (\ref{DefKrawtchouk}) are obtained from the Hahn polynomials
if we take $\alpha=pt$ and $\beta=(1-p)t$ and let $t\rightarrow\infty$:
\begin{equation}
\lim_{t\rightarrow\infty}Q_n(x;pt,(1-p)t,N)=K_n(x;p,N).
\end{equation}

\subsection*{Remark}
If we interchange the role of $x$ and $n$ in (\ref{DefHahn}) we obtain the
dual Hahn polynomials given by (\ref{DefDualHahn}).

\noindent
Since
$$Q_n(x;\alpha,\beta,N)=R_x(\lambda(n);\alpha,\beta,N)$$
we obtain the dual orthogonality relation for the Hahn polynomials from the
orthogonality relation (\ref{OrtDualHahn}) of the dual Hahn polynomials:
\begin{eqnarray*}
& &\sum_{n=0}^N\frac{(2n+\alpha+\beta+1)(\alpha+1)_n(-N)_nN!}{(-1)^n(n+\alpha+\beta+1)_{N+1}(\beta+1)_nn!}
Q_n(x;\alpha,\beta,N)Q_n(y;\alpha,\beta,N)\\
& &{}=\frac{\delta_{xy}}{\dbinom{\alpha+x}{x}\dbinom{\beta+N-x}{N-x}},\quad x,y \in \{0,1,2,\ldots,N\}.
\end{eqnarray*}
% RS: add begin\label{sec9.5}
%
\paragraph{\large\bf KLSadd: Special values}\begin{equation}
Q_n(0;\al,\be,N)=1,\quad
Q_n(N;\al,\be,N)=\frac{(-1)^n(\be+1)_n}{(\al+1)_n}\,.
\label{95}
\end{equation}
Use (9.5.1) and compare with (9.8.1) and \eqref{50}.

From (9.5.3) and \eqref{1} it follows that
\begin{equation}
Q_{2n}(N;\al,\al,2N)=\frac{(\thalf)_n(N+\al+1)_n}{(-N+\thalf)_n(\al+1)_n}\,.
\label{30}
\end{equation}
From (9.5.1) and \mycite{DLMF}{(15.4.24)} it follows that
\begin{equation}
Q_N(x;\al,\be,N)=\frac{(-N-\be)_x}{(\al+1)_x}\qquad(x=0,1,\ldots,N).
\label{44}
\end{equation}
%
\paragraph{\large\bf KLSadd: Symmetries}By the orthogonality relation (9.5.2):
\begin{equation}
\frac{Q_n(N-x;\al,\be,N)}{Q_n(N;\al,\be,N)}=Q_n(x;\be,\al,N),
\label{96}
\end{equation}
It follows from \eqref{97} and \eqref{45} that
\begin{equation}
\frac{Q_{N-n}(x;\al,\be,N)}{Q_N(x;\al,\be,N)}
=Q_n(x;-N-\be-1,-N-\al-1,N)
\qquad(x=0,1,\ldots,N).
\label{100}
\end{equation}
%
\paragraph{\large\bf KLSadd: Duality}The Remark on p.208 gives the duality between Hahn and dual Hahn polynomials:
%
\begin{equation}
Q_n(x;\al,\be,N)=R_x(n(n+\al+\be+1);\al,\be,N)\quad(n,x\in\{0,1,\ldots N\}).
\label{45}
\end{equation}
%
% RS: add end
\subsection*{References}
\cite{AlSalam90}, \cite{AndrewsAskey85}, \cite{Area+II}, \cite{Askey75},
\cite{Askey89I}, \cite{Askey2005}, \cite{AskeyGasper77}, \cite{AskeyWilson85},
\cite{AtakRahmanSuslov}, \cite{AtakSuslov88}, \cite{Chihara78},
\cite{Ciesielski}, \cite{Cooper+}, \cite{Dette95}, \cite{Dunkl76},
\cite{Dunkl78I}, \cite{Gasper73I}, \cite{Gasper74}, \cite{HoareRahman},
\cite{Ismail77}, \cite{Ismail2005II}, \cite{Karlin61}, \cite{Koorn81}, \cite{Koorn88},
\cite{LabelleYehI}, \cite{LabelleYehII}, \cite{Laine}, \cite{Lesky62},
\cite{Lesky88}, \cite{Lesky89}, \cite{Lesky94I}, \cite{Lesky95II},
\cite{LewanowiczII}, \cite{Neuman}, \cite{Nikiforov+}, \cite{NikiforovUvarov},
\cite{Rahman76III}, \cite{Rahman78I}, \cite{Rahman78II}, \cite{Rahman81III},
\cite{Sablonniere}, \cite{Stanton84}, \cite{Stanton90}, \cite{Wilson80}, \cite{Wilson70II},
\cite{Zarzo+}.


\section{Dual Hahn}\index{Dual Hahn polynomials}\index{Hahn polynomials!Dual}

\par\setcounter{equation}{0}

\subsection*{Hypergeometric representation}
\begin{equation}
\label{DefDualHahn}
R_n(\lambda(x);\gamma,\delta,N)=
\hyp{3}{2}{-n,-x,x+\gamma+\delta+1}{\gamma+1,-N}{1},\quad n=0,1,2,\ldots,N,
\end{equation}
where
$$\lambda(x)=x(x+\gamma+\delta+1).$$

\subsection*{Orthogonality relation}
For $\gamma>-1$ and $\delta>-1$, or for $\gamma<-N$ and $\delta<-N$, we have
\begin{eqnarray}
\label{OrtDualHahn}
& &\sum_{x=0}^N\frac{(2x+\gamma+\delta+1)(\gamma+1)_x(-N)_xN!}{(-1)^x(x+\gamma+\delta+1)_{N+1}(\delta+1)_xx!}
R_m(\lambda(x);\gamma,\delta,N)R_n(\lambda(x);\gamma,\delta,N)\nonumber\\
& &{}=\frac{\,\delta_{mn}}{\dbinom{\gamma+n}{n}\dbinom{\delta+N-n}{N-n}}.
\end{eqnarray}

\subsection*{Recurrence relation}
\begin{equation}
\label{RecDualHahn}
\lambda(x)R_n(\lambda(x))
=A_nR_{n+1}(\lambda(x))-\left(A_n+C_n\right)R_n(\lambda(x))+C_nR_{n-1}(\lambda(x)),
\end{equation}
where
$$R_n(\lambda(x)):=R_n(\lambda(x);\gamma,\delta,N)$$
and
$$\left\{\begin{array}{l}
\displaystyle A_n=(n+\gamma+1)(n-N)\\[5mm]
\displaystyle C_n=n(n-\delta-N-1).
\end{array}\right.$$

\subsection*{Normalized recurrence relation}
\begin{equation}
\label{NormRecDualHahn}
xp_n(x)=p_{n+1}(x)-(A_n+C_n)p_n(x)+A_{n-1}C_np_{n-1}(x),
\end{equation}
where
$$R_n(\lambda(x);\gamma,\delta,N)=\frac{1}{(\gamma+1)_n(-N)_n}p_n(\lambda(x)).$$

\subsection*{Difference equation}
\begin{equation}
\label{dvDualHahn}
-ny(x)=B(x)y(x+1)-\left[B(x)+D(x)\right]y(x)+D(x)y(x-1),
\end{equation}
where
$$y(x)=R_n(\lambda(x);\gamma,\delta,N)$$
and
$$\left\{\begin{array}{l}
\displaystyle B(x)=\frac{(x+\gamma+1)(x+\gamma+\delta+1)(N-x)}{(2x+\gamma+\delta+1)(2x+\gamma+\delta+2)}\\[5mm]
\displaystyle D(x)=\frac{x(x+\gamma+\delta+N+1)(x+\delta)}{(2x+\gamma+\delta)(2x+\gamma+\delta+1)}.
\end{array}\right.$$

\subsection*{Forward shift operator}
\begin{eqnarray}
\label{shift1DualHahnI}
& &R_n(\lambda(x+1);\gamma,\delta,N)-R_n(\lambda(x);\gamma,\delta,N)\nonumber\\
& &{}=-\frac{n(2x+\gamma+\delta+2)}{(\gamma+1)N}R_{n-1}(\lambda(x);\gamma+1,\delta,N-1)
\end{eqnarray}
or equivalently
\begin{equation}
\label{shift1DualHahnII}
\frac{\Delta R_n(\lambda(x);\gamma,\delta,N)}{\Delta\lambda(x)}=
-\frac{n}{(\gamma+1)N}R_{n-1}(\lambda(x);\gamma+1,\delta,N-1).
\end{equation}

\subsection*{Backward shift operator}
\begin{eqnarray}
\label{shift2DualHahnI}
& &(x+\gamma)(x+\gamma+\delta)(N+1-x)R_n(\lambda(x);\gamma,\delta,N)\nonumber\\
& &{}\mathindent{}-x(x+\gamma+\delta+N+1)(x+\delta)R_n(\lambda(x-1);\gamma,\delta,N)\nonumber\\
& &{}=\gamma(N+1)(2x+\gamma+\delta)R_{n+1}(\lambda(x);\gamma-1,\delta,N+1)
\end{eqnarray}
or equivalently
\begin{eqnarray}
\label{shift2DualHahnII}
& &\frac{\nabla\left[\omega(x;\gamma,\delta,N)R_n(\lambda(x);\gamma,\delta,N)\right]}{\nabla\lambda(x)}\nonumber\\
& &{}=\frac{1}{\gamma+\delta}\omega(x;\gamma-1,\delta,N+1)R_{n+1}(\lambda(x);\gamma-1,\delta,N+1),
\end{eqnarray}
where
$$\omega(x;\gamma,\delta,N)=\frac{(-1)^x(\gamma+1)_x(\gamma+\delta+1)_x(-N)_x}
{(\gamma+\delta+N+2)_x(\delta+1)_xx!}.$$

\newpage

\subsection*{Rodrigues-type formula}
\begin{equation}
\label{RodDualHahn}
\omega(x;\gamma,\delta,N)R_n(\lambda(x);\gamma,\delta,N)
=(\gamma+\delta+1)_n\left(\nabla_{\lambda}\right)^n\left[\omega(x;\gamma+n,\delta,N-n)\right],
\end{equation}
where
$$\nabla_{\lambda}:=\frac{\nabla}{\nabla\lambda(x)}.$$

\subsection*{Generating functions}
For $x=0,1,2,\ldots,N$ we have
\begin{equation}
\label{GenDualHahn1}
(1-t)^{N-x}\,\hyp{2}{1}{-x,-x-\delta}{\gamma+1}{t}
=\sum_{n=0}^N\frac{(-N)_n}{n!}R_n(\lambda(x);\gamma,\delta,N)t^n.
\end{equation}

\begin{eqnarray}
\label{GenDualHahn2}
& &(1-t)^x\,\hyp{2}{1}{x-N,x+\gamma+1}{-\delta-N}{t}\nonumber\\
& &=\sum_{n=0}^N\frac{(\gamma+1)_n(-N)_n}{(-\delta-N)_nn!}R_n(\lambda(x);\gamma,\delta,N)t^n.
\end{eqnarray}

\begin{equation}
\label{GenDualHahn3}
\left[\expe^t\,\hyp{2}{2}{-x,x+\gamma+\delta+1}{\gamma+1,-N}{-t}\right]_N=\sum_{n=0}^N
\frac{R_n(\lambda(x);\gamma,\delta,N)}{n!}t^n.
\end{equation}

\begin{eqnarray}
\label{GenDualHahn4}
& &\left[(1-t)^{-\epsilon}\,\hyp{3}{2}{\epsilon,-x,x+\gamma+\delta+1}{\gamma+1,-N}{\frac{t}{t-1}}\right]_N\nonumber\\
& &{}=\sum_{n=0}^N\frac{(\epsilon)_n}{n!}R_n(\lambda(x);\gamma,\delta,N)t^n,
\quad\textrm{$\epsilon$ arbitrary}.
\end{eqnarray}

\subsection*{Limit relations}

\subsubsection*{Racah $\rightarrow$ Dual Hahn}
If we take $\alpha+1=-N$ and let $\beta\rightarrow\infty$ in the definition (\ref{DefRacah})
of the Racah polynomials, then we obtain the dual Hahn polynomials:
$$\lim_{\beta\rightarrow\infty}
R_n(\lambda(x);-N-1,\beta,\gamma,\delta)=R_n(\lambda(x);\gamma,\delta,N).$$
And if we take $\beta=-\delta-N-1$ and let $\alpha\rightarrow\infty$ in the definition (\ref{DefRacah})
of the Racah polynomials, then we also obtain the dual Hahn polynomials:
$$\lim_{\alpha\rightarrow\infty}
R_n(\lambda(x);\alpha,-\delta-N-1,\gamma,\delta)=R_n(\lambda(x);\gamma,\delta,N).$$
Finally, if we take $\gamma+1=-N$ and $\delta\rightarrow\alpha+\delta+N+1$ in the definition
(\ref{DefRacah}) of the Racah polynomials and take the limit
$\beta\rightarrow\infty$ we find the dual Hahn polynomials in the following way:
$$\lim_{\beta\rightarrow\infty}
R_n(\lambda(x);\alpha,\beta,-N-1,\alpha+\delta+N+1)=R_n(\lambda(x);\alpha,\delta,N).$$

\subsubsection*{Dual Hahn $\rightarrow$ Meixner}
The Meixner polynomials given by (\ref{DefMeixner}) are obtained from the dual Hahn polynomials
if we take $\gamma=\beta-1$ and $\delta=N(1-c)c^{-1}$ and let $N\rightarrow\infty$:
\begin{equation}
\lim_{N\rightarrow\infty}
R_n(\lambda(x);\beta-1,N(1-c)c^{-1},N)=M_n(x;\beta,c).
\end{equation}

\subsubsection*{Dual Hahn $\rightarrow$ Krawtchouk}
The Krawtchouk polynomials given by (\ref{DefKrawtchouk}) can be obtained from the dual
Hahn polynomials by setting $\gamma=pt$, $\delta=(1-p)t$ and letting $t\rightarrow\infty$:
\begin{equation}
\lim_{t\rightarrow\infty}R_n(\lambda(x);pt,(1-p)t,N)=K_n(x;p,N).
\end{equation}

\subsection*{Remark}
If we interchange the role of $x$ and $n$ in the definition (\ref{DefDualHahn})
of the dual Hahn polynomials we obtain the Hahn polynomials given by (\ref{DefHahn}).

\noindent
Since
$$R_n(\lambda(x);\gamma,\delta,N)=Q_x(n;\gamma,\delta,N)$$
we obtain the dual orthogonality relation for the dual Hahn polynomials
from the orthogonality relation (\ref{OrtHahn}) for the Hahn polynomials:
\begin{eqnarray*}
& &\sum_{n=0}^N\binom{\gamma+n}{n}\binom{\delta+N-n}{N-n} R_n(\lambda(x);\gamma,\delta,N)R_n(\lambda(y);\gamma,\delta,N)\\
& &{}=\frac{(-1)^x(x+\gamma+\delta+1)_{N+1}(\delta+1)_xx!}
{(2x+\gamma+\delta+1)(\gamma+1)_x(-N)_xN!}\delta_{xy},\quad x,y \in \{0,1,2,\ldots,N\}.
\end{eqnarray*}
% RS: add begin\label{sec9.6}
%
\paragraph{\large\bf KLSadd: Special values}By \eqref{44} and \eqref{45} we have
\begin{equation}
R_n(N(N+\ga+\de+1);\ga,\de,N)=\frac{(-N-\de)_n}{(\ga+1)_n}\,.
\label{47}
\end{equation}
It follows from \eqref{95} and \eqref{45} that
\begin{equation}
R_N(x(x+\ga+\de+1);\ga,\de,N)
=\frac{(-1)^x(\de+1)_x}{(\ga+1)_x}\qquad(x=0,1,\ldots,N).
\label{101}
\end{equation}
%
\paragraph{\large\bf KLSadd: Symmetries}Write the weight in (9.6.2) as
\begin{equation}
w_x(\al,\be,N):=N!\,\frac{2x+\ga+\de+1}{(x+\ga+\de+1)_{N+1}}\,
\frac{(\ga+1)_x}{(\de+1)_x}\,\binom Nx.
\label{98}
\end{equation}
Then
\begin{equation}
(\de+1)_N\,w_{N-x}(\ga,\de,N)=
(-\ga-N)_N\,w_x(-\de-N-1,-\ga-N-1,N).
\label{99}
\end{equation}
Hence, by (9.6.2),
\begin{equation}
\frac{R_n((N-x)(N-x+\ga+\de+1);\ga,\de,N)}{R_n(N(N+\ga+\de+1);\ga,\de,N)}
=R_n(x(x-2N-\ga-\de-1);-N-\de-1,-N-\ga-1,N).
\label{97}
\end{equation}
Alternatively, \eqref{97} follows from (9.6.1) and
\mycite{DLMF}{(16.4.11)}.

It follows from \eqref{96} and \eqref{45} that
\begin{equation}
\frac{R_{N-n}(x(x+\ga+\de+1);\ga,\de,N)}
{R_N(x(x+\ga+\de+1);\ga,\de,N)}
=R_n(x(x+\ga+\de+1);\de,\ga,N)\qquad(x=0,1,\ldots,N).
\label{102}
\end{equation}
%
\paragraph{\large\bf KLSadd: Re: (9.6.11).}The generating function (9.6.11) can be written in a more conceptual way as
\begin{equation}
(1-t)^x\,\hyp21{x-N,x+\ga+1}{-\de-N}t=\frac{N!}{(\de+1)_N}\,
\sum_{n=0}^N \om_n\,R_n(\la(x);\ga,\de,N)\,t^n,
\label{2}
\end{equation}
where
\begin{equation}
\om_n:=\binom{\ga+n}n \binom{\de+N-n}{N-n},
\label{3}
\end{equation}
i.e., the denominator on the \RHS\ of (9.6.2).
By the duality between Hahn polynomials and dual Hahn polynomials (see \eqref{45}) the above generating function can be rewritten in
terms of Hahn polynomials:
\begin{equation}
(1-t)^n\,\hyp21{n-N,n+\al+1}{-\be-N}t=\frac{N!}{(\be+1)_N}\,
\sum_{x=0}^N w_x\,Q_n(x;\al,\be,N)\,t^x,
\label{4}
\end{equation}
where
\begin{equation}
w_x:=\binom{\al+x}x \binom{\be+N-x}{N-x},
\label{5}
\end{equation}
i.e., the weight occurring in the orthogonality relation (9.5.2)
for Hahn polynomials.
\paragraph{\large\bf KLSadd: Re: (9.6.15).}There should be a closing bracket before the equality sign.
%
% RS: add end
\subsection*{References}
\cite{Askey2005}, \cite{AskeyWilson85}, \cite{AtakRahmanSuslov}, \cite{AtakSuslov88},
\cite{Ismail2005II}, \cite{Karlin61}, \cite{Koorn81}, \cite{Koorn88}, \cite{Lesky93},
\cite{Lesky94I}, \cite{Lesky95I}, \cite{Lesky95II}, \cite{Nikiforov+}, \cite{NikiforovUvarov},
\cite{Rahman81II}, \cite{Stanton84}, \cite{Wilson80}.


\section{Meixner-Pollaczek}\index{Meixner-Pollaczek polynomials}

\par\setcounter{equation}{0}

\subsection*{Hypergeometric representation}
\begin{equation}
\label{DefMP}
P_n^{(\lambda)}(x;\phi)=\frac{(2\lambda)_n}{n!}\expe^{in\phi}\,
\hyp{2}{1}{-n,\lambda+ix}{2\lambda}{1-\expe^{-2i\phi}}.
\end{equation}

\subsection*{Orthogonality relation}
\begin{equation}
\label{OrtMP}
\frac{1}{2\pi}\int_{-\infty}^{\infty}
\e^{(2\phi-\pi)x}\left|\Gamma(\lambda+ix)\right|^2
P_m^{(\lambda)}(x;\phi)P_n^{(\lambda)}(x;\phi)\,dx
{}=\frac{\Gamma(n+2\lambda)}{(2\sin\phi)^{2\lambda}n!}\,\delta_{mn},
%  \constraint{
%    $\lambda > 0$ &
%    $0 < \phi < \pi$ }
\end{equation}

\subsection*{Recurrence relation}
\begin{eqnarray}
\label{RecMP}
& &(n+1)P_{n+1}^{(\lambda)}(x;\phi)-2\left[x\sin\phi+(n+\lambda)\cos\phi\right]
P_n^{(\lambda)}(x;\phi)\nonumber\\
& &{}\mathindent{}+(n+2\lambda-1)P_{n-1}^{(\lambda)}(x;\phi)=0.
\end{eqnarray}

\subsection*{Normalized recurrence relation}
\begin{equation}
\label{NormRecMP}
xp_n(x)=p_{n+1}(x)-\left(\frac{n+\lambda}{\tan\phi}\right)p_n(x)+
\frac{n(n+2\lambda-1)}{4\sin^2\phi}p_{n-1}(x),
\end{equation}
where
$$P_n^{(\lambda)}(x;\phi)=\frac{(2\sin\phi)^n}{n!}p_n(x).$$

\subsection*{Difference equation}
\begin{eqnarray}
\label{dvMP}
& &\e^{i\phi}(\lambda-ix)y(x+i)+2i\left[x\cos\phi-(n+\lambda)\sin\phi\right]y(x)\nonumber\\
& &{}\mathindent{}-\e^{-i\phi}(\lambda+ix)y(x-i)=0,\quad y(x)=P_n^{(\lambda)}(x;\phi).
\end{eqnarray}

\subsection*{Forward shift operator}
\begin{equation}
\label{shift1MPI}
P_n^{(\lambda)}(x+\textstyle\frac{1}{2}i;\phi)-
P_n^{(\lambda)}(x-\textstyle\frac{1}{2}i;\phi)=(\expe^{i\phi}-\expe^{-i\phi})
P_{n-1}^{(\lambda+\frac{1}{2})}(x;\phi)
\end{equation}
or equivalently
\begin{equation}
\label{shift1MPII}
\frac{\delta P_n^{(\lambda)}(x;\phi)}{\delta x}
=2\sin\phi\,P_{n-1}^{(\lambda+\frac{1}{2})}(x;\phi).
\end{equation}

\subsection*{Backward shift operator}
\begin{eqnarray}
\label{shift2MPI}
& &\e^{i\phi}(\lambda-\textstyle\frac{1}{2}-ix)
P_n^{(\lambda)}(x+\textstyle\frac{1}{2}i;\phi)+
\e^{-i\phi}(\lambda-\textstyle\frac{1}{2}+ix)
P_n^{(\lambda)}(x-\textstyle\frac{1}{2}i;\phi)\nonumber\\
& &{}=(n+1)P_{n+1}^{(\lambda-\frac{1}{2})}(x;\phi)
\end{eqnarray}
or equivalently
\begin{equation}
\label{shift2MPII}
\frac{\delta\left[\omega(x;\lambda,\phi)P_n^{(\lambda)}(x;\phi)\right]}{\delta x}=
-(n+1)\omega(x;\lambda-\textstyle\frac{1}{2},\phi)
P_{n+1}^{(\lambda-\frac{1}{2})}(x;\phi),
\end{equation}
where
$$\omega(x;\lambda,\phi)=\Gamma(\lambda+ix)\Gamma(\lambda-ix)\e^{(2\phi-\pi)x}.$$

\subsection*{Rodrigues-type formula}
\begin{equation}
\label{RodMP}
\omega(x;\lambda,\phi)P_n^{(\lambda)}(x;\phi)=\frac{(-1)^n}{n!}
\left(\frac{\delta}{\delta x}\right)^n\left[\omega(x;\lambda+\textstyle\frac{1}{2}n,\phi)\right].
\end{equation}

\newpage

\subsection*{Generating functions}
\begin{equation}
\label{GenMP1}
(1-\expe^{i\phi}t)^{-\lambda+ix}(1-\expe^{-i\phi}t)^{-\lambda-ix}=
\sum_{n=0}^{\infty}P_n^{(\lambda)}(x;\phi)t^n.
\end{equation}

\begin{equation}
\label{GenMP2}
\e^t\,\hyp{1}{1}{\lambda+ix}{2\lambda}{(\expe^{-2i\phi}-1)t}=
\sum_{n=0}^{\infty}\frac{P_n^{(\lambda)}(x;\phi)}{(2\lambda)_n\expe^{in\phi}}t^n.
\end{equation}

\begin{eqnarray}
\label{GenMP3}
& &(1-t)^{-\gamma}\,\hyp{2}{1}{\gamma,\lambda+ix}{2\lambda}{\frac{(1-\e^{-2i\phi})t}{t-1}}\nonumber\\
& &{}=\sum_{n=0}^{\infty}\frac{(\gamma)_n}{(2\lambda)_n}\frac{P_n^{(\lambda)}(x;\phi)}{\e^{in\phi}}t^n,
\quad\textrm{$\gamma$ arbitrary}.
\end{eqnarray}

\subsection*{Limit relations}

\subsubsection*{Continuous dual Hahn $\rightarrow$ Meixner-Pollaczek}
The Meixner-Pollaczek polynomials can be obtained from the continuous dual Hahn polynomials
given by (\ref{DefContDualHahn}) by the substitutions $x\rightarrow x-t$, $a=\lambda+it$,
$b=\lambda-it$ and $c=t\cot\phi$ and the limit $t\rightarrow\infty$:
$$\lim_{t\rightarrow\infty}\frac{S_n((x-t)^2;\lambda+it,\lambda-it,t\cot\phi)}{t^nn!}
=\frac{P_n^{(\lambda)}(x;\phi)}{(\sin\phi)^n}.$$

\subsubsection*{Continuous Hahn $\rightarrow$ Meixner-Pollaczek}
By setting $x\rightarrow x+t$, $a=\lambda-it$, $c=\lambda+it$ and $b=d=t\tan\phi$ in the
definition (\ref{DefContHahn}) of the continuous Hahn polynomials and taking the limit
$t\rightarrow\infty$ we obtain the Meixner-Pollaczek polynomials:
$$\lim_{t\rightarrow\infty}\frac{p_n(x+t;\lambda-it,t\tan\phi,\lambda+it,t\tan\phi)}{t^nn!}
=\frac{P_n^{(\lambda)}(x;\phi)}{(\cos\phi)^n}.$$

\newpage

\subsubsection*{Meixner-Pollaczek $\rightarrow$ Laguerre}
The Laguerre polynomials given by (\ref{DefLaguerre}) can be obtained from the
Meixner-Pollaczek polynomials by the substitution $\lambda=\frac{1}{2}(\alpha+1)$,
$x\rightarrow -\frac{1}{2}\phi^{-1}x$ and the limit $\phi\rightarrow 0$:
\begin{equation}
\lim_{\phi\rightarrow 0}
P_n^{(\frac{1}{2}\alpha+\frac{1}{2})}(-\textstyle\frac{1}{2}\phi^{-1}x;\phi)=L_n^{(\alpha)}(x).
\end{equation}

\subsubsection*{Meixner-Pollaczek $\rightarrow$ Hermite}
The Hermite polynomials given by (\ref{DefHermite}) are obtained from the Meixner-Pollaczek
polynomials if we substitute $x\rightarrow (\sin\phi)^{-1}(x\sqrt{\lambda}-\lambda\cos\phi)$
and then let $\lambda\rightarrow\infty$:
\begin{equation}
\lim_{\lambda\rightarrow\infty}
\lambda^{-\frac{1}{2}n}P_n^{(\lambda)}
((\sin\phi)^{-1}(x\sqrt{\lambda}-\lambda\cos\phi);\phi)=\frac{H_n(x)}{n!}.
\end{equation}

\subsection*{Remark}
Since we have for $k<n$
$$\frac{(2\lambda)_n}{(2\lambda)_k}=(2\lambda+k)_{n-k},$$
the Meixner-Pollaczek polynomials defined by (\ref{DefMP}) can also be seen as polynomials
in the parameter $\lambda$.
% RS: add begin\label{sec9.7}
%
\paragraph{\large\bf KLSadd: Uniqueness of orthogonality measure}The coefficient of $p_{n-1}(x)$ in (9.7.4) behaves as $O(n^2)$ as $n\to\iy$.
Hence \eqref{93} holds, by which the orthogonality measure is unique.
%
% RS: add end
\subsection*{References}
\cite{AlSalam90}, \cite{AlSalamChihara76}, \cite{AndrewsAskey85}, \cite{Araaya2004},
\cite{Araaya2005}, \cite{Askey89I}, \cite{AskeyWilson85}, \cite{AtakRahmanSuslov},
\cite{AtakSuslov88}, \cite{CharrisIsmail}, \cite{ChenIsmail97}, \cite{Chihara78},
\cite{Ismail85II}, \cite{Ismail94}, \cite{Ismail2005II}, \cite{IsmailLi},
\cite{IsmailStanton97}, \cite{Koekoek2000}, \cite{Koorn88}, \cite{Koorn89II},
\cite{LabelleYehI}, \cite{LabelleYehII}, \cite{Lesky95II}, \cite{LiWong}, \cite{Meixner},
\cite{Nikiforov+}, \cite{Rahman78I}, \cite{Wilson80}, \cite{Wimp90}.


\section{Jacobi}\index{Jacobi polynomials}

\par\setcounter{equation}{0}

\subsection*{Hypergeometric representation}
\begin{equation}
\label{DefJacobi}
P_n^{(\alpha,\beta)}(x)=\frac{(\alpha+1)_n}{n!}\,
\hyp{2}{1}{-n,n+\alpha+\beta+1}{\alpha+1}{\frac{1-x}{2}}.
\end{equation}

\subsection*{Orthogonality relation}
For $\alpha>-1$ and $\beta>-1$ we have
\begin{eqnarray}
\label{OrtJacobi1}
& &\int_{-1}^1(1-x)^{\alpha}(1+x)^{\beta}P_m^{(\alpha,\beta)}(x)P_n^{(\alpha,\beta)}(x)\,dx\nonumber\\
& &{}=\frac{2^{\alpha+\beta+1}}{2n+\alpha+\beta+1}\frac{\Gamma(n+\alpha+1)\Gamma(n+\beta+1)}{\Gamma(n+\alpha+\beta+1)n!}\,\delta_{mn}.
\end{eqnarray}
For $\alpha+\beta<-2N-1$, $\beta>-1$ and $m,n\in\{0,1,2,\ldots,N\}$ we also have
\begin{eqnarray}
\label{OrtJacobi2}
& &\int_1^{\infty}(x+1)^{\alpha}(x-1)^{\beta}P_m^{(\alpha,\beta)}(-x)P_n^{(\alpha,\beta)}(-x)\,dx\nonumber\\
& &{}=-\frac{2^{\alpha+\beta+1}}{2n+\alpha+\beta+1}\frac{\Gamma(-n-\alpha-\beta)\Gamma(n+\alpha+\beta+1)}{\Gamma(-n-\alpha)n!}\,\delta_{mn}.
\end{eqnarray}

\subsection*{Recurrence relation}
\begin{eqnarray}
\label{RecJacobi}
xP_n^{(\alpha,\beta)}(x)&=&\frac{2(n+1)(n+\alpha+\beta+1)}{(2n+\alpha+\beta+1)(2n+\alpha+\beta+2)}P_{n+1}^{(\alpha,\beta)}(x)\nonumber\\
& &{}\mathindent{}+\frac{\beta^2-\alpha^2}{(2n+\alpha+\beta)(2n+\alpha+\beta+2)}P_n^{(\alpha,\beta)}(x)\nonumber\\
& &{}\mathindent\mathindent{}+\frac{2(n+\alpha)(n+\beta)}{(2n+\alpha+\beta)(2n+\alpha+\beta+1)}P_{n-1}^{(\alpha,\beta)}(x).
\end{eqnarray}

\subsection*{Normalized recurrence relation}
\begin{eqnarray}
\label{NormRecJacobi}
xp_n(x)&=&p_{n+1}(x)+\frac{\beta^2-\alpha^2}{(2n+\alpha+\beta)(2n+\alpha+\beta+2)}p_n(x)\nonumber\\
& &{}\mathindent{}+\frac{4n(n+\alpha)(n+\beta)(n+\alpha+\beta)}
{(2n+\alpha+\beta-1)(2n+\alpha+\beta)^2(2n+\alpha+\beta+1)}p_{n-1}(x)
\end{eqnarray}
where
$$P_n^{(\alpha,\beta)}(x)=\frac{(n+\alpha+\beta+1)_n}{2^nn!}p_n(x).$$

\subsection*{Differential equation}
\begin{eqnarray}
\label{dvJacobi}
& &(1-x^2)y''(x)+\left[\beta-\alpha-(\alpha+\beta+2)x\right]y'(x)\nonumber\\
& &{}\mathindent{}+n(n+\alpha+\beta+1)y(x)=0,\quad y(x)=P_n^{(\alpha,\beta)}(x).
\end{eqnarray}

\subsection*{Forward shift operator}
\begin{equation}
\label{shift1Jacobi}
\frac{d}{dx}P_n^{(\alpha,\beta)}(x)=\frac{n+\alpha+\beta+1}{2}P_{n-1}^{(\alpha+1,\beta+1)}(x).
\end{equation}

\subsection*{Backward shift operator}
\begin{eqnarray}
\label{shift2JacobiI}
& &(1-x^2)\frac{d}{dx}P_n^{(\alpha,\beta)}(x)+
\left[(\beta-\alpha)-(\alpha+\beta)x\right]P_n^{(\alpha,\beta)}(x)\nonumber\\
& &{}=-2(n+1)P_{n+1}^{(\alpha-1,\beta-1)}(x)
\end{eqnarray}
or equivalently
\begin{eqnarray}
\label{shift2JacobiII}
& &\frac{d}{dx}\left[(1-x)^\alpha(1+x)^\beta P_n^{(\alpha,\beta)}(x)\right]\nonumber\\
& &{}=-2(n+1)(1-x)^{\alpha-1}(1+x)^{\beta-1}P_{n+1}^{(\alpha-1,\beta-1)}(x).
\end{eqnarray}

\subsection*{Rodrigues-type formula}
\begin{equation}
\label{RodJacobi}
(1-x)^{\alpha}(1+x)^{\beta}P_n^{(\alpha,\beta)}(x)=
\frac{(-1)^n}{2^nn!}\left(\frac{d}{dx}\right)^n
\left[(1-x)^{n+\alpha}(1+x)^{n+\beta}\right].
\end{equation}

\subsection*{Generating functions}
\begin{equation}
\label{GenJacobi1}
\frac{2^{\alpha+\beta}}{R(1+R-t)^{\alpha}(1+R+t)^{\beta}}=
\sum_{n=0}^{\infty}P_n^{(\alpha,\beta)}(x)t^n,\quad R=\sqrt{1-2xt+t^2}.
\end{equation}

\begin{eqnarray}
\label{GenJacobi2}
& &\hyp{0}{1}{-}{\alpha+1}{\frac{(x-1)t}{2}}\,\hyp{0}{1}{-}{\beta+1}{\frac{(x+1)t}{2}}\nonumber\\
& &{}=\sum_{n=0}^{\infty}\frac{P_n^{(\alpha,\beta)}(x)}{(\alpha+1)_n(\beta+1)_n}t^n.
\end{eqnarray}

\begin{eqnarray}
\label{GenJacobi3}
& &(1-t)^{-\alpha-\beta-1}\,\hyp{2}{1}{\frac{1}{2}(\alpha+\beta+1),\frac{1}{2}(\alpha+\beta+2)}
{\alpha+1}{\frac{2(x-1)t}{(1-t)^2}}\nonumber\\
& &{}=\sum_{n=0}^{\infty}\frac{(\alpha+\beta+1)_n}{(\alpha+1)_n}P_n^{(\alpha,\beta)}(x)t^n.
\end{eqnarray}

\begin{eqnarray}
\label{GenJacobi4}
& &(1+t)^{-\alpha-\beta-1}\,\hyp{2}{1}{\frac{1}{2}(\alpha+\beta+1),\frac{1}{2}(\alpha+\beta+2)}
{\beta+1}{\frac{2(x+1)t}{(1+t)^2}}\nonumber\\
& &{}=\sum_{n=0}^{\infty}\frac{(\alpha+\beta+1)_n}{(\beta+1)_n}P_n^{(\alpha,\beta)}(x)t^n.
\end{eqnarray}

\begin{eqnarray}
\label{GenJacobi5}
& &\hyp{2}{1}{\gamma,\alpha+\beta+1-\gamma}{\alpha+1}{\frac{1-R-t}{2}}\,
\hyp{2}{1}{\gamma,\alpha+\beta+1-\gamma}{\beta+1}{\frac{1-R+t}{2}}\nonumber\\
& &{}=\sum_{n=0}^{\infty}
\frac{(\gamma)_n(\alpha+\beta+1-\gamma)_n}{(\alpha+1)_n(\beta+1)_n}P_n^{(\alpha,\beta)}(x)t^n,
\quad R=\sqrt{1-2xt+t^2}
\end{eqnarray}
with $\gamma$ arbitrary.

\subsection*{Limit relations}

\subsubsection*{Wilson $\rightarrow$ Jacobi}
The Jacobi polynomials can be found from the Wilson polynomials given by (\ref{DefWilson}) by
substituting $a=b=\frac{1}{2}(\alpha+1)$, $c=\frac{1}{2}(\beta+1)+it$,
$d=\frac{1}{2}(\beta+1)-it$ and $x\rightarrow t\sqrt{\frac{1}{2}(1-x)}$ in the
definition (\ref{DefWilson}) of the Wilson polynomials and taking the limit
$t\rightarrow\infty$. In fact we have
$$\lim_{t\rightarrow\infty}
\frac{W_n(\frac{1}{2}(1-x)t^2;\frac{1}{2}(\alpha+1),
\frac{1}{2}(\alpha+1),\frac{1}{2}(\beta+1)+it,\frac{1}{2}(\beta+1)-it)}
{t^{2n}n!}=P_n^{(\alpha,\beta)}(x).$$

\subsubsection*{Continuous Hahn $\rightarrow$ Jacobi}
The Jacobi polynomials follow from the continuous Hahn polynomials given by (\ref{DefContHahn})
by using the substitution $x\rightarrow \frac{1}{2}xt$, $a=\frac{1}{2}(\alpha+1-it)$,
$b=\frac{1}{2}(\beta+1+it)$, $c=\frac{1}{2}(\alpha+1+it)$ and $d=\frac{1}{2}(\beta+1-it)$
in (\ref{DefContHahn}), division by $t^n$ and the limit $t\rightarrow\infty$:
$$\lim_{t\rightarrow\infty}
\frac{p_n(\frac{1}{2}xt;\frac{1}{2}(\alpha+1-it),\frac{1}{2}(\beta+1+it),
\frac{1}{2}(\alpha+1+it),\frac{1}{2}(\beta+1-it))}{t^n}=P_n^{(\alpha,\beta)}(x).$$

\subsubsection*{Hahn $\rightarrow$ Jacobi}
To find the Jacobi polynomials from the Hahn polynomials given by (\ref{DefHahn}) we take
$x\rightarrow Nx$ in (\ref{DefHahn}) and let $N\rightarrow\infty$. In fact we have
$$\lim_{N\rightarrow\infty}
Q_n(Nx;\alpha,\beta,N)=\frac{P_n^{(\alpha,\beta)}(1-2x)}{P_n^{(\alpha,\beta)}(1)}.$$

\subsubsection*{Jacobi $\rightarrow$ Laguerre}
The Laguerre polynomials given by (\ref{DefLaguerre}) can be obtained from the Jacobi
polynomials by setting $x\rightarrow 1-2\beta^{-1}x$ and then the limit $\beta\rightarrow\infty$:
\begin{equation}
\lim_{\beta\rightarrow\infty}
P_n^{(\alpha,\beta)}(1-2\beta^{-1}x)=L_n^{(\alpha)}(x).
\end{equation}

\subsubsection*{Jacobi $\rightarrow$ Bessel}
The Bessel polynomials given by (\ref{DefBessel}) are obtained from the Jacobi polynomials
if we take $\beta=a-\alpha$ and let $\alpha\rightarrow-\infty$:
\begin{equation}
\lim_{\alpha\rightarrow-\infty}
\frac{P_n^{(\alpha,a-\alpha)}(1+\alpha x)}{P_n^{(\alpha,a-\alpha)}(1)}=y_n(x;a).
\end{equation}

\subsubsection*{Jacobi $\rightarrow$ Hermite}
The Hermite polynomials given by (\ref{DefHermite}) follow from the Jacobi polynomials
by taking $\beta=\alpha$ and letting $\alpha\rightarrow\infty$ in the following way:
\begin{equation}
\lim_{\alpha\rightarrow\infty}
\alpha^{-\frac{1}{2}n}P_n^{(\alpha,\alpha)}(\alpha^{-\frac{1}{2}}x)=\frac{H_n(x)}{2^nn!}.
\end{equation}

\subsection*{Remarks}
The definition (\ref{DefJacobi}) of the Jacobi polynomials can also be written as:
$$P_n^{(\alpha,\beta)}(x)=\frac{1}{n!}\sum_{k=0}^n\frac{(-n)_k}{k!}
(n+\alpha+\beta+1)_k(\alpha+k+1)_{n-k}\left(\frac{1-x}{2}\right)^k.$$
In this way the Jacobi polynomials can also be seen as polynomials in the parameters $\alpha$
and $\beta$. Therefore they can be defined for all $\alpha$ and $\beta$. Then we have the
following connection with the Meixner polynomials given by (\ref{DefMeixner}):
$$\frac{(\beta)_n}{n!}M_n(x;\beta,c)=P_n^{(\beta-1,-n-\beta-x)}((2-c)c^{-1}).$$

\noindent
The Jacobi polynomials are related to the pseudo Jacobi polynomials defined by
(\ref{DefPseudoJacobi}) in the following way:
$$P_n(x;\nu,N)=\frac{(-2i)^nn!}{(n-2N-1)_n}P_n^{(-N-1+i\nu,-N-1-i\nu)}(ix).$$

\noindent
The Jacobi polynomials are also related to the Gegenbauer (or ultraspherical) polynomials
given by (\ref{DefGegenbauer}) by the quadratic transformations:
$$C_{2n}^{(\lambda)}(x)=\frac{(\lambda)_n}{(\frac{1}{2})_n}
P_n^{(\lambda-\frac{1}{2},-\frac{1}{2})}(2x^2-1)$$
and
$$C_{2n+1}^{(\lambda)}(x)=\frac{(\lambda)_{n+1}}{(\frac{1}{2})_{n+1}}
xP_n^{(\lambda-\frac{1}{2},\frac{1}{2})}(2x^2-1).$$
% RS: add begin\label{sec9.8}
%
\paragraph{\large\bf KLSadd: Orthogonality relation}Write the \RHS\ of (9.8.2) as $h_n\,\de_{m,n}$. Then
\begin{equation}
\begin{split}
&\frac{h_n}{h_0}=
\frac{n+\al+\be+1}{2n+\al+\be+1}\,
\frac{(\al+1)_n(\be+1)_n}{(\al+\be+2)_n\,n!}\,,\quad
h_0=\frac{2^{\al+\be+1}\Ga(\al+1)\Ga(\be+1)}{\Ga(\al+\be+2)}\,,\sLP
&\frac{h_n}{h_0\,(P_n^{(\al,\be)}(1))^2}=
\frac{n+\al+\be+1}{2n+\al+\be+1}\,
\frac{(\be+1)_n\,n!}{(\al+1)_n\,(\al+\be+2)_n}\,.
\end{split}
\label{60}
\end{equation}

In (9.8.3) the numerator factor $\Ga(n+\al+\be+1)$ in the last line should be
$\Ga(\be+1)$. When thus corrected, (9.8.3) can be rewritten as:
\begin{equation}
\begin{split}
&\int_1^\iy P_m^{(\al,\be)}(x)\,P_n^{(\al,\be)}(x)\,(x-1)^\al (x+1)^\be\,dx=h_n\,\de_{m,n}\,,\\
&\qquad\qquad\qquad\qquad\qquad\qquad\qquad\quad-1-\be>\al>-1,\quad m,n<-\thalf(\al+\be+1),\\
&\frac{h_n}{h_0}=
\frac{n+\al+\be+1}{2n+\al+\be+1}\,
\frac{(\al+1)_n(\be+1)_n}{(\al+\be+2)_n\,n!}\,,\quad
h_0=\frac{2^{\al+\be+1}\Ga(\al+1)\Ga(-\al-\be-1)}{\Ga(-\be)}\,.
\end{split}
\label{122}
\end{equation}

%
\paragraph{\large\bf KLSadd: Symmetry}\begin{equation}
P_n^{(\al,\be)}(-x)=(-1)^n\,P_n^{(\be,\al)}(x).
\label{48}
\end{equation}
Use (9.8.2) and (9.8.5b) or see \mycite{DLMF}{Table 18.6.1}.
%
\paragraph{\large\bf KLSadd: Special values}\begin{equation}
P_n^{(\al,\be)}(1)=\frac{(\al+1)_n}{n!}\,,\quad
P_n^{(\al,\be)}(-1)=\frac{(-1)^n(\be+1)_n}{n!}\,,\quad
\frac{P_n^{(\al,\be)}(-1)}{P_n^{(\al,\be)}(1)}=\frac{(-1)^n(\be+1)_n}{(\al+1)_n}\,.
\label{50}
\end{equation}
Use (9.8.1) and \eqref{48} or see \mycite{DLMF}{Table 18.6.1}.
%
\paragraph{\large\bf KLSadd: Generating functions}Formula (9.8.15) was first obtained by Brafman \myciteKLS{109}.
%
\paragraph{\large\bf KLSadd: Bilateral generating functions}For $0\le r<1$ and $x,y\in[-1,1]$ we have in terms of $F_4$ (see~\eqref{62}):
\begin{align}
&\sum_{n=0}^\iy\frac{(\al+\be+1)_n\,n!}{(\al+1)_n(\be+1)_n}\,r^n\,
P_n^{(\al,\be)}(x)\,P_n^{(\al,\be)}(y)
=\frac1{(1+r)^{\al+\be+1}}
\nonumber\\
&\qquad\quad\times F_4\Big(\thalf(\al+\be+1),\thalf(\al+\be+2);\al+1,\be+1;
\frac{r(1-x)(1-y)}{(1+r)^2},\frac{r(1+x)(1+y)}{(1+r)^2}\Big),
\label{58}\sLP
&\sum_{n=0}^\iy\frac{2n+\al+\be+1}{n+\al+\be+1}
\frac{(\al+\be+2)_n\,n!}{(\al+1)_n(\be+1)_n}\,r^n\,
P_n^{(\al,\be)}(x)\,P_n^{(\al,\be)}(y)
=\frac{1-r}{(1+r)^{\al+\be+2}}\nonumber\\
&\qquad\quad\times F_4\Big(\thalf(\al+\be+2),\thalf(\al+\be+3);\al+1,\be+1;
\frac{r(1-x)(1-y)}{(1+r)^2},\frac{r(1+x)(1+y)}{(1+r)^2}\Big).
\label{59}
\end{align}
Formulas \eqref{58} and \eqref{59} were first
given by Bailey \myciteKLS{91}{(2.1), (2.3)}.
See Stanton \myciteKLS{485} for a shorter proof.
(However, in the second line of
\myciteKLS{485}{(1)} $z$ and $Z$ should be interchanged.)$\;$
As observed in Bailey \myciteKLS{91}{p.10}, \eqref{59} follows
from \eqref{58}
by applying the operator $r^{-\half(\al+\be-1)}\,\frac d{dr}\circ r^{\half(\al+\be+1)}$
to both sides of \eqref{58}.
In view of \eqref{60}, formula \eqref{59} is the Poisson kernel for Jacobi
polynomials. The \RHS\ of \eqref{59} makes clear that this kernel is positive.
See also the discussion in Askey \myciteKLS{46}{following (2.32)}.
%
\paragraph{\large\bf KLSadd: Quadratic transformations}\begin{align}
\frac{C_{2n}^{(\al+\half)}(x)}{C_{2n}^{(\al+\half)}(1)}
=\frac{P_{2n}^{(\al,\al)}(x)}{P_{2n}^{(\al,\al)}(1)}
&=\frac{P_n^{(\al,-\half)}(2x^2-1)}{P_n^{(\al,-\half)}(1)}\,,
\label{51}\\
\frac{C_{2n+1}^{(\al+\half)}(x)}{C_{2n+1}^{(\al+\half)}(1)}
=\frac{P_{2n+1}^{(\al,\al)}(x)}{P_{2n+1}^{(\al,\al)}(1)}
&=\frac{x\,P_n^{(\al,\half)}(2x^2-1)}{P_n^{(\al,\half)}(1)}\,.
\label{52}
\end{align}
See p.221, Remarks, last two formulas together with \eqref{50} and \eqref{49}.
Or see \mycite{DLMF}{(18.7.13), (18.7.14)}.
%
\paragraph{\large\bf KLSadd: Differentiation formulas}Each differentiation formula is given in two equivalent forms.
\begin{equation}
\begin{split}
\frac d{dx}\left((1-x)^\al P_n^{(\al,\be)}(x)\right)&=
-(n+\al)\,(1-x)^{\al-1} P_n^{(\al-1,\be+1)}(x),\\
\left((1-x)\frac d{dx}-\al\right)P_n^{(\al,\be)}(x)&=
-(n+\al)\,P_n^{(\al-1,\be+1)}(x).
\end{split}
\label{68}
\end{equation}
%
\begin{equation}
\begin{split}
\frac d{dx}\left((1+x)^\be P_n^{(\al,\be)}(x)\right)&=
(n+\be)\,(1+x)^{\be-1} P_n^{(\al+1,\be-1)}(x),\\
\left((1+x)\frac d{dx}+\be\right)P_n^{(\al,\be)}(x)&=
(n+\be)\,P_n^{(\al+1,\be-1)}(x).
\end{split}
\label{69}
\end{equation}
Formulas \eqref{68} and \eqref{69} follow from
\mycite{DLMF}{(15.5.4), (15.5.6)}
together with (9.8.1). They also follow from each other by \eqref{48}.
%
\paragraph{\large\bf KLSadd: Generalized Gegenbauer polynomials}These are defined by
\begin{equation}
S_{2m}^{(\al,\be)}(x):=\const P_m^{(\al,\be)}(2x^2-1),\qquad
S_{2m+1}^{(\al,\be)}(x):=\const x\,P_m^{(\al,\be+1)}(2x^2-1)
\label{70}
\end{equation}
in the notation of \myciteKLS{146}{p.156}
(see also \cite{K27}), while \cite[Section 1.5.2]{K26}
has $C_n^{(\la,\mu)}(x)=\const\allowbreak\times S_n^{(\la-\half,\mu-\half)}(x)$.
For $\al,\be>-1$ we have the orthogonality relation
\begin{equation}
\int_{-1}^1 S_m^{(\al,\be)}(x)\,S_n^{(\al,\be)}(x)\,|x|^{2\be+1}(1-x^2)^\al\,dx
=0\qquad(m\ne n).
\label{71}
\end{equation}
For $\be=\al-1$ generalized Gegenbauer polynomials are limit cases of
continuous $q$-ultraspherical polynomials, see \eqref{176}.

If we define the {\em Dunkl operator} $T_\mu$ by
\begin{equation}
(T_\mu f)(x):=f'(x)+\mu\,\frac{f(x)-f(-x)}x
\label{72}
\end{equation}
and if we choose the constants in \eqref{70} as
\begin{equation}
S_{2m}^{(\al,\be)}(x)=\frac{(\al+\be+1)_m}{(\be+1)_m}\, P_m^{(\al,\be)}(2x^2-1),\quad
S_{2m+1}^{(\al,\be)}(x)=\frac{(\al+\be+1)_{m+1}}{(\be+1)_{m+1}}\,
x\,P_m^{(\al,\be+1)}(2x^2-1)
\label{73}
\end{equation}
then (see \cite[(1.6)]{K5})
\begin{equation}
T_{\be+\half}S_n^{(\al,\be)}=2(\al+\be+1)\,S_{n-1}^{(\al+1,\be)}.
\label{74}
\end{equation}
Formula \eqref{74} with \eqref{73} substituted gives rise to two
differentiation formulas involving Jacobi polynomials which are equivalent to
(9.8.7) and \eqref{69}.

Composition of \eqref{74} with itself gives
\[
T_{\be+\half}^2S_n^{(\al,\be)}=4(\al+\be+1)(\al+\be+2)\,S_{n-2}^{(\al+2,\be)},
\]
which is equivalent to the composition of (9.8.7) and \eqref{69}:
\begin{equation}
\left(\frac{d^2}{dx^2}+\frac{2\be+1}x\,\frac d{dx}\right)P_n^{(\al,\be)}(2x^2-1)
=4(n+\al+\be+1)(n+\be)\,P_{n-1}^{(\al+2,\be)}(2x^2-1).
\label{75}
\end{equation}
Formula \eqref{75} was also given in \myciteKLS{322}{(2.4)}.
%
% RS: add end
\subsection*{References}
\cite{Abram}, \cite{Ahmed+82}, \cite{Allaway89}, \cite{NAlSalam66}, \cite{AlSalam64}, 
\cite{AlSalamChihara72}, \cite{AndrewsAskey85}, \cite{AndrewsAskeyRoy}, \cite{Askey68}, 
\cite{Askey70}, \cite{Askey72}, \cite{Askey73}, \cite{Askey74}, \cite{Askey75}, \cite{Askey78}, 
\cite{Askey89I}, \cite{Askey2005}, \cite{AskeyFitch}, \cite{AskeyGasper71I}, \cite{AskeyGasper71II}, 
\cite{AskeyGasper76}, \cite{AskeyWainger}, \cite{AskeyWilson85}, \cite{Bailey38}, \cite{Brafman51}, 
\cite{Brown}, \cite{Carlitz61II}, \cite{Carlitz67}, \cite{ChenIsmail91}, \cite{Chihara78}, 
\cite{Chow+}, \cite{Ciesielski}, \cite{Cooper+}, \cite{DetteStudden92}, \cite{DetteStudden95},
\cite{DijksmaKoorn}, \cite{DimitrovRafaeli}, \cite{DimitrovRodrigues}, \cite{Doha2002I},
\cite{Doha2003I}, \cite{Doha2004II}, \cite{Dunkl84}, \cite{ElbertLaforgia87II},
\cite{ElbertLaforgiaRodono}, \cite{Erdelyi+}, \cite{Faldey}, \cite{FoataLeroux},
\cite{Gasper69}, \cite{Gasper70I}, \cite{Gasper70II}, \cite{Gasper71I}, \cite{Gasper71II},
\cite{Gasper72I}, \cite{Gasper72II}, \cite{Gasper73I}, \cite{Gasper73II}, \cite{Gasper74},
\cite{Gasper77}, \cite{Gatteschi87}, \cite{Gautschi2008}, \cite{Gautschi2009I},
\cite{Gautschi2009II}, \cite{GautschiLeopardi}, \cite{GawronskiShawyer}, \cite{Godoy+},
\cite{Grad}, \cite{Hajmirzaahmad94}, \cite{HartmannStephan}, \cite{Horton}, \cite{Ismail74},
\cite{Ismail77}, \cite{Ismail96}, \cite{Ismail2005II}, \cite{IsmailLi},
\cite{IsmailMassonRahman}, \cite{Kochneff97I}, \cite{Koekoek99}, \cite{Koekoek2000},
\cite{Koelink96II}, \cite{Koorn73}, \cite{Koorn74}, \cite{Koorn75}, \cite{Koorn77II},
\cite{Koorn78}, \cite{Koorn72}, \cite{Koorn85}, \cite{Koorn88}, \cite{KuijlaarsMartinez},
\cite{Kuijlaars+}, \cite{LabelleYehI}, \cite{LabelleYehII}, \cite{Laine}, \cite{Lesky95II},
\cite{Lesky96}, \cite{Li96}, \cite{Li97}, \cite{LopezTemme2004}, \cite{Luke}, \cite{Martinez},
\cite{Mathai}, \cite{Meijer}, \cite{Miller89}, \cite{MoakSaffVarga}, \cite{Nikiforov+},
\cite{NikiforovUvarov}, \cite{Olver}, \cite{Prasad}, \cite{Rahman76I}, \cite{Rahman76II},
\cite{Rahman77}, \cite{Rahman81I}, \cite{RahmanShah}, \cite{Rainville}, \cite{Rusev},
\cite{Shi}, \cite{Srivastava69II}, \cite{Srivastava71}, \cite{Srivastava82},
\cite{SrivastavaSinghal}, \cite{Stanton80I}, \cite{Stanton90}, \cite{Szego75}, \cite{Temme},
\cite{Vertesi}, \cite{Wimp87}, \cite{WongZhang94}, \cite{WongZhang2006}, \cite{Zarzo+},
\cite{Zayed}.

\section*{Special cases}

\subsection{Gegenbauer / Ultraspherical}\index{Gegenbauer polynomials}
\index{Ultraspherical polynomials}

\par

\subsection*{Hypergeometric representation}
The Gegenbauer (or ultraspherical) polynomials are Jacobi polynomials with
$\alpha=\beta=\lambda-\frac{1}{2}$ and another normalization:
\begin{eqnarray}
\label{DefGegenbauer}
C_n^{(\lambda)}(x)&=&\frac{(2\lambda)_n}{(\lambda+\frac{1}{2})_n}
P_n^{(\lambda-\frac{1}{2},\lambda-\frac{1}{2})}(x)\nonumber\\
&=&\frac{(2\lambda)_n}{n!}\,\hyp{2}{1}{-n,n+2\lambda}
{\lambda+\frac{1}{2}}{\frac{1-x}{2}},\quad\lambda\neq 0.
\end{eqnarray}

\subsection*{Orthogonality relation}
\begin{eqnarray}
\label{OrtGegenbauer}
& &\int_{-1}^1(1-x^2)^{\lambda-\frac{1}{2}}C_m^{(\lambda)}(x)C_n^{(\lambda)}(x)\,dx\nonumber\\
& &{}=\frac{\pi\Gamma(n+2\lambda)2^{1-2\lambda}}{\left\{\Gamma(\lambda)\right\}^2(n+\lambda)n!}\,\delta_{mn},
\quad\lambda>-\frac{1}{2}\quad\lambda\neq 0.
\end{eqnarray}

\subsection*{Recurrence relation}
\begin{equation}
\label{RecGegenbauer}
2(n+\lambda)xC_n^{(\lambda)}(x)=(n+1)C_{n+1}^{(\lambda)}(x)+(n+2\lambda-1)C_{n-1}^{(\lambda)}(x).
\end{equation}

\subsection*{Normalized recurrence relation}
\begin{equation}
\label{NormRecGegenbauer}
xp_n(x)=p_{n+1}(x)+\frac{n(n+2\lambda-1)}{4(n+\lambda-1)(n+\lambda)}p_{n-1}(x),
\end{equation}
where
$$C_n^{(\lambda)}(x)=\frac{2^n(\lambda)_n}{n!}p_n(x).$$

\subsection*{Differential equation}
\begin{equation}
\label{dvGegenbauer}
(1-x^2)y''(x)-(2\lambda+1)xy'(x)+n(n+2\lambda)y(x)=0,\quad y(x)=C_n^{(\lambda)}(x).
\end{equation}

\subsection*{Forward shift operator}
\begin{equation}
\label{shift1Gegenbauer}
\frac{d}{dx}C_n^{(\lambda)}(x)=2\lambda C_{n-1}^{(\lambda+1)}(x).
\end{equation}

\subsection*{Backward shift operator}
\begin{equation}
\label{shift2GegenbauerI}
(1-x^2)\frac{d}{dx}C_n^{(\lambda)}(x)+(1-2\lambda)xC_n^{(\lambda)}(x)=
-\frac{(n+1)(2\lambda+n-1)}{2(\lambda-1)} C_{n+1}^{(\lambda-1)}(x)
\end{equation}
or equivalently
\begin{eqnarray}
\label{shift2GegenbauerII}
& &\frac{d}{dx}\left[(1-x^2)^{\lambda-\textstyle\frac{1}{2}}C_n^{(\lambda)}(x)\right]\nonumber\\
& &{}=-\frac{(n+1)(2\lambda+n-1)}{2(\lambda-1)}(1-x^2)^{\lambda-\textstyle\frac{3}{2}}C_{n+1}^{(\lambda-1)}(x).
\end{eqnarray}

\subsection*{Rodrigues-type formula}
\begin{equation}
\label{RodGegenbauer}
(1-x^2)^{\lambda-\frac{1}{2}}C_n^{(\lambda)}(x)=
\frac{(2\lambda)_n(-1)^n}{(\lambda+\frac{1}{2})_n2^nn!}\left(\frac{d}{dx}\right)^n
\left[(1-x^2)^{\lambda+n-\frac{1}{2}}\right].
\end{equation}

\subsection*{Generating functions}
\begin{equation}
\label{GenGegenbauer1}
(1-2xt+t^2)^{-\lambda}=\sum_{n=0}^{\infty}C_n^{(\lambda)}(x)t^n.
\end{equation}

\begin{equation}
\label{GenGegenbauer2}
R^{-1}\left(\frac{1+R-xt}{2}\right)^{\frac{1}{2}-\lambda}=\sum_{n=0}^{\infty}
\frac{(\lambda+\frac{1}{2})_n}{(2\lambda)_n}C_n^{(\lambda)}(x)t^n,
\quad R=\sqrt{1-2xt+t^2}.
\end{equation}

\begin{equation}
\label{GenGegenbauer3}
\hyp{0}{1}{-}{\lambda+\frac{1}{2}}{\frac{(x-1)t}{2}}\,
\hyp{0}{1}{-}{\lambda+\frac{1}{2}}{\frac{(x+1)t}{2}}
=\sum_{n=0}^{\infty}\frac{C_n^{(\lambda)}(x)}
{(2\lambda)_n(\lambda+\frac{1}{2})_n}t^n.
\end{equation}

\begin{equation}
\label{GenGegenbauer4}
\e^{xt}\,\hyp{0}{1}{-}{\lambda+\frac{1}{2}}{\frac{(x^2-1)t^2}{4}}=
\sum_{n=0}^{\infty}\frac{C_n^{(\lambda)}(x)}{(2\lambda)_n}t^n.
\end{equation}

\begin{eqnarray}
\label{GenGegenbauer5}
& &\hyp{2}{1}{\gamma,2\lambda-\gamma}{\lambda+\frac{1}{2}}{\frac{1-R-t}{2}}\,
\hyp{2}{1}{\gamma,2\lambda-\gamma}{\lambda+\frac{1}{2}}{\frac{1-R+t}{2}}\nonumber\\
& &{}=\sum_{n=0}^{\infty}\frac{(\gamma)_n(2\lambda-\gamma)_n}
{(2\lambda)_n(\lambda+\frac{1}{2})_n}C_n^{(\lambda)}(x)t^n,
\quad R=\sqrt{1-2xt+t^2},\quad\textrm{$\gamma$ arbitrary}.
\end{eqnarray}

\begin{eqnarray}
\label{GenGegenbauer6}
& &(1-xt)^{-\gamma}\,\hyp{2}{1}{\frac{1}{2}\gamma,\frac{1}{2}\gamma+\frac{1}{2}}
{\lambda+\frac{1}{2}}{\frac{(x^2-1)t^2}{(1-xt)^2}}\nonumber\\
& &{}=\sum_{n=0}^{\infty}\frac{(\gamma)_n}{(2\lambda)_n}C_n^{(\lambda)}(x)t^n,
\quad\textrm{$\gamma$ arbitrary}.
\end{eqnarray}

\subsection*{Limit relation}

\subsubsection*{Gegenbauer / Ultraspherical $\rightarrow$ Hermite}
The Hermite polynomials given by (\ref{DefHermite}) follow from the Gegenbauer (or
ultraspherical) polynomials by taking $\lambda=\alpha+\frac{1}{2}$ and letting
$\alpha\rightarrow\infty$ in the following way:
\begin{equation}
\lim_{\alpha\rightarrow\infty}
\alpha^{-\frac{1}{2}n}C_n^{(\alpha+\frac{1}{2})}(\alpha^{-\frac{1}{2}}x)=\frac{H_n(x)}{n!}.
\end{equation}

\subsection*{Remarks}
The case $\lambda=0$ needs another normalization. In that case we have the
Chebyshev polynomials of the first kind described in the next subsection.

\noindent
The Gegenbauer (or ultraspherical) polynomials are related to the Jacobi polynomials
given by (\ref{DefJacobi}) by the quadratic transformations:
$$C_{2n}^{(\lambda)}(x)=\frac{(\lambda)_n}{(\frac{1}{2})_n}
P_n^{(\lambda-\frac{1}{2},-\frac{1}{2})}(2x^2-1)$$
and
$$C_{2n+1}^{(\lambda)}(x)=\frac{(\lambda)_{n+1}}{(\frac{1}{2})_{n+1}}
xP_n^{(\lambda-\frac{1}{2},\frac{1}{2})}(2x^2-1).$$
% RS: add begin\label{sec9.8.1}
%
\paragraph{\large\bf KLSadd: Notation}Here the Gegenbauer polynomial is denoted by $C_n^\la$ instead of $C_n^{(\la)}$.
%
\paragraph{\large\bf KLSadd: Orthogonality relation}Write the \RHS\ of (9.8.20) as $h_n\,\de_{m,n}$. Then
\begin{equation}
\frac{h_n}{h_0}=
\frac\la{\la+n}\,\frac{(2\la)_n}{n!}\,,\quad
h_0=\frac{\pi^\half\,\Ga(\la+\thalf)}{\Ga(\la+1)},\quad
\frac{h_n}{h_0\,(C_n^\la(1))^2}=
\frac\la{\la+n}\,\frac{n!}{(2\la)_n}\,.
\label{61}
\end{equation}
%
\paragraph{\large\bf KLSadd: Hypergeometric representation}Beside (9.8.19) we have also
\begin{equation}
C_n^\lambda(x)=\sum_{\ell=0}^{\lfloor n/2\rfloor}\frac{(-1)^{\ell}(\lambda)_{n-\ell}}
{\ell!\;(n-2\ell)!}\,(2x)^{n-2\ell}
=(2x)^{n}\,\frac{(\lambda)_{n}}{n!}\,
\hyp21{-\thalf n,-\thalf n+\thalf}{1-\la-n}{\frac1{x^2}}.
\label{57}
\end{equation}
See \mycite{DLMF}{(18.5.10)}.
%
\paragraph{\large\bf KLSadd: Special value}\begin{equation}
C_n^{\la}(1)=\frac{(2\la)_n}{n!}\,.
\label{49}
\end{equation}
Use (9.8.19) or see \mycite{DLMF}{Table 18.6.1}.
%
\paragraph{\large\bf KLSadd: Expression in terms of Jacobi}%
\begin{equation}
\frac{C_n^\la(x)}{C_n^\la(1)}=
\frac{P_n^{(\la-\half,\la-\half)}(x)}{P_n^{(\la-\half,\la-\half)}(1)}\,,\qquad
C_n^\la(x)=\frac{(2\la)_n}{(\la+\thalf)_n}\,P_n^{(\la-\half,\la-\half)}(x).
\label{65}
\end{equation}
%
\paragraph{\large\bf KLSadd: Re: (9.8.21)}By iteration of recurrence relation (9.8.21):
\begin{multline}
x^2 C_n^\la(x)=
\frac{(n+1)(n+2)}{4(n+\la)(n+\la+1)}\,C_{n+2}^\la(x)+
\frac{n^2+2n\la+\la-1}{2(n+\la-1)(n+\la+1)}\,C_n^\la(x)\\
+\frac{(n+2\la-1)(n+2\la-2)}{4(n+\la)(n+\la-1)}\,C_{n-2}^\la(x).
\label{6}
\end{multline}
%
\paragraph{\large\bf KLSadd: Bilateral generating functions}\begin{multline}
\sum_{n=0}^\iy\frac{n!}{(2\la)_n}\,r^n\,C_n^\la(x)\,C_n^\la(y)
=\frac1{(1-2rxy+r^2)^\la}\,\hyp21{\thalf\la,\thalf(\la+1)}{\la+\thalf}
{\frac{4r^2(1-x^2)(1-y^2)}{(1-2rxy+r^2)^2}}\\
(r\in(-1,1),\;x,y\in[-1,1]).
\label{66}
\end{multline}
For the proof put $\be:=\al$ in \eqref{58}, then use \eqref{63} and \eqref{65}.
The Poisson kernel for Gegenbauer polynomials can be derived in a similar way
from \eqref{59}, or alternatively by applying the operator
$r^{-\la+1}\frac d{dr}\circ r^\la$ to both sides of \eqref{66}:
\begin{multline}
\sum_{n=0}^\iy\frac{\la+n}\la\,\frac{n!}{(2\la)_n}\,r^n\,C_n^\la(x)\,C_n^\la(y)
=\frac{1-r^2}{(1-2rxy+r^2)^{\la+1}}\\
\times\hyp21{\thalf(\la+1),\thalf(\la+2)}{\la+\thalf}
{\frac{4r^2(1-x^2)(1-y^2)}{(1-2rxy+r^2)^2}}\qquad
(r\in(-1,1),\;x,y\in[-1,1]).
\label{67}
\end{multline}
Formula \eqref{67} was obtained by Gasper \& Rahman \myciteKLS{234}{(4.4)}
as a limit case of their formula for the Poisson kernel for continuous
$q$-ultraspherical polynomials.
%
\paragraph{\large\bf KLSadd: Trigonometric expansions}By \mycite{DLMF}{(18.5.11), (15.8.1)}:
\begin{align}
C_n^{\la}(\cos\tha)
&=\sum_{k=0}^n\frac{(\la)_k(\la)_{n-k}}{k!\,(n-k)!}\,e^{i(n-2k)\tha}
=e^{in\tha}\frac{(\la)_n}{n!}\,
\hyp21{-n,\la}{1-\la-n}{e^{-2i\tha}}\label{103}\\
&=\frac{(\la)_n}{2^\la n!}\,
e^{-\half i\la\pi}e^{i(n+\la)\tha}\,(\sin\tha)^{-\la}\,
\hyp21{\la,1-\la}{1-\la-n}{\frac{i e^{-i\tha}}{2\sin\tha}}\label{104}\\
&=\frac{(\la)_n}{n!}\,\sum_{k=0}^\iy\frac{(\la)_k(1-\la)_k}{(1-\la-n)_k k!}\,
\frac{\cos((n-k+\la)\tha+\thalf(k-\la)\pi)}{(2\sin\tha)^{k+\la}}\,.\label{105}
\end{align}
In \eqref{104} and \eqref{105} we require that
$\tfrac16\pi<\tha<\tfrac56\pi$. Then the convergence is absolute for $\la>\thalf$
and conditional for $0<\la\le\thalf$.

By \mycite{DLMF}{(14.13.1), (14.3.21), (15.8.1)]}:
\begin{align}
C_n^\la(\cos\tha)&=\frac{2\Ga(\la+\thalf)}{\pi^\half\Ga(\la+1)}\,
\frac{(2\la)_n}{(\la+1)_n}\,(\sin\tha)^{1-2\la}\,
\sum_{k=0}^\iy\frac{(1-\la)_k(n+1)_k}{(n+\la+1)_k k!}\,
\sin\big((2k+n+1)\tha\big)
\label{7}\\
&=\frac{2\Ga(\la+\thalf)}{\pi^\half\Ga(\la+1)}\,
\frac{(2\la)_n}{(\la+1)_n}\,(\sin\tha)^{1-2\la}\,
\Im\!\!\left(e^{i(n+1)\tha}\,\hyp21{1-\la,n+1}{n+\la+1}{e^{2i\tha}}\right)\nonumber\\
&=\frac{2^\la\Ga(\la+\thalf)}{\pi^\half\Ga(\la+1)}\,
\frac{(2\la)_n}{(\la+1)_n}\,(\sin\tha)^{-\la}\,
\Re\!\!\left(e^{-\thalf i\la\pi}e^{i(n+\la)\tha}\,
\hyp21{\la,1-\la}{1+\la+n}{\frac{e^{i\tha}}{2i\sin\tha}}\right)\nonumber\\
&=\frac{2^{2\la}\Ga(\la+\thalf)}{\pi^\half\Ga(\la+1)}\,\frac{(2\la)_n}{(\la+1)_n}\,
\sum_{k=0}^\iy\frac{(\la)_k(1-\la)_k}{(1+\la+n)_k k!}\,
\frac{\cos((n+k+\la)\tha-\thalf(k+\la)\pi)}{(2\sin\tha)^{k+\la}}\,.
\label{106}
\end{align}
We require that $0<\tha<\pi$ in \eqref{7} and $\tfrac16\pi<\tha<\tfrac56\pi$ in
\eqref{106} The convergence is absolute for $\la>\thalf$ and conditional for
$0<\la\le\thalf$.
For $\la\in\Zpos$ the above series terminate after the term with
$k=\la-1$.
Formulas \eqref{7} and \eqref{106} are also given in
\mycite{Sz}{(4.9.22), (4.9.25)}.
%
\paragraph{\large\bf KLSadd: Fourier transform}\begin{equation}
\frac{\Ga(\la+1)}{\Ga(\la+\thalf)\,\Ga(\thalf)}\,
\int_{-1}^1 \frac{C_n^\la(y)}{C_n^\la(1)}\,(1-y^2)^{\la-\half}\,
e^{ixy}\,dy
=i^n\,2^\la\,\Ga(\la+1)\,x^{-\la}\,J_{\la+n}(x).
\label{8}
\end{equation}
See \mycite{DLMF}{(18.17.17) and (18.17.18)}.
%
\paragraph{\large\bf KLSadd: Laplace transforms}\begin{equation}
\frac2{n!\,\Ga(\la)}\,
\int_0^\iy H_n(tx)\,t^{n+2\la-1}\,e^{-t^2}\,dt=C_n^\la(x).
\label{56}
\end{equation}
See Nielsen \cite[p.48, (4) with p.47, (1) and p.28, (10)]{K4} (1918)
or Feldheim \cite[(28)]{K3} (1942).
\begin{equation}
\frac2{\Ga(\la+\thalf)}\,\int_0^1 \frac{C_n^\la(t)}{C_n^\la(1)}\,
(1-t^2)^{\la-\half}\,t^{-1}\,(x/t)^{n+2\la+1}\,e^{-x^2/t^2}\,dt
=2^{-n}\,H_n(x)\,e^{-x^2}\quad(\la>-\thalf).
\label{46}
\end{equation}
Use Askey \& Fitch \cite[(3.29)]{K2} for $\al=\pm\thalf$ together with
\eqref{48}, \eqref{51}, \eqref{52}, \eqref{54} and \eqref{55}.
\paragraph{\large\bf KLSadd: Addition formula}\begin{multline}
R_n^{(\al,\al)}\big(xy+(1-x^2)^\half(1-y^2)^\half t\big)
=\sum_{k=0}^n \frac{(-1)^k(-n)_k\,(n+2\al+1)_k}{2^{2k}((\al+1)_k)^2}\\
\times(1-x^2)^{k/2} R_{n-k}^{(\al+k,\al+k)}(x)\,(1-y^2)^{k/2} R_{n-k}^{(\al+k,\al+k)}(y)\,
\om_k^{(\al-\half,\al-\half)}\,R_k^{(\al-\half,\al-\half)}(t),
\label{108}
\end{multline}
where
\[
R_n^{(\al,\be)}(x):=P_n^{(\al,\be)}(x)/P_n^{(\al,\be)}(1),\quad
\om_n^{(\al,\be)}:=\frac{\int_{-1}^1 (1-x)^\al(1+x)^\be\,dx}
{\int_{-1}^1 (R_n^{(\al,\be)}(x))^2\,(1-x)^\al(1+x)^\be\,dx}\,.
\]
%
% RS: add end
\subsection*{References}
\cite{Abram}, \cite{Ahmed+86}, \cite{AndrewsAskeyRoy}, \cite{Area+I}, \cite{Askey67}, \cite{Askey74}, \cite{Askey75},
\cite{Askey89I}, \cite{AskeyFitch}, \cite{AskeyKoornRahman}, \cite{Berg}, \cite{BilodeauI},
\cite{BojanovNikolov}, \cite{Brafman51}, \cite{Brafman57I}, \cite{Brown},
\cite{BustozIsmail82}, \cite{BustozIsmail83}, \cite{BustozIsmail89}, \cite{BustozSavage79},
\cite{BustozSavage80}, \cite{Carlitz61II}, \cite{Chihara78}, \cite{Common}, \cite{Danese},
\cite{Dette94}, \cite{DijksmaKoorn}, \cite{DilcherStolarsky}, \cite{Dimitrov96},
\cite{Dimitrov2003}, \cite{Doha2002II}, \cite{Driver}, \cite{ElbertLaforgia86I},
\cite{ElbertLaforgia86II}, \cite{ElbertLaforgia90}, \cite{Erdelyi+}, \cite{Exton96},
\cite{Gasper69}, \cite{Gasper72II}, \cite{Gasper85}, \cite{Grad}, \cite{Ismail74},
\cite{Ismail2005II}, \cite{Koekoek2000}, \cite{Koorn88}, \cite{Laforgia}, \cite{LewanowiczI},
\cite{Lorch}, \cite{Mathai}, \cite{Nagel}, \cite{Nikiforov+}, \cite{NikiforovUvarov},
\cite{RahmanShah}, \cite{Rainville}, \cite{Reimer}, \cite{Sartoretto}, \cite{Srivastava71},
\cite{Szego75}, \cite{Temme}, \cite{Viswanathan}, \cite{Zayed}.

\subsection{Chebyshev}\index{Chebyshev polynomials}

\par

\subsection*{Hypergeometric representation}
The Chebyshev polynomials of the first kind can be obtained from the Jacobi
polynomials by taking $\alpha=\beta=-\frac{1}{2}$:
\begin{equation}
\label{DefChebyshevI}
T_n(x)=\frac{P_n^{(-\frac{1}{2},-\frac{1}{2})}(x)}{P_n^{(-\frac{1}{2},-\frac{1}{2})}(1)}
=\hyp{2}{1}{-n,n}{\frac{1}{2}}{\frac{1-x}{2}}
\end{equation}
and the Chebyshev polynomials of the second kind can be obtained from the
Jacobi polynomials by taking $\alpha=\beta=\frac{1}{2}$:
\begin{equation}
\label{DefChebyshevII}
U_n(x)=(n+1)\frac{P_n^{(\frac{1}{2},\frac{1}{2})}(x)}{P_n^{(\frac{1}{2},\frac{1}{2})}(1)}
=(n+1)\,\hyp{2}{1}{-n,n+2}{\frac{3}{2}}{\frac{1-x}{2}}.
\end{equation}

\subsection*{Orthogonality relation}
\begin{equation}
\label{OrtChebyshevI}
\int_{-1}^1(1-x^2)^{-\frac{1}{2}}T_m(x)T_n(x)\,dx=
\left\{\begin{array}{ll}
\displaystyle\frac{\cpi}{2}\,\delta_{mn}, & n\neq 0\\[5mm]
\cpi\,\delta_{mn}, & n=0.
\end{array}\right.
\end{equation}

\begin{equation}
\label{OrtChebyshevII}
\int_{-1}^1(1-x^2)^{\frac{1}{2}}U_m(x)U_n(x)\,dx=\frac{\cpi}{2}\,\delta_{mn}.
\end{equation}

\subsection*{Recurrence relations}
\begin{equation}
\label{RecChebyshevI}
2xT_n(x)=T_{n+1}(x)+T_{n-1}(x),\quad T_{0}(x)=1\quad\textrm{and}\quad T_1(x)=x.
\end{equation}

\begin{equation}
\label{RecChebyshevII}
2xU_n(x)=U_{n+1}(x)+U_{n-1}(x),\quad U_{0}(x)=1\quad\textrm{and}\quad U_1(x)=2x.
\end{equation}

\subsection*{Normalized recurrence relations}
\begin{equation}
\label{NormRecChebyshevI}
xp_n(x)=p_{n+1}(x)+\frac{1}{4}p_{n-1}(x),
\end{equation}
where
$$T_1(x)=p_1(x)=x\quad\textrm{and}\quad T_n(x)=2^np_n(x),\quad n\neq 1.$$

\begin{equation}
\label{NormRecChebyshevII}
xp_n(x)=p_{n+1}(x)+\frac{1}{4}p_{n-1}(x),
\end{equation}
where
$$U_n(x)=2^np_n(x).$$

\subsection*{Differential equations}
\begin{equation}
\label{dvChebyshevI}
(1-x^2)y''(x)-xy'(x)+n^2y(x)=0,\quad y(x)=T_n(x).
\end{equation}

\begin{equation}
\label{dvChebyshevII}
(1-x^2)y''(x)-3xy'(x)+n(n+2)y(x)=0,\quad y(x)=U_n(x).
\end{equation}

\subsection*{Forward shift operator}
\begin{equation}
\label{shift1Chebyshev}
\frac{d}{dx}T_n(x)=nU_{n-1}(x).
\end{equation}

\subsection*{Backward shift operator}
\begin{equation}
\label{shift2ChebyshevI}
(1-x^2)\frac{d}{dx}U_n(x)-xU_n(x)=-(n+1)T_{n+1}(x)
\end{equation}
or equivalently
\begin{equation}
\label{shift2ChebyshevII}
\frac{d}{dx}\left[\left(1-x^2\right)^{\frac{1}{2}}U_n(x)\right]
=-(n+1)\left(1-x^2\right)^{-\frac{1}{2}}T_{n+1}(x).
\end{equation}

\subsection*{Rodrigues-type formulas}
\begin{equation}
\label{RodChebyshevI}
(1-x^2)^{-\frac{1}{2}}T_n(x)=\frac{(-1)^n}{(\frac{1}{2})_n2^n}
\left(\frac{d}{dx}\right)^n\left[(1-x^2)^{n-\frac{1}{2}}\right].
\end{equation}

\begin{equation}
\label{RodChebyshevII}
(1-x^2)^{\frac{1}{2}}U_n(x)=\frac{(n+1)(-1)^n}{(\frac{3}{2})_n2^n}
\left(\frac{d}{dx}\right)^n\left[(1-x^2)^{n+\frac{1}{2}}\right].
\end{equation}

\subsection*{Generating functions}
\begin{equation}
\label{GenChebyshevI1}
\frac{1-xt}{1-2xt+t^2}=\sum_{n=0}^{\infty}T_n(x)t^n.
\end{equation}

\begin{equation}
\label{GenChebyshevI2}
R^{-1}\sqrt{\frac{1}{2}(1+R-xt)}=\sum_{n=0}^{\infty}
\frac{\left(\frac{1}{2}\right)_n}{n!}T_n(x)t^n,\quad R=\sqrt{1-2xt+t^2}.
\end{equation}

\begin{equation}
\label{GenChebyshevI3}
\hyp{0}{1}{-}{\frac{1}{2}}{\frac{(x-1)t}{2}}\,
\hyp{0}{1}{-}{\frac{1}{2}}{\frac{(x+1)t}{2}}=
\sum_{n=0}^{\infty}\frac{T_n(x)}{\left(\frac{1}{2}\right)_nn!}t^n.
\end{equation}

\begin{equation}
\label{GenChebyshevI4}
\e^{xt}\,\hyp{0}{1}{-}{\frac{1}{2}}{\frac{(x^2-1)t^2}{4}}=
\sum_{n=0}^{\infty}\frac{T_n(x)}{n!}t^n.
\end{equation}

\begin{eqnarray}
\label{GenChebyshevI5}
& &\hyp{2}{1}{\gamma,-\gamma}{\frac{1}{2}}{\frac{1-R-t}{2}}\,
\hyp{2}{1}{\gamma,-\gamma}{\frac{1}{2}}{\frac{1-R+t}{2}}\nonumber\\
& &{}=\sum_{n=0}^{\infty}\frac{(\gamma)_n(-\gamma)_n}{\left(\frac{1}{2}\right)_nn!}T_n(x)t^n,
\quad R=\sqrt{1-2xt+t^2},\quad\textrm{$\gamma$ arbitrary}.
\end{eqnarray}

\begin{eqnarray}
\label{GenChebyshevI6}
& &(1-xt)^{-\gamma}\,\hyp{2}{1}{\frac{1}{2}\gamma,\frac{1}{2}\gamma+\frac{1}{2}}{\frac{1}{2}}
{\frac{(x^2-1)t^2}{(1-xt)^2}}\nonumber\\
& &=\sum_{n=0}^{\infty}\frac{(\gamma)_n}{n!}T_n(x)t^n,\quad\textrm{$\gamma$ arbitrary}.
\end{eqnarray}

\begin{equation}
\label{GenChebyshevII1}
\frac{1}{1-2xt+t^2}=\sum_{n=0}^{\infty}U_n(x)t^n.
\end{equation}

\begin{equation}
\label{GenChebyshevII2}
\frac{1}{R\sqrt{\frac{1}{2}(1+R-xt)}}=\sum_{n=0}^{\infty}
\frac{\left(\frac{3}{2}\right)_n}{(n+1)!}U_n(x)t^n,\quad R=\sqrt{1-2xt+t^2}.
\end{equation}

\begin{equation}
\label{GenChebyshevII3}
\hyp{0}{1}{-}{\frac{3}{2}}{\frac{(x-1)t}{2}}\,
\hyp{0}{1}{-}{\frac{3}{2}}{\frac{(x+1)t}{2}}=
\sum_{n=0}^{\infty}\frac{U_n(x)}{\left(\frac{3}{2}\right)_n(n+1)!}t^n.
\end{equation}

\begin{equation}
\label{GenChebyshevII4}
\e^{xt}\,\hyp{0}{1}{-}{\frac{3}{2}}{\frac{(x^2-1)t^2}{4}}=
\sum_{n=0}^{\infty}\frac{U_n(x)}{(n+1)!}t^n.
\end{equation}

\begin{eqnarray}
\label{GenChebyshevII5}
& &\hyp{2}{1}{\gamma,2-\gamma}{\frac{3}{2}}{\frac{1-R-t}{2}}\,
\hyp{2}{1}{\gamma,2-\gamma}{\frac{3}{2}}{\frac{1-R+t}{2}}\nonumber\\
& &{}=\sum_{n=0}^{\infty}\frac{(\gamma)_n(2-\gamma)_n}{\left(\frac{3}{2}\right)_n(n+1)!}U_n(x)t^n,
\quad R=\sqrt{1-2xt+t^2},\quad\textrm{$\gamma$ arbitrary}.
\end{eqnarray}

\begin{eqnarray}
\label{GenChebyshevII6}
& &(1-xt)^{-\gamma}\,\hyp{2}{1}{\frac{1}{2}\gamma,\frac{1}{2}\gamma+\frac{1}{2}}{\frac{3}{2}}
{\frac{(x^2-1)t^2}{(1-xt)^2}}\nonumber\\
& &{}=\sum_{n=0}^{\infty}\frac{(\gamma)_n}{(n+1)!}U_n(x)t^n,\quad\textrm{$\gamma$ arbitrary}.
\end{eqnarray}

\subsection*{Remarks}
The Chebyshev polynomials can also be written as:
$$T_n(x)=\cos(n\theta),\quad x=\cos\theta$$
and
$$U_n(x)=\frac{\sin (n+1)\theta}{\sin\theta},\quad x=\cos\theta.$$
Further we have
$$U_n(x)=C_n^{(1)}(x)$$
where $C_n^{(\lambda)}(x)$ denotes the Gegenbauer (or ultraspherical)
polynomial given by (\ref{DefGegenbauer}) in the preceding subsection.
% RS: add begin\label{sec9.8.2}
In addition to the Chebyshev polynomials $T_n$ of the first kind (9.8.35)
and $U_n$ of the second kind (9.8.36),
\begin{align}
T_n(x)&:=\frac{P_n^{(-\half,-\half)}(x)}{P_n^{(-\half,-\half)}(1)}
=\cos(n\tha),\quad x=\cos\tha,\\
U_n(x)&:=(n+1)\,\frac{P_n^{(\half,\half)}(x)}{P_n^{(\half,\half)}(1)}
=\frac{\sin((n+1)\tha)}{\sin\tha}\,,\quad x=\cos\tha,
\end{align}
we have Chebyshev polynomials $V_n$ {\em of the third kind}
and $W_n$ {\em of the fourth kind},
\begin{align}
V_n(x)&:=\frac{P_n^{(-\half,\half)}(x)}{P_n^{(-\half,\half)}(1)}
=\frac{\cos((n+\thalf)\tha)}{\cos(\thalf\tha)}\,,\quad x=\cos\tha,\\
W_n(x)&:=(2n+1)\,\frac{P_n^{(\half,-\half)}(x)}{P_n^{(\half,-\half)}(1)}
=\frac{\sin((n+\thalf)\tha)}{\sin(\thalf\tha)}\,,\quad x=\cos\tha,
\end{align}
see \cite[Section 1.2.3]{K20}. Then there is the symmetry
\begin{equation}
V_n(-x)=(-1)^n W_n(x).
\label{140}
\end{equation}

The names of Chebyshev polynomials of the third and fourth kind
and the notation $V_n(x)$ are due to Gautschi \cite{K21}.
The notation $W_n(x)$ was first used by Mason \cite{K22}.
Names and notations for Chebyshev polynomials of the third and fourth
kind are interchanged in \mycite{AAR}{Remark 2.5.3} and
\mycite{DLMF}{Table 18.3.1}.
%
% RS: add end
\subsection*{References}
\cite{Abram}, \cite{AskeyFitch}, \cite{AskeyGasperHarris}, \cite{AskeyIsmail76},
\cite{Bavinck95}, \cite{Chihara78}, \cite{Danese}, \cite{DilcherStolarsky},
\cite{Erdelyi+}, \cite{Grad}, \cite{HartmannStephan}, \cite{Ismail2005II}, \cite{Koekoek2000},
\cite{Luke}, \cite{Mathai}, \cite{Nikiforov+}, \cite{NikiforovUvarov}, \cite{Rainville},
\cite{Rayes+}, \cite{Rivlin}, \cite{SainteViennot}, \cite{Szego75}, \cite{Temme},
\cite{Wilson70I}, \cite{Zayed}, \cite{Zhang}, \cite{ZhangWang}.

\subsection{Legendre / Spherical}\index{Legendre polynomials}
\index{Spherical polynomials}

\par

\subsection*{Hypergeometric representation}
The Legendre (or spherical) polynomials are Jacobi polynomials with $\alpha=\beta=0$:
\begin{equation}
\label{DefLegendre}
P_n(x)=P_n^{(0,0)}(x)=\hyp{2}{1}{-n,n+1}{1}{\frac{1-x}{2}}.
\end{equation}

\subsection*{Orthogonality relation}
\begin{equation}
\label{OrtLegendre}
\int_{-1}^1P_m(x)P_n(x)\,dx=\frac{2}{2n+1}\,\delta_{mn}.
\end{equation}

\subsection*{Recurrence relation}
\begin{equation}
\label{RecLegendre}
(2n+1)xP_n(x)=(n+1)P_{n+1}(x)+nP_{n-1}(x).
\end{equation}

\subsection*{Normalized recurrence relation}
\begin{equation}
\label{NormRecLegendre}
xp_n(x)=p_{n+1}(x)+\frac{n^2}{(2n-1)(2n+1)}p_{n-1}(x),
\end{equation}
where
$$P_n(x)=\binom{2n}{n}\frac{1}{2^n}p_n(x).$$

\subsection*{Differential equation}
\begin{equation}
\label{dvLegendre}
(1-x^2)y''(x)-2xy'(x)+n(n+1)y(x)=0,\quad y(x)=P_n(x).
\end{equation}

\subsection*{Rodrigues-type formula}
\begin{equation}
\label{RodLegendre}
P_n(x)=\frac{(-1)^n}{2^nn!}\left(\frac{d}{dx}\right)^n\left[(1-x^2)^n\right].
\end{equation}

\subsection*{Generating functions}
\begin{equation}
\label{GenLegendre1}
\frac{1}{\sqrt{1-2xt+t^2}}=\sum_{n=0}^{\infty}P_n(x)t^n.
\end{equation}

\begin{equation}
\label{GenLegendre2}
\hyp{0}{1}{-}{1}{\frac{(x-1)t}{2}}\,\hyp{0}{1}{-}{1}{\frac{(x+1)t}{2}}=
\sum_{n=0}^{\infty}\frac{P_n(x)}{(n!)^2}t^n.
\end{equation}

\begin{equation}
\label{GenLegendre3}
\e^{xt}\,\hyp{0}{1}{-}{1}{\frac{(x^2-1)t^2}{4}}=
\sum_{n=0}^{\infty}\frac{P_n(x)}{n!}t^n.
\end{equation}

\begin{eqnarray}
\label{GenLegendre4}
& &\hyp{2}{1}{\gamma,1-\gamma}{1}{\frac{1-R-t}{2}}\,
\hyp{2}{1}{\gamma,1-\gamma}{1}{\frac{1-R+t}{2}}\nonumber\\
& &{}=\sum_{n=0}^{\infty}\frac{(\gamma)_n(1-\gamma)_n}{(n!)^2}P_n(x)t^n,
\quad R=\sqrt{1-2xt+t^2},\quad\textrm{$\gamma$ arbitrary}.
\end{eqnarray}

\begin{eqnarray}
\label{GenLegendre5}
& &(1-xt)^{-\gamma}\,\hyp{2}{1}{\frac{1}{2}\gamma,\frac{1}{2}\gamma+\frac{1}{2}}{1}
{\frac{(x^2-1)t^2}{(1-xt)^2}}\nonumber\\
& &{}=\sum_{n=0}^{\infty}\frac{(\gamma)_n}{n!}P_n(x)t^n,\quad\textrm{$\gamma$ arbitrary}.
\end{eqnarray}

\subsection*{References}
\cite{Abram}, \cite{Alladi}, \cite{AlSalam90}, \cite{Bhonsle}, \cite{Brafman51},
\cite{Carlitz57II}, \cite{Chihara78}, \cite{Danese}, \cite{Dattoli2001},
\cite{DilcherStolarsky}, \cite{ElbertLaforgia94}, \cite{Erdelyi+}, \cite{Grad},
\cite{Mathai}, \cite{Nikiforov+}, \cite{NikiforovUvarov}, \cite{Olver}, \cite{Rainville},
\cite{Szego75}, \cite{Temme}, \cite{Zayed}.

\newpage

\section{Pseudo Jacobi}\index{Pseudo Jacobi polynomials}\index{Jacobi polynomials!Pseudo}

\par\setcounter{equation}{0}

\subsection*{Hypergeometric representation}
\begin{eqnarray}
\label{DefPseudoJacobi}
P_n(x;\nu,N)&=&\frac{(-2i)^n(-N+i\nu)_n}{(n-2N-1)_n}\,\hyp{2}{1}{-n,n-2N-1}{-N+i\nu}{\frac{1-ix}{2}}\\
&=&(x+i)^n\,\hyp{2}{1}{-n,N+1-n-i\nu}{2N+2-2n}{\frac{2}{1-ix}},\quad n=0,1,2,\ldots,N.\nonumber
\end{eqnarray}

\subsection*{Orthogonality relation}
\begin{eqnarray}
\label{OrtPseudoJacobi}
& &\frac{1}{2\pi}\int_{-\infty}^{\infty}(1+x^2)^{-N-1}\e^{2\nu\arctan x}P_m(x;\nu,N)P_n(x;\nu,N)\,dx\nonumber\\
& &{}=\frac{\Gamma(2N+1-2n)\Gamma(2N+2-2n)2^{2n-2N-1}n!}{\Gamma(2N+2-n)\left|\Gamma(N+1-n+i\nu)\right|^2}\,\delta_{mn}.
\end{eqnarray}

\subsection*{Recurrence relation}
\begin{eqnarray}
\label{RecPseudoJacobi}
xP_n(x;\nu,N)&=&P_{n+1}(x;\nu,N)+\frac{(N+1)\nu}{(n-N-1)(n-N)}P_n(x;\nu,N)\nonumber\\
& &{}\mathindent{}-\frac{n(n-2N-2)}{(2n-2N-3)(n-N-1)^2(2n-2N-1)}\nonumber\\
& &{}\mathindent\mathindent\times(n-N-1-i\nu)(n-N-1+i\nu)P_{n-1}(x;\nu,N).
\end{eqnarray}

\subsection*{Normalized recurrence relation}
\begin{eqnarray}
\label{NormRecPseudoJacobi}
xp_n(x)&=&p_{n+1}(x)+\frac{(N+1)\nu}{(n-N-1)(n-N)}p_n(x)\nonumber\\
& &{}\mathindent{}-\frac{n(n-2N-2)(n-N-1-i\nu)(n-N-1+i\nu)}{(2n-2N-3)(n-N-1)^2(2n-2N-1)}p_{n-1}(x),
\end{eqnarray}
where
$$P_n(x;\nu,N)=p_n(x).$$

\subsection*{Differential equation}
\begin{equation}
\label{dvPseudoJacobi}
(1+x^2)y''(x)+2\left(\nu-Nx\right)y'(x)-n(n-2N-1)y(x)=0,
\end{equation}
where
$$y(x)=P_n(x;\nu,N).$$

\subsection*{Forward shift operator}
\begin{equation}
\label{shift1PseudoJacobi}
\frac{d}{dx}P_n(x;\nu,N)=nP_{n-1}(x;\nu,N-1).
\end{equation}

\subsection*{Backward shift operator}
\begin{eqnarray}
\label{shift2PseudoJacobiI}
& &(1+x^2)\frac{d}{dx}P_n(x;\nu,N)+2\left[\nu-(N+1)x\right]P_n(x;\nu,N)\nonumber\\
& &{}=(n-2N-2)P_{n+1}(x;\nu,N+1)
\end{eqnarray}
or equivalently
\begin{eqnarray}
\label{shift2PseudoJacobiII}
& &\frac{d}{dx}\left[(1+x^2)^{-N-1}\e^{2\nu\arctan x}P_n(x;\nu,N)\right]\nonumber\\
& &{}=(n-2N-2)(1+x^2)^{-N-2}\e^{2\nu\arctan x}P_{n+1}(x;\nu,N+1).
\end{eqnarray}

\subsection*{Rodrigues-type formula}
\begin{equation}
\label{RodPseudoJacobi}
P_n(x;\nu,N)=\frac{(1+x^2)^{N+1}\expe^{-2\nu\arctan x}}{(n-2N-1)_n}
\left(\frac{d}{dx}\right)^n\left[(1+x^2)^{n-N-1}\expe^{2\nu\arctan x}\right].
\end{equation}

\subsection*{Generating function}
\begin{eqnarray}
\label{GenPseudoJacobi}
& &\left[\hyp{0}{1}{-}{-N+i\nu}{(x+i)t}\,\hyp{0}{1}{-}{-N-i\nu}{(x-i)t}\right]_N\nonumber\\
& &{}=\sum_{n=0}^N\frac{(n-2N-1)_n}{(-N+i\nu)_n(-N-i\nu)_nn!}P_n(x;\nu,N)t^n.
\end{eqnarray}

\subsection*{Limit relation}

\subsubsection*{Continuous Hahn $\rightarrow$ Pseudo Jacobi}
The pseudo Jacobi polynomials follow from the continuous Hahn polynomials given by
(\ref{DefContHahn}) by the substitutions $x\rightarrow xt$, $a=\frac{1}{2}(-N+i\nu-2t)$,
$b=\frac{1}{2}(-N-i\nu+2t)$, $c=\frac{1}{2}(-N-i\nu-2t)$ and $d=\frac{1}{2}(-N+i\nu+2t)$,
division by $t^n$ and the limit $t\rightarrow\infty$:
\begin{eqnarray*}
& &\lim_{t\rightarrow\infty}\frac{p_n(xt;\frac{1}{2}(-N+i\nu-2t),\frac{1}{2}(-N-i\nu+2t),
\frac{1}{2}(-N+i\nu-2t),\frac{1}{2}(-N-i\nu+2t))}{t^n}\\
& &{}=\frac{(n-2N-1)_n}{n!}P_n(x;\nu,N).
\end{eqnarray*}

\subsection*{Remarks}
Since we have for $k<n$
$$\frac{(-N+i\nu)_n}{(-N+i\nu)_k}=(-N+i\nu+k)_{n-k},$$
the pseudo Jacobi polynomials given by (\ref{DefPseudoJacobi}) can also be seen as
polynomials in the parameter~$\nu$.

\noindent
The weight function for the pseudo Jacobi polynomials can be written as
$$(1+x^2)^{-N-1}\e^{2\nu\arctan x}=(1+ix)^{-N-1-i\nu}(1-ix)^{-N-1+i\nu}.$$

\noindent
The pseudo Jacobi polynomials are related to the Jacobi polynomials defined by
(\ref{DefJacobi}) in the following way:
$$P_n(x;\nu,N)=\frac{(-2i)^nn!}{(n-2N-1)_n}P_n^{(-N-1+i\nu,-N-1-i\nu)}(ix).$$

\noindent
If we set $x\rightarrow\nu x$ in the definition (\ref{DefPseudoJacobi}) of the pseudo Jacobi
polynomials and take the limit $\nu\rightarrow\infty$ we obtain a special case of the Bessel
polynomials given by (\ref{DefBessel}) in the following way:
$$\lim\limits_{\nu\rightarrow\infty}\frac{P_n(\nu x;\nu,N)}{\nu^n}
=\frac{2^n}{(n-2N-1)_n}y_n(x;-2N-2).$$

\subsection*{References}
\cite{Askey87}, \cite{BorodinOlshanski}, \cite{Lesky96}.


\section{Meixner}\index{Meixner polynomials}

\par\setcounter{equation}{0}

\subsection*{Hypergeometric representation}
\begin{equation}
\label{DefMeixner}
M_n(x;\beta,c)=\hyp{2}{1}{-n,-x}{\beta}{1-\frac{1}{c}}.
\end{equation}

\subsection*{Orthogonality relation}
\begin{equation}
\label{OrtMeixner}
\sum_{x=0}^{\infty}\frac{(\beta)_x}{x!}c^xM_m(x;\beta,c)M_n(x;\beta,c)
{}=\frac{c^{-n}n!}{(\beta)_n(1-c)^{\beta}}\,\delta_{mn}
%  \constraint{
%    $\beta > 0$ &
%    $0 < c < 1$ }
\end{equation}

\subsection*{Recurrence relation}
\begin{eqnarray}
\label{RecMeixner}
(c-1)xM_n(x;\beta,c)&=&c(n+\beta)M_{n+1}(x;\beta,c)\nonumber\\
& &{}\mathindent{}-\left[n+(n+\beta)c\right]M_n(x;\beta,c)+nM_{n-1}(x;\beta,c).
\end{eqnarray}

\subsection*{Normalized recurrence relation}
\begin{equation}
\label{NormRecMeixner}
xp_n(x)=p_{n+1}(x)+\frac{n+(n+\beta)c}{1-c}p_n(x)+
\frac{n(n+\beta-1)c}{(1-c)^2}p_{n-1}(x),
\end{equation}
where
$$M_n(x;\beta,c)=\frac{1}{(\beta)_n}\left(\frac{c-1}{c}\right)^np_n(x).$$

\subsection*{Difference equation}
\begin{equation}
\label{dvMeixner}
n(c-1)y(x)=c(x+\beta)y(x+1)-\left[x+(x+\beta)c\right]y(x)+xy(x-1),
\end{equation}
where
$$y(x)=M_n(x;\beta,c).$$

\subsection*{Forward shift operator}
\begin{equation}
\label{shift1MeixnerI}
M_n(x+1;\beta,c)-M_n(x;\beta,c)=
\frac{n}{\beta}\left(\frac{c-1}{c}\right)M_{n-1}(x;\beta+1,c)
\end{equation}
or equivalently
\begin{equation}
\label{shift1MeixnerII}
\Delta M_n(x;\beta,c)=\frac{n}{\beta}\left(\frac{c-1}{c}\right)M_{n-1}(x;\beta+1,c).
\end{equation}

\subsection*{Backward shift operator}
\begin{equation}
\label{shift2MeixnerI}
c(\beta+x-1)M_n(x;\beta,c)-xM_n(x-1;\beta,c)=c(\beta-1)M_{n+1}(x;\beta-1,c)
\end{equation}
or equivalently
\begin{equation}
\label{shift2MeixnerII}
\nabla\left[\frac{(\beta)_xc^x}{x!}M_n(x;\beta,c)\right]=
\frac{(\beta-1)_xc^x}{x!}M_{n+1}(x;\beta-1,c).
\end{equation}

\subsection*{Rodrigues-type formula}
\begin{equation}
\label{RodMeixner}
\frac{(\beta)_xc^x}{x!}M_n(x;\beta,c)=\nabla^n\left[\frac{(\beta+n)_xc^x}{x!}\right].
\end{equation}

\subsection*{Generating functions}
\begin{equation}
\label{GenMeixner1}
\left(1-\frac{t}{c}\right)^x(1-t)^{-x-\beta}=
\sum_{n=0}^{\infty}\frac{(\beta)_n}{n!}M_n(x;\beta,c)t^n.
\end{equation}

\begin{equation}
\label{GenMeixner2}
\e^t\,\hyp{1}{1}{-x}{\beta}{\left(\frac{1-c}{c}\right)t}=
\sum_{n=0}^{\infty}\frac{M_n(x;\beta,c)}{n!}t^n.
\end{equation}

\begin{equation}
\label{GenMeixner3}
(1-t)^{-\gamma}\,\hyp{2}{1}{\gamma,-x}{\beta}{\frac{(1-c)t}{c(1-t)}}
=\sum_{n=0}^{\infty}\frac{(\gamma)_n}{n!}M_n(x;\beta,c)t^n,
\quad\textrm{$\gamma$ arbitrary}.
\end{equation}

\subsection*{Limit relations}

\subsubsection*{Hahn $\rightarrow$ Meixner}
If we take $\alpha=b-1$, $\beta=N(1-c)c^{-1}$ in the definition (\ref{DefHahn})
of the Hahn polynomials and let $N\rightarrow\infty$ we find the Meixner polynomials:
$$\lim_{N\rightarrow\infty}
Q_n(x;b-1,N(1-c)c^{-1},N)=M_n(x;b,c).$$

\subsubsection*{Dual Hahn $\rightarrow$ Meixner}
To obtain the Meixner polynomials from the dual Hahn polynomials we have to take
$\gamma=\beta-1$ and $\delta=N(1-c)c^{-1}$ in the definition (\ref{DefDualHahn}) of
the dual Hahn polynomials and let $N\rightarrow\infty$:
$$\lim_{N\rightarrow\infty}
R_n(\lambda(x);\beta-1,N(1-c)c^{-1},N)=M_n(x;\beta,c).$$

\subsubsection*{Meixner $\rightarrow$ Laguerre}
The Laguerre polynomials given by (\ref{DefLaguerre}) are obtained from the Meixner polynomials
if we take $\beta=\alpha+1$ and $x\rightarrow (1-c)^{-1}x$ and let $c\rightarrow 1$:
\begin{equation}
\lim_{c\rightarrow 1}
M_n((1-c)^{-1}x;\alpha+1,c)=\frac{L_n^{(\alpha)}(x)}{L_n^{(\alpha)}(0)}.
\end{equation}

\subsubsection*{Meixner $\rightarrow$ Charlier}
The Charlier polynomials given by (\ref{DefCharlier}) are obtained from the Meixner polynomials
if we take $c=(a+\beta)^{-1}a$ and let $\beta\rightarrow\infty$:
\begin{equation}
\lim_{\beta\rightarrow\infty}
M_n(x;\beta,(a+\beta)^{-1}a)=C_n(x;a).
\end{equation}

\subsection*{Remarks}
The Meixner polynomials are related to the Jacobi polynomials given by (\ref{DefJacobi})
in the following way:
$$\frac{(\beta)_n}{n!}M_n(x;\beta,c)=P_n^{(\beta-1,-n-\beta-x)}((2-c)c^{-1}).$$

\noindent
The Meixner polynomials are also related to the Krawtchouk polynomials given by
(\ref{DefKrawtchouk}) in the following way:
$$K_n(x;p,N)=M_n(x;-N,(p-1)^{-1}p).$$
% RS: add begin\label{sec9.9}
In this section in \mycite{KLS} the pseudo Jacobi polynomial $P_n(x;\nu,N)$ in (9.9.1)
is considered
for $N\in\ZZ_{\ge0}$ and $n=0,1,\ldots,n$. However, we can more generally take
$-\thalf<N\in\RR$ (so here I overrule my convention formulated in the
beginning of this paper), $N_0$ integer such that $N-\thalf\le N_0<N+\thalf$, and $n=0,1,\ldots,N_0$
(see \myciteKLS{382}{\S5, case A.4}). The orthogonality relation (9.9.2)
is valid for $m,n=0,1,\ldots,N_0$.
%
\paragraph{\large\bf KLSadd: History}These polynomials were first obtained by Routh \cite{K13} in 1885, and later, independently,
by Romanovski \myciteKLS{463} in 1929.
%
\paragraph{\large\bf KLSadd: Limit relation:}{\bf Pseudo big $q$-Jacobi $\longrightarrow$ Pseudo Jacobi}\\
See also \eqref{118}.
%
\paragraph{\large\bf KLSadd: References}See also \mycite{Ism}{\S20.1}, \myciteKLS{51},
\myciteKLS{384}, \cite{K11}, \cite{K10}, \cite{K12}.
%
% RS: add end
\subsection*{References}
\cite{Allaway76}, \cite{NAlSalam66}, \cite{AlSalam90}, \cite{AlSalamChihara76},
\cite{AlSalamIsmail76}, \cite{Alvarez+}, \cite{AndrewsAskey85}, \cite{Area+II}, \cite{Askey75},
\cite{Askey89I}, \cite{Askey2005}, \cite{AskeyGasper77}, \cite{AskeyIsmail76},
\cite{AskeyWilson85}, \cite{AtakRahmanSuslov}, \cite{AtakSuslov88}, \cite{Bavinck98},
\cite{BavinckHaeringen}, \cite{Campigotto+}, \cite{Chihara78}, \cite{Cooper+},
\cite{Erdelyi+}, \cite{FoataLabelle}, \cite{Gabutti}, \cite{GabuttiMathis}, \cite{Gasper73I},
\cite{Gasper74}, \cite{HoareRahman}, \cite{Ismail2005II}, \cite{IsmailLetVal88},
\cite{IsmailLi}, \cite{IsmailMuldoon}, \cite{IsmailStanton97}, \cite{JinWong},
\cite{Karlin58}, \cite{Koekoek2000}, \cite{Koorn88}, \cite{LabelleYehI}, \cite{LabelleYehII},
\cite{Lesky89}, \cite{Lesky94I}, \cite{Lesky95II}, \cite{LewanowiczII}, \cite{Meixner},
\cite{Nikiforov+}, \cite{NikiforovUvarov}, \cite{Rahman78I}, \cite{ValentAssche},
\cite{Viennot}, \cite{Zarzo+}, \cite{Zeng90}.


\section{Krawtchouk}\index{Krawtchouk polynomials}

\par\setcounter{equation}{0}

\subsection*{Hypergeometric representation}
\begin{equation}
\label{DefKrawtchouk}
K_n(x;p,N)=\hyp{2}{1}{-n,-x}{-N}{\frac{1}{p}},\quad n=0,1,2,\ldots,N.
\end{equation}

\subsection*{Orthogonality relation}
\begin{eqnarray}
\label{OrtKrawtchouk}
& &\sum_{x=0}^N\binom{N}{x}p^x(1-p)^{N-x} K_m(x;p,N)K_n(x;p,N)\nonumber\\
& &{}=\frac{(-1)^nn!}{(-N)_n}\left(\frac{1-p}{p}\right)^n\,\delta_{mn},\quad0 < p < 1.
\end{eqnarray}

\subsection*{Recurrence relation}
\begin{eqnarray}
\label{RecKrawtchouk}
-xK_n(x;p,N)&=&p(N-n)K_{n+1}(x;p,N)\nonumber\\
& &{}\mathindent{}-\left[p(N-n)+n(1-p)\right]K_n(x;p,N)\nonumber\\
& &{}\mathindent\mathindent{}+n(1-p)K_{n-1}(x;p,N).
\end{eqnarray}

\subsection*{Normalized recurrence relation}
\begin{eqnarray}
\label{NormRecKrawtchouk}
xp_n(x)&=&p_{n+1}(x)+\left[p(N-n)+n(1-p)\right]p_n(x)\nonumber\\
& &{}\mathindent{}+np(1-p)(N+1-n)p_{n-1}(x),
\end{eqnarray}
where
$$K_n(x;p,N)=\frac{1}{(-N)_np^n}p_n(x).$$

\subsection*{Difference equation}
\begin{eqnarray}
\label{dvKrawtchouk}
-ny(x)&=&p(N-x)y(x+1)\nonumber\\
& &{}\mathindent{}-\left[p(N-x)+x(1-p)\right]y(x)+x(1-p)y(x-1),
\end{eqnarray}
where
$$y(x)=K_n(x;p,N).$$

\subsection*{Forward shift operator}
\begin{equation}
\label{shift1KrawtchoukI}
K_n(x+1;p,N)-K_n(x;p,N)=-\frac{n}{Np}K_{n-1}(x;p,N-1)
\end{equation}
or equivalently
\begin{equation}
\label{shift1KrawtchoukII}
\Delta K_n(x;p,N)=-\frac{n}{Np}K_{n-1}(x;p,N-1).
\end{equation}

\subsection*{Backward shift operator}
\begin{eqnarray}
\label{shift2KrawtchoukI}
& &(N+1-x)K_n(x;p,N)-x\left(\frac{1-p}{p}\right)K_n(x-1;p,N)\nonumber\\
& &{}=(N+1)K_{n+1}(x;p,N+1)
\end{eqnarray}
or equivalently
\begin{equation}
\label{shift2KrawtchoukII}
\nabla\left[\binom{N}{x}\left(\frac{p}{1-p}\right)^xK_n(x;p,N)\right]=
\binom{N+1}{x}\left(\frac{p}{1-p}\right)^xK_{n+1}(x;p,N+1).
\end{equation}

\subsection*{Rodrigues-type formula}
\begin{equation}
\label{RodKrawtchouk}
\binom{N}{x}\left(\frac{p}{1-p}\right)^xK_n(x;p,N)=
\nabla^n\left[\binom{N-n}{x}\left(\frac{p}{1-p}\right)^x\right].
\end{equation}

\subsection*{Generating functions}
For $x=0,1,2,\ldots,N$ we have
\begin{equation}
\label{GenKrawtchouk1}
\left(1-\frac{(1-p)}{p}t\right)^x(1+t)^{N-x}=
\sum_{n=0}^N\binom{N}{n}K_n(x;p,N)t^n.
\end{equation}

\begin{equation}
\label{GenKrawtchouk2}
\left[\expe^t\,\hyp{1}{1}{-x}{-N}{-\frac{t}{p}}\right]_N=
\sum_{n=0}^N\frac{K_n(x;p,N)}{n!}t^n.
\end{equation}

\begin{eqnarray}
\label{GenKrawtchouk3}
& &\left[(1-t)^{-\gamma}\,\hyp{2}{1}{\gamma,-x}{-N}{\frac{t}{p(t-1)}}\right]_N\nonumber\\
& &{}=\sum_{n=0}^N\frac{(\gamma)_n}{n!}K_n(x;p,N)t^n,
\quad\textrm{$\gamma$ arbitrary}.
\end{eqnarray}

\subsection*{Limit relations}

\subsubsection*{Hahn $\rightarrow$ Krawtchouk}
If we take $\alpha=pt$ and $\beta=(1-p)t$ in the definition (\ref{DefHahn}) of the Hahn
polynomials and let $t\rightarrow\infty$ we obtain the Krawtchouk polynomials:
$$\lim_{t\rightarrow\infty}Q_n(x;pt,(1-p)t,N)=K_n(x;p,N).$$

\subsubsection*{Dual Hahn $\rightarrow$ Krawtchouk}
The Krawtchouk polynomials follow from the dual Hahn polynomials given by
(\ref{DefDualHahn}) if we set $\gamma=pt$, $\delta=(1-p)t$ and let $t\rightarrow\infty$:
$$\lim_{t\rightarrow\infty}R_n(\lambda(x);pt,(1-p)t,N)=K_n(x;p,N).$$

\subsubsection*{Krawtchouk $\rightarrow$ Charlier}
The Charlier polynomials given by (\ref{DefCharlier}) can be found from the Krawtchouk
polynomials by taking $p=N^{-1}a$ and letting $N\rightarrow\infty$:
\begin{equation}
\lim_{N\rightarrow\infty}K_n(x;N^{-1}a,N)=C_n(x;a).
\end{equation}

\subsubsection*{Krawtchouk $\rightarrow$ Hermite}
The Hermite polynomials given by (\ref{DefHermite}) follow from the Krawtchouk polynomials
by setting $x\rightarrow pN+x\sqrt{2p(1-p)N}$ and then letting $N\rightarrow\infty$:
\begin{equation}
\lim_{N\rightarrow\infty}
\sqrt{\binom{N}{n}}K_n(pN+x\sqrt{2p(1-p)N};p,N)
=\frac{\displaystyle (-1)^nH_n(x)}{\displaystyle\sqrt{2^nn!\left(\frac{p}{1-p}\right)^n}}.
\end{equation}

\subsection*{Remarks}
The Krawtchouk polynomials are self-dual, which means that
$$K_n(x;p,N)=K_x(n;p,N),\quad n,x\in\{0,1,2,\ldots,N\}.$$
By using this relation we easily obtain the so-called dual orthogonality
relation from the orthogonality relation (\ref{OrtKrawtchouk}):
$$\sum_{n=0}^N\binom{N}{n}p^n(1-p)^{N-n} K_n(x;p,N)K_n(y;p,N)=
\frac{\displaystyle\left(\frac{1-p}{p}\right)^x}{\dbinom{N}{x}}\delta_{xy},$$
where $0 < p < 1$ and $x,y\in\{0,1,2,\ldots,N\}$.

\noindent
The Krawtchouk polynomials are related to the Meixner polynomials given by (\ref{DefMeixner})
in the following way:
$$K_n(x;p,N)=M_n(x;-N,(p-1)^{-1}p).$$
% RS: add begin\label{sec9.10}
\paragraph{\large\bf KLSadd: History}In 1934 Meixner \myciteKLS{406} (see
(1.1) and case IV on pp.~10, 11 and 12) gave the orthogonality
measure for the polynomials $P_n$ given by the generating function
\[
e^{x u(t)}\,f(t)=\sum_{n=0}^\iy P_n(x)\,\frac{t^n}{n!}\,,
\]
where
\[
e^{u(t)}=\left(\frac{1-\be t}{1-\al t}\right)^{\frac1{\al-\be}},\quad
f(t)=\frac{(1-\be t)^{\frac{k_2}{\be(\al-\be)}}}{(1-\al t)^{\frac{k_2}{\al(\al-\be)}}}\quad
(k_2<0;\;\al>\be>0\;\;{\rm or}\;\;\al<\be<0).
\]
Then $P_n$ can be expressed as a Meixner polynomial:
\[
P_n(x)=(-k_2(\al\be)^{-1})_n\,\be^n\,
M_n\left(-\,\frac{x+k_2\al^{-1}}{\al-\be},-k_2(\al\be)^{-1},\be\al^{-1}\right).
\]

In 1938 Gottlieb \cite[\S2]{K1} introduces polynomials $l_n$ ``of Laguerre type''
which turn out to be special Meixner polynomials:
$l_n(x)=e^{-n\la} M_n(x;1,e^{-\la})$.
%
\paragraph{\large\bf KLSadd: Uniqueness of orthogonality measure}The coefficient of $p_{n-1}(x)$ in (9.10.4) behaves as $O(n^2)$ as $n\to\iy$.
Hence \eqref{93} holds, by which the orthogonality measure is unique.
%
% RS: add end
\subsection*{References}
\cite{AlSalam90}, \cite{AndrewsAskey85}, \cite{Area+II}, \cite{Askey75}, \cite{Askey89I},
\cite{AskeyGasper77}, \cite{AskeyWilson85}, \cite{AtakRahmanSuslov}, \cite{Campigotto+},
\cite{LChiharaStanton}, \cite{Chihara78}, \cite{Dette95}, \cite{Dominici}, \cite{Dunkl76},
\cite{Dunkl84}, \cite{DunklRamirez}, \cite{Erdelyi+}, \cite{FeinsilverSchott},
\cite{Gasper73I}, \cite{Gasper74}, \cite{HoareRahman}, \cite{Ismail2005II}, \cite{Karlin58},
\cite{Koorn82}, \cite{Koorn88}, \cite{LabelleYehI}, \cite{LabelleYehII}, \cite{Lesky62},
\cite{Lesky89}, \cite{Lesky94I}, \cite{Lesky95II}, \cite{LewanowiczII}, \cite{Nikiforov+},
\cite{NikiforovUvarov}, \cite{Qiu}, \cite{Rahman78I}, \cite{Rahman79}, \cite{Stanton84},
\cite{Stanton90}, \cite{Szego75}, \cite{Zarzo+}, \cite{Zeng90}.


\section{Laguerre}\index{Laguerre polynomials}

\par\setcounter{equation}{0}

\subsection*{Hypergeometric representation}
\begin{equation}
\label{DefLaguerre}
L_n^{(\alpha)}(x)=\frac{(\alpha+1)_n}{n!}\,\hyp{1}{1}{-n}{\alpha+1}{x}.
\end{equation}

\subsection*{Orthogonality relation}
\begin{equation}
\label{OrtLaguerre}
\int_0^{\infty}\expe^{-x}x^{\alpha}L_m^{(\alpha)}(x)L_n^{(\alpha)}(x)\,dx=
\frac{\Gamma(n+\alpha+1)}{n!}\,\delta_{mn},\quad\alpha > -1.
\end{equation}

\subsection*{Recurrence relation}
\begin{equation}
\label{RecLaguerre}
(n+1)L_{n+1}^{(\alpha)}(x)-(2n+\alpha+1-x)L_n^{(\alpha)}(x)+(n+\alpha)L_{n-1}^{(\alpha)}(x)=0.
\end{equation}

\subsection*{Normalized recurrence relation}
\begin{equation}
\label{NormRecLaguerre}
xp_n(x)=p_{n+1}(x)+(2n+\alpha+1)p_n(x)+n(n+\alpha)p_{n-1}(x),
\end{equation}
where
$$L_n^{(\alpha)}(x)=\frac{(-1)^n}{n!}p_n(x).$$

\subsection*{Differential equation}
\begin{equation}
\label{dvLaguerre}
xy''(x)+(\alpha+1-x)y'(x)+ny(x)=0,\quad y(x)=L_n^{(\alpha)}(x).
\end{equation}

\newpage

\subsection*{Forward shift operator}
\begin{equation}
\label{shift1Laguerre}
\frac{d}{dx}L_n^{(\alpha)}(x)=-L_{n-1}^{(\alpha+1)}(x).
\end{equation}

\subsection*{Backward shift operator}
\begin{equation}
\label{shift2LaguerreI}
x\frac{d}{dx}L_n^{(\alpha)}(x)+(\alpha-x)L_n^{(\alpha)}(x)=(n+1)L_{n+1}^{(\alpha-1)}(x)
\end{equation}
or equivalently
\begin{equation}
\label{shift2LaguerreII}
\frac{d}{dx}\left[\expe^{-x}x^{\alpha}L_n^{(\alpha)}(x)\right]=(n+1)\expe^{-x}x^{\alpha-1}L^{(\alpha-1)}_{n+1}(x).
\end{equation}

\subsection*{Rodrigues-type formula}
\begin{equation}
\label{RodLaguerre}
\e^{-x}x^{\alpha}L_n^{(\alpha)}(x)=\frac{1}{n!}\left(\frac{d}{dx}\right)^n\left[\expe^{-x}x^{n+\alpha}\right].
\end{equation}

\subsection*{Generating functions}
\begin{equation}
\label{GenLaguerre1}
(1-t)^{-\alpha-1}\exp\left(\frac{xt}{t-1}\right)=
\sum_{n=0}^{\infty}L_n^{(\alpha)}(x)t^n.
\end{equation}

\begin{equation}
\label{GenLaguerre2}
\e^t\,\hyp{0}{1}{-}{\alpha+1}{-xt}
=\sum_{n=0}^{\infty}\frac{L_n^{(\alpha)}(x)}{(\alpha+1)_n}t^n.
\end{equation}

\begin{equation}
\label{GenLaguerre3}
(1-t)^{-\gamma}\,\hyp{1}{1}{\gamma}{\alpha+1}{\frac{xt}{t-1}}
=\sum_{n=0}^{\infty}\frac{(\gamma)_n}{(\alpha+1)_n}L_n^{(\alpha)}(x)t^n,
\quad\textrm{$\gamma$ arbitrary}.
\end{equation}

\subsection*{Limit relations}

\subsubsection*{Meixner-Pollaczek $\rightarrow$ Laguerre}
The Laguerre polynomials can be obtained from the Meixner-Pollaczek polynomials given by
(\ref{DefMP}) by the substitution $\lambda=\frac{1}{2}(\alpha+1)$,
$x\rightarrow -\frac{1}{2}\phi^{-1}x$ and the limit $\phi\rightarrow 0$:
$$\lim_{\phi\rightarrow 0}
P_n^{(\frac{1}{2}\alpha+\frac{1}{2})}(-\textstyle\frac{1}{2}\phi^{-1}x;\phi)=L_n^{(\alpha)}(x).$$

\subsubsection*{Jacobi $\rightarrow$ Laguerre}
The Laguerre polynomials are obtained from the Jacobi polynomials given by (\ref{DefJacobi})
if we set $x\rightarrow 1-2\beta^{-1}x$ and then take the limit $\beta\rightarrow\infty$:
$$\lim_{\beta\rightarrow\infty}
P_n^{(\alpha,\beta)}(1-2\beta^{-1}x)=L_n^{(\alpha)}(x).$$

\subsubsection*{Meixner $\rightarrow$ Laguerre}
If we take $\beta=\alpha+1$ and $x\rightarrow (1-c)^{-1}x$ in the definition
(\ref{DefMeixner}) of the Meixner polynomials and let $c\rightarrow 1$ we obtain
the Laguerre polynomials:
$$\lim_{c\rightarrow 1}
M_n((1-c)^{-1}x;\alpha+1,c)=\frac{L_n^{(\alpha)}(x)}{L_n^{(\alpha)}(0)}.$$

\subsubsection*{Laguerre $\rightarrow$ Hermite}
The Hermite polynomials given by (\ref{DefHermite}) can be obtained from the Laguerre
polynomials by taking the limit $\alpha\rightarrow\infty$ in the following way:
\begin{equation}
\lim_{\alpha\rightarrow\infty}
\left(\frac{2}{\alpha}\right)^{\frac{1}{2}n}
L_n^{(\alpha)}((2\alpha)^{\frac{1}{2}}x+\alpha)=\frac{(-1)^n}{n!}H_n(x).
\end{equation}

\subsection*{Remarks}
The definition (\ref{DefLaguerre}) of the Laguerre polynomials can also be
written as:
$$L_n^{(\alpha)}(x)=\frac{1}{n!}\sum_{k=0}^n\frac{(-n)_k}{k!}(\alpha+k+1)_{n-k}x^k.$$
In this way the Laguerre polynomials can also be seen as polynomials in the parameter $\alpha$.
Therefore they can be defined for all $\alpha$.

\noindent
The Laguerre polynomials are related to the Bessel polynomials given by (\ref{DefBessel})
in the following way:
$$L_n^{(\alpha)}(x)=\frac{(-x)^n}{n!}y_n(2x^{-1};-2n-\alpha-1).$$

\noindent
The Laguerre polynomials are related to the Charlier polynomials given by (\ref{DefCharlier})
in the following way:
$$\frac{(-a)^n}{n!}C_n(x;a)=L_n^{(x-n)}(a).$$

\noindent
The Laguerre polynomials and the Hermite polynomials given by (\ref{DefHermite}) are also
connected by the following quadratic transformations:
$$H_{2n}(x)=(-1)^nn!\,2^{2n}L_n^{(-\frac{1}{2})}(x^2)$$
and
$$H_{2n+1}(x)=(-1)^nn!\,2^{2n+1}xL_n^{(\frac{1}{2})}(x^2).$$

\noindent
In combinatorics the Laguerre polynomials with $\alpha=0$ are often called Rook
polynomials.
% RS: add begin\label{sec9.11}
%
\paragraph{\large\bf KLSadd: Special values}By (9.11.1) and the binomial formula:
\begin{equation}
K_n(0;p,N)=1,\qquad
K_n(N;p,N)=(1-p^{-1})^n.
\label{9}
\end{equation}
The self-duality (p.240, Remarks, first formula)
\begin{equation}
K_n(x;p,N)=K_x(n;p,N)\qquad (n,x\in \{0,1,\ldots,N\})
\label{147}
\end{equation}
combined with \eqref{9} yields:
\begin{equation}
K_N(x;p,N)=(1-p^{-1})^x\qquad(x\in\{0,1,\ldots,N\}).
\label{148}
\end{equation}
%
\paragraph{\large\bf KLSadd: Symmetry}By the orthogonality relation (9.11.2):
\begin{equation}
\frac{K_n(N-x;p,N)}{K_n(N;p,N)}=K_n(x;1-p,N).
\label{10}
\end{equation}
By \eqref{10} and \eqref{147} we have also
\begin{equation}
\frac{K_{N-n}(x;p,N)}{K_N(x;p,N)}=K_n(x;1-p,N)
\qquad(n,x\in\{0,1,\ldots,N\}),
\label{149}
\end{equation}
and, by \eqref{149}, \eqref{10} and \eqref{9},
\begin{equation}
K_{N-n}(N-x;p,N)=\left(\frac p{p-1}\right)^{n+x-N}K_n(x;p,N)
\qquad(n,x\in\{0,1,\ldots,N\}).
\label{150}
\end{equation}
A particular case of \eqref{10} is:
\begin{equation}
K_n(N-x;\thalf,N)=(-1)^n K_n(x;\thalf,N).
\label{11}
\end{equation}
Hence
\begin{equation}
K_{2m+1}(N;\thalf,2N)=0.
\label{12}
\end{equation}
From (9.11.11):
\begin{equation}
K_{2m}(N;\thalf,2N)=\frac{(\thalf)_m}{(-N+\thalf)_m}\,.
\label{13}
\end{equation}
%
\paragraph{\large\bf KLSadd: Quadratic transformations}\begin{align}
K_{2m}(x+N;\thalf,2N)&=\frac{(\thalf)_m}{(-N+\thalf)_m}\,
R_m(x^2;-\thalf,-\thalf,N),
\label{31}\\
K_{2m+1}(x+N;\thalf,2N)&=-\,\frac{(\tfrac32)_m}{N\,(-N+\thalf)_m}\,
x\,R_m(x^2-1;\thalf,\thalf,N-1),
\label{33}\\
K_{2m}(x+N+1;\thalf,2N+1)&=\frac{(\tfrac12)_m}{(-N-\thalf)_m}\,
R_m(x(x+1);-\thalf,\thalf,N),
\label{32}\\
K_{2m+1}(x+N+1;\thalf,2N+1)&=\frac{(\tfrac32)_m}{(-N-\thalf)_{m+1}}\,
(x+\thalf)\,R_m(x(x+1);\thalf,-\thalf,N),
\label{34}
\end{align}
where $R_m$ is a dual Hahn polynomial (9.6.1). For the proofs use
(9.6.2), (9.11.2), (9.6.4) and (9.11.4).
%
\paragraph{\large\bf KLSadd: Generating functions}\begin{multline}
\sum_{x=0}^N\binom Nx K_m(x;p,N)K_n(x;q,N)z^x\\
=\left(\frac{p-z+pz}p\right)^m
\left(\frac{q-z+qz}q\right)^n
(1+z)^{N-m-n}
K_m\left(n;-\,\frac{(p-z+pz)(q-z+qz)}z,N\right).
\label{107}
\end{multline}
This follows immediately from Rosengren \cite[(3.5)]{K8}, which goes back
to Meixner \cite{K9}.
%
% RS: add end
\subsection*{References}
\cite{Abdul}, \cite{Abram}, \cite{Ahmed+82}, \cite{Allaway76}, \cite{Allaway91},
\cite{NAlSalam66}, \cite{AlSalam64}, \cite{AlSalam90}, \cite{AlSalamChihara72},
\cite{AlSalamChihara76}, \cite{AndrewsAskey85}, \cite{AndrewsAskeyRoy}, \cite{Askey68}, 
\cite{Askey75}, \cite{Askey89I}, \cite{Askey2005}, \cite{AskeyGasper76}, \cite{AskeyGasper77},
\cite{AskeyIsmail76}, \cite{AskeyIsmailKoorn}, \cite{AskeyWilson85}, \cite{Barrett},
\cite{Bavinck96}, \cite{Brafman51}, \cite{Brafman57II}, \cite{Brenke}, \cite{Brown},
\cite{BustozSavage79}, \cite{BustozSavage80}, \cite{Carlitz60}, \cite{Carlitz61I},
\cite{Carlitz61II}, \cite{Carlitz62}, \cite{Carlitz68}, \cite{ChenSrivastava},
\cite{ChenIsmail91}, \cite{Chihara68II}, \cite{Chihara78}, \cite{Cohen}, \cite{Cooper+},
\cite{Dattoli2001}, \cite{DetteStudden92}, \cite{DetteStudden95}, \cite{Dimitrov2003},
\cite{Doha2003II}, \cite{ElbertLaforgia87I}, \cite{Erdelyi}, \cite{Erdelyi+}, \cite{Exton98},
\cite{Faldey}, \cite{Gasper73II}, \cite{Gasper77}, \cite{Gatteschi2002}, \cite{Gawronski87},
\cite{Gawronski93}, \cite{Gillis}, \cite{GillisIsmailOffer}, \cite{Godoy+}, \cite{Grad},
\cite{Hajmirzaahmad95}, \cite{Ismail74}, \cite{Ismail77}, \cite{Ismail2005II},
\cite{IsmailLetVal88}, \cite{IsmailLi}, \cite{IsmailMassonRahman}, \cite{IsmailMuldoon},
\cite{IsmailStanton97}, \cite{IsmailTamhankar}, \cite{Jackson}, \cite{Karlin58},
\cite{Kochneff95}, \cite{Kochneff97II}, \cite{Koekoek2000}, \cite{Koorn77I}, \cite{Koorn78},
\cite{Koorn85}, \cite{Koorn88}, \cite{Krasikov2003}, \cite{Krasikov2005},
\cite{KuijlaarsMcLaughlin}, \cite{KwonLittle}, \cite{LabelleYehI}, \cite{LabelleYehII},
\cite{LabelleYehIII}, \cite{Lee97I}, \cite{Lee97II}, \cite{Lee97III}, \cite{Lesky95II},
\cite{Lesky96}, \cite{LewanowiczI}, \cite{LopezTemme2004}, \cite{Mathai}, \cite{Meixner},
\cite{Nikiforov+}, \cite{NikiforovUvarov}, \cite{Olver}, \cite{PerezPinar}, \cite{Pittaluga},
\cite{Rainville}, \cite{SainteViennot}, \cite{Srivastava66}, \cite{Srivastava69I},
\cite{Srivastava69II}, \cite{Srivastava70}, \cite{Srivastava71}, \cite{Szego75}, \cite{Temme},
\cite{Trickovic}, \cite{Trivedi}, \cite{Viennot}.


\section{Bessel}\index{Bessel polynomials}

\par\setcounter{equation}{0}

\subsection*{Hypergeometric representation}
\begin{eqnarray}
\label{DefBessel}
y_n(x;a)&=&\hyp{2}{0}{-n,n+a+1}{-}{-\frac{x}{2}}\\
&=&(n+a+1)_n\left(\frac{x}{2}\right)^n\,\hyp{1}{1}{-n}{-2n-a}{\frac{2}{x}},
\quad n=0,1,2,\ldots,N.\nonumber
\end{eqnarray}

\subsection*{Orthogonality relation}
\begin{eqnarray}
\label{OrtBessel}
& &\int_0^{\infty}x^a\e^{-\frac{2}{x}}y_m(x;a)y_n(x;a)\,dx\nonumber\\
& &=-\frac{2^{a+1}}{2n+a+1}\Gamma(-n-a)n!\,\delta_{mn},\quad a<-2N-1.
\end{eqnarray}

\subsection*{Recurrence relation}
\begin{eqnarray}
\label{RecBessel}
& &2(n+a+1)(2n+a)y_{n+1}(x;a)\nonumber\\
& &{}=(2n+a+1)\left[2a+(2n+a)(2n+a+2)x\right]y_n(x;a)\nonumber\\
& &{}\mathindent{}+2n(2n+a+2)y_{n-1}(x;a).
\end{eqnarray}

\subsection*{Normalized recurrence relation}
\begin{eqnarray}
\label{NormRecBessel}
xp_n(x)&=&p_{n+1}(x)-\frac{2a}{(2n+a)(2n+a+2)}p_n(x)\nonumber\\
& &{}\mathindent{}-\frac{4n(n+a)}{(2n+a-1)(2n+a)^2(2n+a+1)}p_{n-1}(x),
\end{eqnarray}
where
$$y_n(x;a)=\frac{(n+a+1)_n}{2^n}p_n(x).$$

\subsection*{Differential equation}
\begin{equation}
\label{dvBessel}
x^2y''(x)+\left[(a+2)x+2\right]y'(x)-n(n+a+1)y(x)=0,\quad y(x)=y_n(x;a).
\end{equation}

\subsection*{Forward shift operator}
\begin{equation}
\label{shift1Bessel}
\frac{d}{dx}y_n(x;a)=\frac{n(n+a+1)}{2}y_{n-1}(x;a+2).
\end{equation}

\subsection*{Backward shift operator}
\begin{equation}
\label{shift2BesselI}
x^2\frac{d}{dx}y_n(x;a)+(ax+2)y_n(x;a)=2y_{n+1}(x;a-2)
\end{equation}
or equivalently
\begin{equation}
\label{shift2BesselII}
\frac{d}{dx}\left[x^a\expe^{-\frac{2}{x}}y_n(x;a)\right]=2x^{a-2}\expe^{-\frac{2}{x}}y_{n+1}(x;a-2).
\end{equation}

\subsection*{Rodrigues-type formula}
\begin{equation}
\label{RodBessel}
y_n(x;a)=2^{-n}x^{-a}\expe^{\frac{2}{x}}D^n\left(x^{2n+a}\expe^{-\frac{2}{x}}\right).
\end{equation}

\subsection*{Generating function}
\begin{equation}
\label{GenBessel}
\left(1-2xt\right)^{-\frac{1}{2}}\left(\frac{2}{1+\sqrt{1-2xt}}\right)^a
\exp\left(\frac{2t}{1+\sqrt{1-2xt}}\right)=\sum_{n=0}^{\infty}y_n(x;a)\frac{t^n}{n!}.
\end{equation}

\subsection*{Limit relation}

\subsubsection*{Jacobi $\rightarrow$ Bessel}
If we take $\beta=a-\alpha$ in the definition (\ref{DefJacobi}) of the Jacobi polynomials
and let $\alpha\rightarrow -\infty$ we find the Bessel polynomials:
$$\lim_{\alpha\rightarrow -\infty}
\frac{P_n^{(\alpha,a-\alpha)}(1+\alpha x)}{P_n^{(\alpha,a-\alpha)}(1)}=y_n(x;a).$$

\subsection*{Remarks}
The following notations are also used for the Bessel polynomials:
$$y_n(x;a,b)=y_n(2b^{-1}x;a)\quad\textrm{and}\quad\theta_n(x;a,b)=x^ny_n(x^{-1};a,b).$$
However, the Bessel polynomials essentially depend on only one parameter.

\noindent
The Bessel polynomials are related to the Laguerre polynomials given by (\ref{DefLaguerre})
in the following way:
$$L_n^{(\alpha)}(x)=\frac{(-x)^n}{n!}y_n(2x^{-1};-2n-\alpha-1).$$

\noindent
The special case $a=-2N-2$ of the Bessel polynomials can be obtained from the pseudo Jacobi
polynomials by setting $x\rightarrow\nu x$ in the definition (\ref{DefPseudoJacobi}) of the
pseudo Jacobi polynomials and taking the limit $\nu\rightarrow\infty$ in the following way:
$$\lim\limits_{\nu\rightarrow\infty}\frac{P_n(\nu x;\nu,N)}{\nu^n}
=\frac{2^n}{(n-2N-1)_n}y_n(x;-2N-2).$$

\subsection*{References}
\cite{Andrade}, \cite{BergVignat}, \cite{Carlitz57I}, \cite{Dattoli2003},
\cite{DohaAhmed2004}, \cite{DohaAhmed2006}, \cite{Grosswald}, \cite{Ismail2005II},
\cite{KrallFrink}, \cite{Lesky98}, \cite{NikiforovUvarov}.


\section{Charlier}\index{Charlier polynomials}

\par\setcounter{equation}{0}

\subsection*{Hypergeometric representation}
\begin{equation}
\label{DefCharlier}
C_n(x;a)=\hyp{2}{0}{-n,-x}{-}{-\frac{1}{a}}.
\end{equation}

\subsection*{Orthogonality relation}
\begin{equation}
\label{OrtCharlier}
\sum_{x=0}^{\infty}\frac{a^x}{x!}C_m(x;a)C_n(x;a)=
a^{-n}\expe^an!\,\delta_{mn},\quad a > 0.
\end{equation}

\subsection*{Recurrence relation}
\begin{equation}
\label{RecCharlier}
-xC_n(x;a)=aC_{n+1}(x;a)-(n+a)C_n(x;a)+nC_{n-1}(x;a).
\end{equation}

\subsection*{Normalized recurrence relation}
\begin{equation}
\label{NormRecCharlier}
xp_n(x)=p_{n+1}(x)+(n+a)p_n(x)+nap_{n-1}(x),
\end{equation}
where
$$C_n(x;a)=\left(-\frac{1}{a}\right)^np_n(x).$$

\subsection*{Difference equation}
\begin{equation}
\label{dvCharlier}
-ny(x)=ay(x+1)-(x+a)y(x)+xy(x-1),\quad y(x)=C_n(x;a).
\end{equation}

\subsection*{Forward shift operator}
\begin{equation}
\label{shift1CharlierI}
C_n(x+1;a)-C_n(x;a)=-\frac{n}{a}C_{n-1}(x;a)
\end{equation}
or equivalently
\begin{equation}
\label{shift1CharlierII}
\Delta C_n(x;a)=-\frac{n}{a}C_{n-1}(x;a).
\end{equation}

\subsection*{Backward shift operator}
\begin{equation}
\label{shift2CharlierI}
C_n(x;a)-\frac{x}{a}C_n(x-1;a)=C_{n+1}(x;a)
\end{equation}
or equivalently
\begin{equation}
\label{shift2CharlierII}
\nabla\left[\frac{a^x}{x!}C_n(x;a)\right]=\frac{a^x}{x!}C_{n+1}(x;a).
\end{equation}

\subsection*{Rodrigues-type formula}
\begin{equation}
\label{RodCharlier}
\frac{a^x}{x!}C_n(x;a)=\nabla^n\left[\frac{a^x}{x!}\right].
\end{equation}

\subsection*{Generating function}
\begin{equation}
\label{GenCharlier}
\e^t\left(1-\frac{t}{a}\right)^x=\sum_{n=0}^{\infty}\frac{C_n(x;a)}{n!}t^n.
\end{equation}

\subsection*{Limit relations}

\subsubsection*{Meixner $\rightarrow$ Charlier}
If we take $c=(a+\beta)^{-1}a$ in the definition (\ref{DefMeixner}) of the Meixner polynomials
and let $\beta\rightarrow\infty$ we find the Charlier polynomials:
$$\lim_{\beta\rightarrow\infty}M_n(x;\beta,(a+\beta)^{-1}a)=C_n(x;a).$$

\subsubsection*{Krawtchouk $\rightarrow$ Charlier}
The Charlier polynomials can be found from the Krawtchouk polynomials given by
(\ref{DefKrawtchouk}) by taking $p=N^{-1}a$ and letting $N\rightarrow\infty$:
$$\lim_{N\rightarrow\infty}K_n(x;N^{-1}a,N)=C_n(x;a).$$

\subsubsection*{Charlier $\rightarrow$ Hermite}
The Hermite polynomials given by (\ref{DefHermite}) are obtained from the Charlier polynomials
if we set $x\rightarrow (2a)^{1/2}x+a$ and let $a\rightarrow\infty$. In fact we have
\begin{equation}
\lim_{a\rightarrow\infty}
(2a)^{\frac{1}{2}n}C_n((2a)^{\frac{1}{2}}x+a;a)=(-1)^nH_n(x).
\end{equation}

\subsection*{Remark}
The Charlier polynomials are related to the Laguerre polynomials given by (\ref{DefLaguerre})
in the following way:
$$\frac{(-a)^n}{n!}C_n(x;a)=L_n^{(x-n)}(a).$$
% RS: add begin\label{sec9.12}
\paragraph{\large\bf KLSadd: Notation}Here the Laguerre polynomial is denoted by $L_n^\al$ instead of
$L_n^{(\al)}$.
%
\paragraph{\large\bf KLSadd: Hypergeometric representation}\begin{align}
L_n^\al(x)&=
\frac{(\al+1)_n}{n!}\,\hyp11{-n}{\al+1}x
\label{182}\\
&=\frac{(-x)^n}{n!} \hyp20{-n,-n-\al}-{-\,\frac1x}
\label{183}\\
&=\frac{(-x)^n}{n!}\,C_n(n+\al;x),
\label{184}
\end{align}
where $C_n$ in \eqref{184} is a
\hyperref[sec9.14]{Charlier polynomial}.
Formula \eqref{182} is (9.12.1). Then \eqref{183} follows by reversal
of summation. Finally \eqref{184} follows by \eqref{183} and \eqref{179}.
It is also the remark on top of p.244 in \mycite{KLS}, and it is essentially
\myciteKLS{416}{(2.7.10)}.
%
\paragraph{\large\bf KLSadd: Uniqueness of orthogonality measure}The coefficient of $p_{n-1}(x)$ in (9.12.4) behaves as $O(n^2)$ as $n\to\iy$.
Hence \eqref{93} holds, by which the orthogonality measure is unique.
%
\paragraph{\large\bf KLSadd: Special value}\begin{equation}
L_n^{\al}(0)=\frac{(\al+1)_n}{n!}\,.
\label{53}
\end{equation}
Use (9.12.1) or see \mycite{DLMF}{18.6.1)}.
%
\paragraph{\large\bf KLSadd: Quadratic transformations}\begin{align}
H_{2n}(x)&=(-1)^n\,2^{2n}\,n!\,L_n^{-1/2}(x^2),
\label{54}\\
H_{2n+1}(x)&=(-1)^n\,2^{2n+1}\,n!\,x\,L_n^{1/2}(x^2).
\label{55}
\end{align}
See p.244, Remarks, last two formulas.
Or see \mycite{DLMF}{(18.7.19), (18.7.20)}.
%
\paragraph{\large\bf KLSadd: Fourier transform}\begin{equation}
\frac1{\Ga(\al+1)}\,\int_0^\iy \frac{L_n^\al(y)}{L_n^\al(0)}\,
e^{-y}\,y^\al\,e^{ixy}\,dy=
i^n\,\frac{y^n}{(iy+1)^{n+\al+1}}\,,
\label{14}
\end{equation}
see \mycite{DLMF}{(18.17.34)}.
%
\paragraph{\large\bf KLSadd: Differentiation formulas}Each differentiation formula is given in two equivalent forms.
\begin{equation}
\frac d{dx}\left(x^\al L_n^\al(x)\right)=
(n+\al)\,x^{\al-1} L_n^{\al-1}(x),\qquad
\left(x\frac d{dx}+\al\right)L_n^\al(x)=
(n+\al)\,L_n^{\al-1}(x).
\label{76}
\end{equation}
%
\begin{equation}
\frac d{dx}\left(e^{-x} L_n^\al(x)\right)=
-e^{-x} L_n^{\al+1}(x),\qquad
\left(\frac d{dx}-1\right)L_n^\al(x)=
-L_n^{\al+1}(x).
\label{77}
\end{equation}
%
Formulas \eqref{76} and \eqref{77} follow from
\mycite{DLMF}{(13.3.18), (13.3.20)}
together with (9.12.1). 
%
\paragraph{\large\bf KLSadd: Generalized Hermite polynomials}See \myciteKLS{146}{p.156}, \cite[Section 1.5.1]{K26}.
These are defined by
\begin{equation}
H_{2m}^\mu(x):=\const L_m^{\mu-\half}(x^2),\qquad
H_{2m+1}^\mu(x):=\const x\,L_m^{\mu+\half}(x^2).
\label{78}
\end{equation}
Then for $\mu>-\thalf$ we have orthogonality relation
\begin{equation}
\int_{-\iy}^{\iy} H_m^\mu(x)\,H_n^\mu(x)\,|x|^{2\mu}e^{-x^2}\,dx
=0\qquad(m\ne n).
\label{79}
\end{equation}
Let the Dunkl operator $T_\mu$ be defined by \eqref{72}.
If we choose the constants in \eqref{78} as
\begin{equation}
H_{2m}^\mu(x)=\frac{(-1)^m(2m)!}{(\mu+\thalf)_m}\,L_m^{\mu-\half}(x^2),\qquad
H_{2m+1}^\mu(x)=\frac{(-1)^m(2m+1)!}{(\mu+\thalf)_{m+1}}\,
 x\,L_m^{\mu+\half}(x^2)
 \label{80}
\end{equation}
then (see \cite[(1.6)]{K5})
\begin{equation}
T_\mu H_n^\mu=2n\,H_{n-1}^\mu.
\label{81}
\end{equation}
Formula \eqref{81} with \eqref{80} substituted gives rise to two
differentiation formulas involving Laguerre polynomials which are equivalent to
(9.12.6) and \eqref{76}.

Composition of \eqref{81} with itself gives
\[
T_\mu^2 H_n^\mu=4n(n-1)\,H_{n-2}^\mu,
\]
which is equivalent to the composition of (9.12.6) and \eqref{76}:
\begin{equation}
\left(\frac{d^2}{dx^2}+\frac{2\al+1}x\,\frac d{dx}\right)L_n^\al(x^2)
=-4(n+\al)\,L_{n-1}^\al(x^2).
\label{82}
\end{equation}
%
% RS: add end
\subsection*{References}
\cite{Allaway76}, \cite{NAlSalam66}, \cite{AlSalam90}, \cite{AlSalamChihara76},
\cite{AlSalamIsmail76}, \cite{AndrewsAskey85}, \cite{Area+II}, \cite{Askey75},
\cite{AskeyGasper77}, \cite{AskeyWilson85}, \cite{AtakRahmanSuslov}, \cite{Bavinck98},
\cite{BavinckKoekoek}, \cite{Chihara78}, \cite{Chihara79}, \cite{Dunkl76}, \cite{Erdelyi+},
\cite{Gasper73I}, \cite{Gasper74}, \cite{Goh}, \cite{HoareRahman}, \cite{IsmailLetVal88},
\cite{Koekoek2000}, \cite{Koorn88}, \cite{Krasikov2002}, \cite{LabelleYehI},
\cite{LabelleYehII}, \cite{LabelleYehIII}, \cite{Lesky62}, \cite{Lesky89}, \cite{Lesky94I},
\cite{Lesky95II}, \cite{LewanowiczII}, \cite{LopezTemme2004}, \cite{Meixner}, \cite{Nikiforov+},
\cite{NikiforovUvarov}, \cite{Szafraniec}, \cite{Szego75}, \cite{Viennot}, \cite{Zarzo+},
\cite{Zeng90}.


\section{Hermite}\index{Hermite polynomials}

\par\setcounter{equation}{0}

\subsection*{Hypergeometric representation}
\begin{equation}
\label{DefHermite}
H_n(x)=(2x)^n\,\hyp{2}{0}{-n/2,-(n-1)/2}{-}{-\frac{1}{x^2}}.
\end{equation}

\subsection*{Orthogonality relation}
\begin{equation}
\label{OrtHermite}
\frac{1}{\sqrt{\cpi}}\int_{-\infty}^{\infty}\expe^{-x^2}H_m(x)H_n(x)\,dx
=2^nn!\,\delta_{mn}.
\end{equation}

\subsection*{Recurrence relation}
\begin{equation}
\label{RecHermite}
H_{n+1}(x)-2xH_n(x)+2nH_{n-1}(x)=0.
\end{equation}

\subsection*{Normalized recurrence relation}
\begin{equation}
\label{NormRecHermite}
xp_n(x)=p_{n+1}(x)+\frac{n}{2}p_{n-1}(x),
\end{equation}
where
$$H_n(x)=2^np_n(x).$$

\subsection*{Differential equation}
\begin{equation}
\label{dvHermite}
y''(x)-2xy'(x)+2ny(x)=0,\quad y(x)=H_n(x).
\end{equation}

\subsection*{Forward shift operator}
\begin{equation}
\label{shift1Hermite}
\frac{d}{dx}H_n(x)=2nH_{n-1}(x).
\end{equation}

\subsection*{Backward shift operator}
\begin{equation}
\label{shift2HermiteI}
\frac{d}{dx}H_n(x)-2xH_n(x)=-H_{n+1}(x)
\end{equation}
or equivalently
\begin{equation}
\label{shift2HermiteII}
\frac{d}{dx}\left[\expe^{-x^2}H_n(x)\right]=-\expe^{-x^2}H_{n+1}(x).
\end{equation}

\subsection*{Rodrigues-type formula}
\begin{equation}
\label{RodHermite}
\e^{-x^2}H_n(x)=(-1)^n\left(\frac{d}{dx}\right)^n\left[\expe^{-x^2}\right].
\end{equation}

\subsection*{Generating functions}
\begin{equation}
\label{GenHermite1}
\exp\left(2xt-t^2\right)=\sum_{n=0}^{\infty}\frac{H_n(x)}{n!}t^n.
\end{equation}

\begin{equation}
\label{GenHermite2}
\left\{\begin{array}{l}
\displaystyle\expe^t\cos(2x\sqrt{t})=\sum_{n=0}^{\infty}
\frac{(-1)^n}{(2n)!}H_{2n}(x)t^n\\[5mm]
\displaystyle\frac{\expe^t}{\sqrt{t}}\sin(2x\sqrt{t})=\sum_{n=0}^{\infty}
\frac{(-1)^n}{(2n+1)!}H_{2n+1}(x)t^n.
\end{array}\right.
\end{equation}

\begin{equation}
\label{GenHermite3}
\left\{\begin{array}{l}
\displaystyle \expe^{-t^2}\cosh(2xt)=\sum_{n=0}^{\infty}
\frac{H_{2n}(x)}{(2n)!}t^{2n}\\[5mm]
\displaystyle\expe^{-t^2}\sinh(2xt)=\sum_{n=0}^{\infty}
\frac{H_{2n+1}(x)}{(2n+1)!}t^{2n+1}.
\end{array}\right.
\end{equation}

\begin{equation}
\label{GenHermite4}
\left\{\begin{array}{l}
\displaystyle (1+t^2)^{-\gamma}\,\hyp{1}{1}{\gamma}{\frac{1}{2}}{\frac{x^2t^2}{1+t^2}}=
\sum_{n=0}^{\infty}\frac{(\gamma)_n}{(2n)!}H_{2n}(x)t^{2n}\\[5mm]
\displaystyle\frac{xt}{\sqrt{1+t^2}}\,
\hyp{1}{1}{\gamma+\frac{1}{2}}{\frac{3}{2}}{\frac{x^2t^2}{1+t^2}}
=\sum_{n=0}^{\infty}\frac{(\gamma+\frac{1}{2})_n}{(2n+1)!}H_{2n+1}(x)t^{2n+1}
\end{array}\right.
\end{equation}
with $\gamma$ arbitrary.

\begin{equation}
\label{GenHermite5}
\frac{1+2xt+4t^2}{(1+4t^2)^{\frac{3}{2}}}\exp\left(\frac{4x^2t^2}{1+4t^2}\right)
=\sum_{n=0}^{\infty}\frac{H_n(x)}{\lfloor n/2\rfloor\,!}t^n,
\end{equation}
where $\lfloor n/2\rfloor$ denotes the largest integer smaller than or equal to $n/2$.

\subsection*{Limit relations}

\subsubsection*{Meixner-Pollaczek $\rightarrow$ Hermite}
If we take $x\rightarrow (\sin\phi)^{-1}(x\sqrt{\lambda}-\lambda\cos\phi)$
in the definition (\ref{DefMP}) of the Meixner-Pollaczek polynomials and
then let $\lambda\rightarrow\infty$ we obtain the Hermite polynomials:
$$\lim_{\lambda\rightarrow\infty}
\lambda^{-\frac{1}{2}n}P_n^{(\lambda)}
((\sin\phi)^{-1}(x\sqrt{\lambda}-\lambda\cos\phi);\phi)=\frac{H_n(x)}{n!}.$$

\subsubsection*{Jacobi $\rightarrow$ Hermite}
The Hermite polynomials follow from the Jacobi polynomials given by (\ref{DefJacobi}) by
taking $\beta=\alpha$ and letting $\alpha\rightarrow\infty$ in the following way:
$$\lim_{\alpha\rightarrow\infty}
\alpha^{-\frac{1}{2}n}P_n^{(\alpha,\alpha)}(\alpha^{-\frac{1}{2}}x)=\frac{H_n(x)}{2^nn!}.$$

\subsubsection*{Gegenbauer / Ultraspherical $\rightarrow$ Hermite}
The Hermite polynomials follow from the Gegenbauer (or ultraspherical) polynomials given by
(\ref{DefGegenbauer}) by taking $\lambda=\alpha+\frac{1}{2}$ and letting
$\alpha\rightarrow\infty$ in the following way:
$$\lim_{\alpha\rightarrow\infty}
\alpha^{-\frac{1}{2}n}C_n^{(\alpha+\frac{1}{2})}(\alpha^{-\frac{1}{2}}x)=\frac{H_n(x)}{n!}.$$

\subsubsection*{Krawtchouk $\rightarrow$ Hermite}
The Hermite polynomials follow from the Krawtchouk polynomials given by (\ref{DefKrawtchouk})
by setting $x\rightarrow pN+x\sqrt{2p(1-p)N}$ and then letting $N\rightarrow\infty$:
$$\lim_{N\rightarrow\infty}
\sqrt{\binom{N}{n}}K_n(pN+x\sqrt{2p(1-p)N};p,N)
=\frac{\displaystyle (-1)^nH_n(x)}{\displaystyle\sqrt{2^nn!\left(\frac{p}{1-p}\right)^n}}.$$

\subsubsection*{Laguerre $\rightarrow$ Hermite}
The Hermite polynomials can be obtained from the Laguerre polynomials given by
(\ref{DefLaguerre}) by taking the limit $\alpha\rightarrow\infty$ in the following way:
$$\lim_{\alpha\rightarrow\infty}
\left(\frac{2}{\alpha}\right)^{\frac{1}{2}n}
L_n^{(\alpha)}((2\alpha)^{\frac{1}{2}}x+\alpha)=\frac{(-1)^n}{n!}H_n(x).$$

\subsubsection*{Charlier $\rightarrow$ Hermite}
If we set $x\rightarrow (2a)^{1/2}x+a$ in the definition (\ref{DefCharlier})
of the Charlier polynomials and let $a\rightarrow\infty$ we find the Hermite
polynomials. In fact we have
$$\lim_{a\rightarrow\infty}
(2a)^{\frac{1}{2}n}C_n((2a)^{\frac{1}{2}}x+a;a)=(-1)^nH_n(x).$$

\subsection*{Remarks}
The Hermite polynomials can also be written as:
$$\frac{H_n(x)}{n!}=\sum_{k=0}^{\lfloor n/2\rfloor}
\frac{(-1)^k(2x)^{n-2k}}{k!\,(n-2k)!},$$
where $\lfloor n/2\rfloor$ denotes the largest integer smaller than or equal to $n/2$.

\noindent
The Hermite polynomials and the Laguerre polynomials given by (\ref{DefLaguerre}) are also
connected by the following quadratic transformations:
$$H_{2n}(x)=(-1)^nn!\,2^{2n}L_n^{(-\frac{1}{2})}(x^2)$$
and
$$H_{2n+1}(x)=(-1)^nn!\,2^{2n+1}xL_n^{(\frac{1}{2})}(x^2).$$
% RS: add begin\label{sec9.14}
%
\paragraph{\large\bf KLSadd: Hypergeometric representation}\begin{align}
C_n(x;a)&=\hyp20{-n,-x}-{-\,\frac1a}
\label{179}\\
&=\frac{(-x)_n}{a^n} \hyp11{-n}{x-n+1}a
\label{180}\\
&=\frac{n!}{(-a)^n}\,L_n^{x-n}(a),
\label{181}
\end{align}
where $L_n^\al(x)$ is a
\hyperref[sec9.12]{Laguerre polynomial}.
Formula \eqref{179} is (9.14.1). Then \eqref{180} follows by reversal
of the summation. Finally \eqref{181} follows by \eqref{180} and
(9.12.1). It is also the Remark on p.249 of \mycite{KLS}, and it
was earlier given in \myciteKLS{416}{(2.7.10)}.
%
\paragraph{\large\bf KLSadd: Uniqueness of orthogonality measure}The coefficient of $p_{n-1}(x)$ in (9.14.4) behaves as $O(n)$ as $n\to\iy$.
Hence \eqref{93} holds, by which the orthogonality measure is unique.
%
% RS: add end
\subsection*{References}
\label{sec9.15}
%
\paragraph{\large\bf KLSadd: Uniqueness of orthogonality measure}The coefficient of $p_{n-1}(x)$ in (9.15.4) behaves as $O(n)$ as $n\to\iy$.
Hence \eqref{93} holds, by which the orthogonality measure is unique.
%
\paragraph{\large\bf KLSadd: Fourier transforms}\begin{equation}
\frac1{\sqrt{2\pi}}\,\int_{-\iy}^\iy H_n(y)\,e^{-\half y^2}\,e^{ixy}\,dy=
i^n\,H_n(x)\,e^{-\half x^2},
\label{15}
\end{equation}
see \mycite{AAR}{(6.1.15) and Exercise 6.11}.
\begin{equation}
\frac1{\sqrt\pi}\,\int_{-\iy}^\iy H_n(y)\,e^{-y^2}\,e^{ixy}\,dy=
i^n\,x^n\,e^{-\frac14 x^2},
\label{16}
\end{equation}
see \mycite{DLMF}{(18.17.35)}.
\begin{equation}
\frac{i^n}{2\sqrt\pi}\,\int_{-\iy}^\iy y^n\,e^{-\frac14 y^2}\,e^{-ixy}\,dy=
H_n(x)\,e^{-x^2},
\label{17}
\end{equation}
see \mycite{AAR}{(6.1.4)}.
%
\cite{Abram}, \cite{NAlSalam66}, \cite{AlSalam90}, \cite{AlSalamChihara72},
\cite{AlSalamChihara76}, \cite{AndrewsAskey85}, \cite{AndrewsAskeyRoy}, \cite{Area+I}, 
\cite{Askey68}, \cite{Askey75}, \cite{Askey89I}, \cite{AskeyGasper76}, \cite{AskeyWilson85}, 
\cite{Azor}, \cite{Berg}, \cite{BilodeauII}, \cite{Brafman51}, \cite{Brafman57II}, 
\cite{Brenke}, \cite{CarlitzSrivastava}, \cite{ChenSrivastava}, \cite{Chihara78}, 
\cite{Cohen}, \cite{Danese}, \cite{DetteStudden92}, \cite{DetteStudden95}, \cite{Doha2004I},
\cite{Erdelyi+}, \cite{Faldey}, \cite{Gawronski87}, \cite{Gawronski93}, \cite{Gillis+},
\cite{Godoy+}, \cite{Grad}, \cite{Ismail74}, \cite{Ismail2005II}, \cite{IsmailStanton97},
\cite{Koekoek2000}, \cite{Koorn88}, \cite{Krasikov2004}, \cite{Kwon+}, \cite{LabelleYehIII},
\cite{Lesky95II}, \cite{Lesky96}, \cite{LewanowiczI}, \cite{LopezTemme99}, \cite{Mathai},
\cite{Meixner}, \cite{Nikiforov+}, \cite{NikiforovUvarov}, \cite{Olver}, \cite{Pittaluga},
\cite{Rainville}, \cite{SainteViennot}, \cite{Srivastava71}, \cite{SrivastavaMathur},
\cite{Szego75}, \cite{Temme}, \cite{TemmeLopez2000}, \cite{Trickovic}, \cite{Viennot},
\cite{Weisner59}, \cite{Wyman}.

\end{document}
